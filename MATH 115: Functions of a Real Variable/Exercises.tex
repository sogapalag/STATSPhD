%%%%%%%%%%%%%%%%%%%%%%%%%%%%%%%%%%%%%%%%%
% Short Sectioned Assignment
% LaTeX Template
% Version 1.0 (5/5/12)
%
% This template has been downloaded from:
% http://www.LaTeXTemplates.com
%
% Original author:
% Frits Wenneker (http://www.howtotex.com)
%
% License:
% CC BY-NC-SA 3.0 (http://creativecommons.org/licenses/by-nc-sa/3.0/)
%
%%%%%%%%%%%%%%%%%%%%%%%%%%%%%%%%%%%%%%%%%

%----------------------------------------------------------------------------------------
%	PACKAGES AND OTHER DOCUMENT CONFIGURATIONS
%----------------------------------------------------------------------------------------

\documentclass[paper=a4, fontsize=11pt]{scrartcl} % A4 paper and 11pt font size

\usepackage[T1]{fontenc} % Use 8-bit encoding that has 256 glyphs
\usepackage{fourier} % Use the Adobe Utopia font for the document - comment this line to return to the LaTeX default
\usepackage[english]{babel} % English language/hyphenation
\usepackage{amsmath,amsfonts,amsthm} % Math packages

\usepackage{sectsty} % Allows customizing section commands
\allsectionsfont{\centering \normalfont\scshape} % Make all sections centered, the default font and small caps
\usepackage{url}
\usepackage{fancyhdr} % Custom headers and footers
\pagestyle{fancyplain} % Makes all pages in the document conform to the custom headers and footers
\fancyhead{} % No page header - if you want one, create it in the same way as the footers below
\fancyfoot[L]{} % Empty left footer
\fancyfoot[C]{} % Empty center footer
\fancyfoot[R]{\thepage} % Page numbering for right footer
\renewcommand{\headrulewidth}{0pt} % Remove header underlines
\renewcommand{\footrulewidth}{0pt} % Remove footer underlines
\setlength{\headheight}{13.6pt} % Customize the height of the header

\numberwithin{equation}{section} % Number equations within sections (i.e. 1.1, 1.2, 2.1, 2.2 instead of 1, 2, 3, 4)
\numberwithin{figure}{section} % Number figures within sections (i.e. 1.1, 1.2, 2.1, 2.2 instead of 1, 2, 3, 4)
\numberwithin{table}{section} % Number tables within sections (i.e. 1.1, 1.2, 2.1, 2.2 instead of 1, 2, 3, 4)

\setlength\parindent{0pt} % Removes all indentation from paragraphs - comment this line for an assignment with lots of text


\title{Exercises}

\author{sogapalag}

\date{\normalsize\today}

\begin{document}

\maketitle
\begin{itemize}
\item[12.12] \begin{itemize}
		\item[(a)]  since $L=\lim\sup s_n = \lim_{N\rightarrow \infty}\sup\{s_n:n>N\} <L_1$, there exists a positive integer $N$ s.t. $\sup\{s_n:n\geq N\}<L_1$, thus $s_n<L_1, \forall n\geq N$, then for $\forall n>M>N$, note $L_1>0$, \begin{equation}\begin{split}
			\sigma_n &= \frac{s_1+\dots+s_n}{n} \\
					&= \frac{s_1+\dots+s_N}{n} + \frac{s_{N+1}+\dots+s_n}{n} \\
					&< \frac{s_1+\dots+s_N}{M} + L_1 \frac{n-N}{n} \\
					&< \frac{s_1+\dots+s_N}{M} + L_1
		\end{split}\end{equation}
		note that $\forall L_1>L$, i.e. \begin{equation}
			\sup\{\sigma_n:n>M\} \leq \frac{s_1+\dots+s_N}{M} + \sup\{s_n:n>N\}
		\end{equation}
		hence \begin{equation}
			\lim \sup \sigma_n = \lim_{M\rightarrow\infty}\sup\{\sigma_n:n>M\} \leq \lim\sup s_n
		\end{equation}\qed
		\item[(b)] %by (a), if $\lim s_n$ exists, equals to some real, easily by thm. For $\lim s_n = +\infty$, i.e. for each $2M>0$, $\exists N$ s.t. $n>N$ implies $s_n>2M$. We let $n>2N$, then \begin{equation}\begin{split}
			%\sigma_n &= \frac{s_1+\dots+s_N}{n} + \frac{s_{N+1}+\dots+s_n}{n}\\
			%&> 2M(\frac{n-N}{n})\\
			%&> 2M(\frac{1}{2})\\
			%&= M
		%\end{split}\end{equation}
		%thus $\lim \sigma_n = +\infty = \lim s_n$.\qed
		by thm 10.7 and (a).\qed
		\item[(c)] $(s_n)=(1,-1,1,-1,...)$, set $N=1/\epsilon$ easily to prove $\lim \sigma_n=0$, however $(s_n)$ diverge.\qed
	\end{itemize}
\item[12.14] \begin{itemize}
	\item[(a)] let $s_n = n!$, then by thm,
		\begin{equation}\begin{split}
			\lim (n!)^{1/n} &= \lim \frac{s_{n+1}}{s_n} \\
			&= \lim (n+1) \\ 
			&= +\infty
		\end{split}\end{equation}
	\item[(b)] let $s_n = n!/n^n$, then by thm,
		\begin{equation}\begin{split}
			\lim \frac{1}{n}(n!)^{1/n} &=  \lim (\frac{n!}{n^n})^{1/n}\\
			&= \lim (s_n)^{1/n}\\
			&= \lim \frac{s_{n+1}}{s_n} \\ 
			&= \lim \frac{1}{(1+\frac{1}{n})^n} \\
			&= e^{-1}
		\end{split}\end{equation}
	\end{itemize}
\item[14.7] since $\sum a_n$ converges, that $\lim a_n=0$, i.e. $\exists N$, $a_n<1, \forall n>N$, then $\sum_{N+1} a_n^p <\sum_{N+1} a_n$, by comparison test, converges.\qed
\item[14.10] consider $\sum a_n = 1 + 1 + 2 + 2+ 4+ 4+ \dots+ 2^{\lfloor n/2\rfloor}+\dots$, then $\lim\inf|\frac{a_{n+1}}{a_n}| = 1$, connot use ratio test, however by the root test, $\lim a_n^{1/n}=\lim\sup |a_n|^{1/n} = \sqrt{2}>1$.
\item[14.12] $\lim\inf|a_n| = 0$, implies, $\forall \epsilon, \exists N$, when $n>N$ s.t.
\begin{equation}
-\epsilon < \inf\{|a_n|: n>N\} <\epsilon
\end{equation}
i.e. $|a_n|>-\epsilon,\forall n>N$, and $\exists m>N$ that $|a_m|<\epsilon$. So set $\epsilon=k,k/2,k/4,...$, $k>0$, corresponding to $N_1,N_2,N_3,...$, after choose $m_1$, $n_{j} = \max\{m_j, N_j\}$, consider $\inf\{|a_n|: n>n_{j}\}$, there can select some $m>n_{j}$ make $|a_m|<k/2^{j}$, otherwise, $\inf>0$. Then we get a subsequence $\sum a_{m_j} < 2k$ converges.\qed
\item[15.7] by Cauchy criterion, $|a_{N+1}+\dots+ a_n| <\epsilon/2$, let $n>2N$, then $n(a_n)\leq 2 (a_{N+1}+\dots+ a_n) < \epsilon$. i.e. $\lim na_n = 0$.\qed
\item[23.5] \begin{itemize}
	\item[(a)] when $|x|>1$, suppose converges to $A$, set $\epsilon= 1$, then $\forall N$, since there are infinite nonzero, then $n,m$ with $a_{n+1},...,a_{m-1}$ zero,(if not, $m=n+1$), $|S_m-S_n| = |a_m x^n|>|a_m|\geq 1$, i.e. diverge. Thus $|R|\leq 1$.\qed
	\item[(b)] since $\lim\sup|a_n|>0$, that $\exists 0<c<\sup\{|a_n|:n>N\}, \forall N$, i.e. there is a subsequence $(a_{n_k})$ with $|a_{n_k}|>c,\forall k$, so $\lim\sup |a_{n_k}|^{1/n_k} \geq 1$, so $\beta =\lim\sup|a_n|^{1/n}\geq 1$, $R\leq 1$.\qed
	\end{itemize}
\item[23.6]\begin{itemize}
	\item[(a)] since $a_n\geq 0$
	\begin{equation}\begin{split}
	|a_m (-R)^m+a_{m+1}(-R)^{m+1}+\dots+a_n (-R)^n| &\leq |a_m R^m+a_{m+1}R^{m+1}+\dots+a_n R^n)|\\
	&<\epsilon
	\end{split}
	\end{equation}
	i.e. still Cauchy sequence.\qed
	\item[(b)] $a_n = -1/n$.
	\end{itemize}
\item[24.5] \begin{itemize}
	\item[(a)] $f(x)=0, x\leq 1$, $f(x)=1, x>1$
	\item[(b)] yes, $\lim\sup|f_n(x)|=\lim \frac{1}{n+1} = 0$.
	\item[(c)] no, $\lim\sup|f_n(x)-f(x)|= \lim\frac{n}{n+1} =1\neq 0$
	\end{itemize}
\item[24.7] \begin{itemize}
	\item[(a)] $f(x)=x, 0\leq x<1$, $f(x)=0, x=1$, converge pointwise.
	\item[(b)] no, $\lim\sup|f_n(x)-f(x)| = \lim \sup |x^n| = \lim 1 = 1\neq 0$. 
	\end{itemize}
\item[24.10] note $|f_n+g_n - f-g|\leq |f_n-f| + |g_n -g|, \forall x$.\qed\
\item[24.13] let $n>N$ be the implies $|f(x)-f_n(x)|<\epsilon/3$, let $\delta>|x-y|$ be the implies $|f_n(x)-f_n(y)|<\epsilon/3$. Then
\begin{equation}\begin{split}
	|f(x)-f(y)| &\leq |f(x)-f_n(x)| + |f_n(x)-f_n(y)| + |f(y)-f_n(y)| \\
	&< \epsilon/3 + \epsilon/3 + \epsilon/3 \\
	&= \epsilon
	\end{split}
	\end{equation}
	i.e. $f$ is uniformly continuous.\qed
\item[24.17] let $n>N$ be the implies $|f_n(x)-f(x)|<\epsilon/2$, $x_n\rightarrow x$ implies $|f_n(x_n)-f_n(x)|<\epsilon/2$. Then
\begin{equation}\begin{split}
	|f_n(x_n) - f(x)| &\leq |f_n(x_n)-f_n(x)| + |f_n(x)-f(x)| \\
	 &< \epsilon/2 + \epsilon/2 \\
	 &= \epsilon
\end{split}
	\end{equation}
i.e. $\lim_{n\rightarrow\infty} f_n(x_n) = f(x)$.\qed
\item[25.4] $\lim\sup$
\begin{equation}\begin{split}
	|f_n - f_m| &\leq |f_n -f| + |f_m -f| \\
	&< \epsilon/2 + \epsilon/2 \\
	&= \epsilon
\end{split}
	\end{equation}\qed
\item[25.7] by M-test, $\leq 1/n^2$.\qed
\item[25.8] $2(n+1)/n \rightarrow 2=R$; then still by M-test.\qed
\item[25.9] \begin{itemize}
	\item[(a)] by M-test, $\leq a^n$, and $\sum a^n<\infty$.
	\item[(b)] \begin{equation}\begin{split}
		|f_n - f | &= |\frac{1-x^{n+1}}{1-x} - \frac{1}{1-x}| \\
			& = |\frac{x^{n+1}}{1-x}|
	\end{split}
	\end{equation}
	thus $\sup\rightarrow \infty$, so not converge uniformly.
	\end{itemize}
\item[25.15]\begin{itemize}
	\item[(a)] first note $f_n(x)\geq 0, \forall x,\forall n$, otherwise by decreasing, $f_n$ connot pointwise to zero in which point; then note that $\sup|f_n| = \sup f_n$ is decreasing as well, if not uniformly to zero, means $\exists \epsilon>0$, that $\sup f_n\geq \epsilon$, so $\exists x_n$ that $f_n(x_n)\geq \epsilon, \forall n$, note $x_n\in[a,b]$ bounded, by Bolzano-Weierstrass theorem, there is subsequence $(x_{n_k})$ converges to some $x_0$. Since pointwise, i.e. $\exists m$ s.t. $f_m(x_0)<\epsilon$, by continuity of $f_m$, $\exists K$, that when $k>K$, $f_m(x_{n_k})<\epsilon$. Then $f_{n_k}(x_{n_k}) \leq f_m(x_{n_k})<\epsilon$, contradiction.\qed 
	\item[(b)] consider $f-f_n$, apply (a).\qed
	\end{itemize}
\end{itemize}
\end{document}