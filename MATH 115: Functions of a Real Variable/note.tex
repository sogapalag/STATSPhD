%%%%%%%%%%%%%%%%%%%%%%%%%%%%%%%%%%%%%%%%%
% Short Sectioned Assignment
% LaTeX Template
% Version 1.0 (5/5/12)
%
% This template has been downloaded from:
% http://www.LaTeXTemplates.com
%
% Original author:
% Frits Wenneker (http://www.howtotex.com)
%
% License:
% CC BY-NC-SA 3.0 (http://creativecommons.org/licenses/by-nc-sa/3.0/)
%
%%%%%%%%%%%%%%%%%%%%%%%%%%%%%%%%%%%%%%%%%

%----------------------------------------------------------------------------------------
%	PACKAGES AND OTHER DOCUMENT CONFIGURATIONS
%----------------------------------------------------------------------------------------

\documentclass[paper=a4, fontsize=11pt]{scrartcl} % A4 paper and 11pt font size

\usepackage[T1]{fontenc} % Use 8-bit encoding that has 256 glyphs
\usepackage{fourier} % Use the Adobe Utopia font for the document - comment this line to return to the LaTeX default
\usepackage[english]{babel} % English language/hyphenation
\usepackage{amsmath,amsfonts,amsthm} % Math packages

\usepackage{sectsty} % Allows customizing section commands
\allsectionsfont{\centering \normalfont\scshape} % Make all sections centered, the default font and small caps
\usepackage{url}
\usepackage{fancyhdr} % Custom headers and footers
\pagestyle{fancyplain} % Makes all pages in the document conform to the custom headers and footers
\fancyhead{} % No page header - if you want one, create it in the same way as the footers below
\fancyfoot[L]{} % Empty left footer
\fancyfoot[C]{} % Empty center footer
\fancyfoot[R]{\thepage} % Page numbering for right footer
\renewcommand{\headrulewidth}{0pt} % Remove header underlines
\renewcommand{\footrulewidth}{0pt} % Remove footer underlines
\setlength{\headheight}{13.6pt} % Customize the height of the header

\numberwithin{equation}{section} % Number equations within sections (i.e. 1.1, 1.2, 2.1, 2.2 instead of 1, 2, 3, 4)
\numberwithin{figure}{section} % Number figures within sections (i.e. 1.1, 1.2, 2.1, 2.2 instead of 1, 2, 3, 4)
\numberwithin{table}{section} % Number tables within sections (i.e. 1.1, 1.2, 2.1, 2.2 instead of 1, 2, 3, 4)

\setlength\parindent{0pt} % Removes all indentation from paragraphs - comment this line for an assignment with lots of text


\title{MATH 115: Functions of a Real Variable}

\author{sogapalag}

\date{\normalsize\today}

\begin{document}

\maketitle

\section{Introduction}
Peano Axioms, basis of mathematical induction.\\
algebraic number, integer polynomial $p(x)=0$. $c_n\neq 0,\ n\geq 1$.\\
Rational Zeros Theorem: c divides $c_0$, d divedes $c_n$.\\
corollary, $x^n+c_{n-1}x^{n-1}+...+c_1x+c_0=0$, Any rational solution of this equation must be an integer that divides $c_0$\\
Field and order def and properties.\\
Def, max, min, upper bound, lower bound;\\
supremum sup S, if S is bounded above and S has a least upper bound;\\
infimum inf S, If S is bounded below and S has a greatest lower bound.\\
Completeness Axiom: Every nonempty subset S of R that is bounded above has a least upper bound. In other words, sup S exists and is a real number.\\
$-\infty,+\infty$ symbol for unbounded set. e.g. $\sup S=+\infty$\\
Archimedean Property. if $a,b>0$, then for some positive integer n, $na>b$.\\
Denseness of $\mathbb{Q}$, if $a,b\in\mathbb{R}$ and $a<b$, then $\exists r\in\mathbb{Q}$, s.t. $a<r<b$.\\

\section{Sequences}
%s_n足够逼近s
def of sequence, limit, converge, diverge. $\epsilon, N$ notation.\\
Thm, Convergent sequences are bounded.\\
Thm, $\lim(ks_n) = k\lim s_n$.\\
Thm, $\lim(s_n+t_n)=\lim s_n+ \lim t_n$.\\
Thm, $\lim(s_nt_n)=(\lim s_n) (\lim t_n)$.\\
lemma, if $s_n, s\neq 0$, then $\lim (1/s_n) = 1/s$.\\
Thm, $\lim(t_n/s_n)=t/s$.\\
Thm, positive real numbers, $\lim s_n=+\infty$ iff $\lim(1/s_n) =0$.\\
monotone sequence; increasing or decreasing\\
%单调确界隐含着逼近
Thm, All bounded monotone sequences converge.\\
Def, $\lim \sup s_n = \lim_{N\rightarrow\infty} sup\{s_n:n>N\}$.\\
Thm, if $\lim s_n$ defined, then $\lim \inf s_n=\lim s_n =\lim \sup s_n$; if $\lim \inf s_n=\lim \sup s_n$, then $\lim s_n$ is defined and $\lim s_n =\lim \sup s_n$.\\
%s_n与s_m足够逼近
%Cauchy 可以不知道limit,只要求“后面的”无穷项<e
Cauchy sequence, $\forall\epsilon, \exists N$, s.t. $m,n>N$ implies $|s_n-s_m|<\epsilon$.\\
lemma, Convergent sequences are Cauchy sequences.\\
lemma, Cauchy sequences are bounded.\\
Thm, Convergent sequences iff Cauchy sequences.\\
def, subsequence, 'select', 'in order', 'infinite';\\
%可挑出(甚至单调)子序列 当逼近集无限时收敛 or 无界时无穷极限
Thm, there is a subsequence of $(s_n)$ converging to $t$ iff set $\{n\in\mathbb{N}: |s_n-t|<\epsilon\}$ is infinite for all $\epsilon>0$. if unbounded above(below), it has a subsequence has limit $\infty$. in each case, the subsequence can be taken to be monotonic.\\
Thm, if $(s_n)$ converges, then every subsequence converges to the same limit.\\
%可挑出单调子序列
Thm, Every $(s_n)$ has a monotonic subsequence.\\
%有界,可挑出收敛子序列
%这个定理可从前面证得单调收敛,不过这个定理更广义
Bolzano-Weierstrass Theorem. Every bounded sequence has a convergent subsequence.\\
%可挑出单调收敛子序列,极限为确界极限
Thm, any $(s_n)$, There exists a monotonic subsequence whose limit is $\lim\sup s_n$, and there exists a monotonic subsequence whose limit is $\lim\inf s_n$.\\
Thm, S denote the set of subsequential limits, then $S$ is nonempty; $\sup S=\lim\sup s_n$ and $\inf S=\lim\inf s_n$; $\lim s_n$ exists iff $S$ has exactly one element.\\
Thm, S as above, suppose $(t_n)$ is a sequence in $S\cap \mathbb{R}$ and $t=\lim t_n$. Then $t\in S$.\\
Thm, if $\lim s_n=s>0$, then $\lim\sup s_nt_n = s\lim\sup t_n$, allow $\infty$.\\
%确界极限的定义蕴含着确界向内单调逼近,于是变化率严格小于,常量极限下,于是变\leq
Thm, $(s_n)$ any sequence of nonzero real numbers. Then
\begin{equation}
\lim\inf|\frac{s_{n+1}}{s_n}| \leq \lim\inf |s_n|^{1/n} \leq \lim\sup |s_n|^{1/n} \leq \lim\sup |\frac{s_{n+1}}{s_n}|
\end{equation}
Corollary, if $\lim |\frac{s_{n+1}}{s_n}|$ exists (equals L), then $\lim|s_n|^{1/n}$ exists (equals L).\\
Series.\\
$\sum_{k=0}^n ar^k = a\frac{1-r^{n+1}}{1-r}$, $r\neq 1$; $\sum_{n=0}^\infty ar^n = \frac{a}{1-r}$ if $|r|<1$\\
$\sum \frac{1}{n^p}$ converges iff $p>1$.\\
Def, a series $\sum a_n$ satisfies the Cauchy criterion if its sequence $(s_n)$ of partial sums is a Cauchy sequence.\\
Thm, A series converges if and only if it satisfies the Cauchy criterion.\\
Corollary, if a series $\sum a_n$ converges, then $\lim a_n = 0$.\\
%注意被比较的极数 a_n需要非负
Comparison Test, $a_n\geq 0, \forall n$: if $\sum a_n$ converges and $|b_n|\leq a_n,\forall n$, then $\sum b_n$ converges; if $\sum a_n = +\infty$ and $b_n\geq a_n, \forall n$, then $\sum b_n=+\infty$.\\
corollary, Absolutely convergent series are convergent.\\
%保证了存在epsilon,使得后续变化率<1-epsilon
Ratio Test, $\sum a_n$ nonzero terms: converges absolutely if $\lim\sup|a_{n+1}/a_n|<1$; diverges if $\lim\inf|a_{n+1}/a_n|>1$.\\
%更紧,所以当更松的增长率极限为1时,使得确界极限夹逼,该TEST也失效
Root Test, let $\alpha = \lim\sup |a_n|^{1/n}$: converges absolutely if $\alpha<1$; diverges if $\alpha >1$.\\
Integral Test.
consider $\lim \int_1^n f(x)dx$. advisa nonnegative, decreasing f.\\
Alternating Series Theorem: if $a_n$ decreasing, and $\lim a_n=0$, then the alternating series $\sum (-1)^{n+1} a_n$ converges.\\
Thm, A real number x has exactly one decimal expansion or else x has two decimal expansions, one ending in a sequence of all 0's and the other ending in a sequence of all 9's.\\
Thm, rational iff its decimal expansion is repeating.

\section{Continuity}
%默认定义在R上
%连续,函数值随着x逼近而逼近
continuous, $\epsilon$-$\delta$ notation.\\
Thm, $|f|,kf$ continuous at $x_0$ if $f$ does.\\
Thm, $f+g, fg, f/g$ so does.\\
Thm, if $f$ continuous at $x_0$, g at $f(x_0)$, then $g\circ f$ at $x_0$.\\
bounded, $|f(x)|\leq M$.\\
Thm, if f continuous on $[a,b]$, then bounded. moreover, maximum and minimum on $[a,b]$.\\
%画连续线(即不抬笔)必穿过介值
Intermediate Value Theorem: if continuous on I: $y$ between $(a),f(b)$, $\exists x\in(a,b)$, s.t. $f(x)=y$.\\
corollay, continuous on I, then $f(I)$ is interval or single point.\\
Thm, f continuous strictly increasing on I, then $f^{-1}$ continuous strictly increasing on $J=f(I)$.\\
%一直朝上画,y投影铺满了区间,故必连续
Thm, if g strictly increasing on J, s.t. $g(J)$ is interval. Then g continuous.\\
%连续画线若折返则有同值
Thm, Let f be a one-to-one continuous function on an interval I. Then f is strictly increasing or strictly decreasing.\\
%值随着x,y逼近而逼近; 与普通连续区别,delta尺子不依赖点,只依赖epsilon
%如1/x, 对于靠近0的点,总是可以普通连续逼近;但没有不变的delta尺子使得任意两点的度量小于epsilon
uniformly continuous;\\
Thm, continuous on $[a,b]$ $\Rightarrow$ uniformly continuous on $[a,b]$.\\
Thm, If f is uniformly continuous on a set S and $(s_n)$ is a Cauchy sequence in S, then $(f(s_n))$ is a Cauchy sequence.\\
%确定斜率(导数)非无穷了,尺子比例也限定了
%注,导数无穷,尺子比依然可能限定,如sqrt{x}
Thm, f on $(a,b)$ is uniformly continuous on $(a,b)$ iff it can be extended to a continuous function $\tilde{f}$ on $[a,b]$.\\
Thm, Let f be a continuous function on I. Let $I^\circ$ removing endpoints, if f differentiable on $I^\circ$ and $f'$ bounded on $I^\circ$, then $f$ uniformly continuous on I.\\
Mean Value Theorem: continuous on $[a,b]$, differentiable on $(a,b)$ then $\exists x\in(a,b)$, s.t. $f'(x) = \frac{f(b)-f(a)}{b-a}$\\
limit, as x tends to a along S of $f(x)$, i.e. $\lim_{x\rightarrow a^S} f(x)$.\\
by this def, $f$ is continuous at $a$ iff $\lim_{x\rightarrow a^S} f(x) = f(a)$.\\
left(right)-hand limit, $\lim_{x\rightarrow a^-} f(x)$, $\lim_{x\rightarrow a^+} f(x)$.\\
Thm, f defined on $J/\{a\}$ for some open interval $J$ containing a. Then $\lim_{x\rightarrow}f(x)$ exists iff $\lim_+$ and $\lim_-$ both exist and equal, i.e. THREE limits equal.\\
Prop, on $[a,b], $$\gamma(t) = (f_j(t),...)$ is continuous iff each $f_j$ is continuous.\\
%开集蕴涵着边界的接近和拓扑可区分性,故定义了连续
Thm, metric space, a function is continuous iff $f^{-1}(U)$ is open $\forall $ open $U$.\\
Thm, if f continuous, E compact. Then $f(E)$ is compact, $f$ uniformly continuous on E. 

\section{Sequences and Series of Functions}
Power Series, $\sum_{n=0}^\infty a_n x^n$.\\
Thm, let $\beta= \lim\sup |a_n|^{1/n}$ and $R=1/\beta$. (set $R=+\infty$ for $\beta=0$, $R=0$ for $\beta=+\infty$). Then the power series converges for $|x|<R$, diverges for $|x|>R$.\\
R is called radius of convergence.\\
%逐点收敛N(e,x);一致收敛N(e)
converges pointwise, $\lim_{n}\rightarrow\infty f_n(x)=f(x), \forall x\in S$.\\
formal def, $\forall \epsilon>0, \forall x\in S, \exists N$, s.t. $|f_n(x)-f(x)|<\epsilon$, when $n>N$.\\
converges uniformly, $\forall \epsilon>0, \exists N$, s.t. $|f_n(x)-f(x)|<\epsilon, \forall x\in S$, when $n>N$.\\
Thm, The uniform limit of continuous functions is continuous. $f_n\rightarrow f$ uniformly on S, if each $f_n$ continuous, $f$ continuous.\\
$f_n\rightarrow f$ uniformly on S  $\Leftrightarrow$ $\lim_{n\rightarrow\infty} \sup\{|f(x)-f_n(x)|:x\in S\} = 0$.\\
Thm, on $[a,b]$, each $(f_n)$ continuous, $f_n\rightarrow f$ uniformly. Then
\begin{equation}
\lim_{n\rightarrow \infty}\int_a^b f_n(x) dx = \int_a^b f(x) dx
\end{equation}
%一致收敛的Cauchy表述
uniformly Cauchy, $\forall\epsilon>0, \exists N$, s.t. $|f_n(x)-f_m(x)|<\epsilon, \forall x\in S$, and $m,n>N$.\\
uniformly Cauchy $\Rightarrow$ $\exists f$ s.t. $f_n\rightarrow f$ uniformly.\\
Thms for sequences of functions translate eaily into Thms for series of functions.\\
Weierstrass M-test: $M_k\geq 0$, $\sum M_k<\infty$. If $|g_k(x)|\leq M_k,\forall x\in S$, then $\sum g_k$ converges uniformly on S.\\
If $\sum g_n$ converges uniformly, then $\lim\sup|g_n(x)| = 0$\\
Thm, If $0<d<R$, then power series converges uniformly on $[-d,d]$.\\
Corollay, power series converges to a continuous function on $(-R, R)$.\\
Lemma, $\sum na_nx^{n-1}$ and $\sum \frac{a_n}{n+1} x^{n+1}$ has radiu of convergence $R$, i.e. differentiate and integral hold radiu.\\
Thm, $\int_0^x f(t)dt = \sum \frac{a_n}{n+1} x^{n+1} $, for $|x|<R$.\\
Thm, $f'(x) = \sum na_nx^{n-1}$ for $|x|<R$.\\
Abel'S Theorem: $0<R<\infty$: if converges at $x=R$, then $f$ continuous at $x=R$; if $-R$, then $-R$.\\
Bernstein polynomials, let $f$ on $[0,1]$, $B_nf$ defined as
\begin{equation}
B_nf(x)  = \sum_{k=0} ^n f(\frac{k}{n})\dot {n \choose k} x^k(1-x)^{n-k}
\end{equation}
Bernstein’s version of the Weierstrass approximation theorem, $\forall$ continuous function $f$ on $[0,1]$, we have:
\begin{equation}
B_nf\rightarrow f\ \ uniformly\ on\ [0,1]
\end{equation}
Weierstrass’s Approximation Theorem. Every continuous function on a closed interval $[a,b]$ can be uniformly approximated by polynomials on $[a,b]$.\\

\section{Differentiation}
differentiable, open interval contain $a$, if $\lim_{x\rightarrow a} \frac{f(x)-f(a)}{x-a}$ exists and finite.\\
Thm, $f$ differentiable at $a$, $\Rightarrow$ $f$ continuous at $a$.\\
Rolle's Theorem, special case $f(a)=f(b)$ of Mean Value Theorem.\\
Mean Value Theorem: continuous on $[a,b]$, differentiable on $(a,b)$. Then $\exists x\in(a,b)$ s.t. $f'(x) = \frac{f(b)-f(a)}{b-a}$.\\
Intermediate Value Theorem for Derivatives.\\
Thm, one-to-one continuous on open interval I, $J=f(I)$. if $f$ differentiable at $x_0$, then $f^{-1}$ differentiable at $y_0=f(x_0)$ and $(f^{-1})'(y_0)  =  \frac{1}{f'(x_0)}$.\\
Generalized Mean Value Theorem: f, g, continuous on $[a,b]$, differentiable on $(a,b)$, then $\exists x\in(a,b)$ s.t.
\begin{equation}
f'(x)[g(b) - g(a)] =  g'(x) [f(b) - f(a)]
\end{equation}
L'Hospital's Rule.\\
Taylor's Theorem. $c\in (a,b)$, there is some y between c and x, s.t.
\begin{equation}
R_n(x) = \frac{f^{(n)}(y)}{n!} (x-c)^n
\end{equation}
Coraollary, if all $f^{(n)}$ exists and bounded by a single constant C, then
\begin{equation}
\lim_{n\rightarrow \infty}R_n(x) = =0,\ \forall x\in(a,b)
\end{equation}
there is integral form.\\
Binomial Series Theorem.\\
Newton's Method.\\
Secant Method.\\

\section{Integration}
notation, let $f$ bounded on $[a,b]$. For $S\subset [a,b]$:\\
$M(f,S)=\sup\{f(x):x\in S\}$ and $m(f,S) = \inf\{f(x):x\in S\}$\\
partition $P = \{a=t_0<t_1<\dots<t_n =b\}$\\
%Darboux integral
upper Darboux sum $U(f,P) = \sum M(f, [t_{k-1},t_k])\cdot (t_k-t_{k-1})$\\
lower Darboux sum $L(f,P) = \sum m(f, [t_{k-1},t_k])\cdot (t_k-t_{k-1})$\\
upper Darboux integral $U(f) = \inf \{U(f,P): P is a partition of [a,b]\}$\\
lower Darboux integral $L(f) = \sup \{L(f,P): P is a partition of [a,b]\}$\\
def, integrable, $L(f)=U(f)$. (called Darboux integral)\\
lemma, bounded, if $P\subset Q$, $L(f,P)\leq L(f,Q)\leq U(f,Q)\leq U(f,P)$\\
lemma, bounded, $L(f,P)\leq U(f,Q)$; proof consider $P\cup Q$\\
Thm, bounded, $L(f)\leq U(f)$\\
Thm, bounded: integrable iff $\forall\epsilon>0, \exists P$, s.t. $U(f,P) - L(f,P)<\epsilon$\\
def, $mesh(P) = \max\{t_k - t_(k-1) : k=1,2,...,n\}$\\
Thm, bounded: integrable iff $\forall\epsilon>0, \exists \delta>0$, s.t. $mesh(P)<\delta$ implies $U(f,P) - L(f,P)<\epsilon, \forall P$\\
%Riemann integral
Riemann sum, $\sum f(x_k)(t_k - t_{k-1}), x_k\in[t_{k-1}, t_k]$.\\
Riemann integrable, $\forall \epsilon>0, \exists \delta>0$, s.t. $|S-r|<\epsilon$, $\forall S$ associated with $P$ have $mesh(P)<\delta$.\\
Thm, bounded, Riemann integrable $\Leftrightarrow$ Darboux integrable.\\
Corollay, bounded Riemann integrable, $\lim mesh(P_n)=0$, then $(S_n)$ converges to $r=\int_a^b f$.\\
Thm, Every monotonic $f$ on $[a,b]$ is integrable.\\
Thm, Every continuous $f$ on $[a,b]$ is integrable.\\
Thm, integrable $f$ $\Rightarrow$ integrable $|f|$, and $|\int_a^b f| \leq \int_a^b|f|$.
%可分割性
Thm, $a<c<b$, integrable on $[a,c]$ and $[c,b]$ $\Rightarrow$ integrable on $[a,b]$\\
def, piecewise monotonic, $\exists P$, s.t. monotonic on each interval $()$.\\
def, piecewise continuous, $\exists P$, s.t. uniformly continuous on each interval $()$.\\
Thm, piecewise continuous or bounded piecewise monotonic $\Rightarrow$ integrable.\\
Intermediate Value Theorem for Integrals. continuous on $[a,b]$, $\exists x\in(a,b)$ $f(x) = \frac{1}{b-a} \int_a^b f $.\\
%dominated, 积分极限等于极限积分
Dominated Convergence Theorem. integrable $(f_n)$ pointwise to integrable $f$. If $\exists M>0$, s.t. $|f_n(x)\leq M|, \forall n\forall x\in[a,b]$, then
\begin{equation}
\lim_{n\rightarrow \infty} \int_a^b f_n(x)dx = \int_a^b \lim_{n\rightarrow \infty} f_n(x) dx
\end{equation}
corol, Monotone Convergence Theorem. pf. monotone $(f_n)$ implies $\exists M$.\\
Fundamental Theorem of Calculus I. continuous on $[a,b]$, differentiable on $(a,b)$, $g'$ integrable on $[a,b]$, then $\int_a^b g' = g(b) - g(a)$.\\
Fundamental Theorem of Calculus II. let $F(x) = \int_a^x f(t)dt$, then $F$ is continuous on $[a,b]$. If $f$ continuous at $x_0\in(a,b)$, then $F$ differentiable at $x_0$ and $F'(x_0)=f(x_0)$.\\
Theorem [Change of Variable].\\
%Darboux-Stieltjes Integrals, 权重函数dF
Darboux-Stieltjes Integrals.\\
notation, increasing $F$, $F(t^-) = \lim_{x\rightarrow t^-} F(x)$; $+$. decree $F(a^-)=F(a)$\\
Def, \begin{equation}
J_F(f,P) = \sum f(t_k) [F(t_k^+) - F(t_k^-)]
\end{equation}
upper Darboux-Stieltjes sum,
\begin{equation}
U_F(f,P) = J_F(f,P) + \sum M(f,(t_{k-1}, t_k)) [F(t_k^-) - F(t_{k-1}^+)]
\end{equation}
lower Darboux-Stieltjes sum,
\begin{equation}
L_F(f,P) = J_F(f,P) + \sum m(f,(t_{k-1}, t_k)) [F(t_k^-) - F(t_{k-1}^+)]
\end{equation}
upper Darboux-Stieltjes integral, $U_F(f) = \inf U_F(f,P)$.\\
lower Darboux-Stieltjes integral, $L_F(f) = \sup L_F(f,P)$.\\
when equal, called F-integrable, write:
\begin{equation}
\int_a^b fdF = \int_a^b f(x) dF(x) = L_F(f) = U_F(f)
\end{equation}
most Thms likely above holds.\\
Thm, $F_1$-integrable, $F_2$-integrable $\Rightarrow$ $cF_1$-integrable, $(F_1+F_2)$-integrable.\\
$(F_j)$ each increasing, $F=\sum^\infty F_j$, converges, $F(a),F(b)$ finite. If bounded $f$ $F_j$-integrable, $\forall j$, then $F$-integrable, and
\begin{equation}
\int_a^b fdF = \sum^\infty \int_a^b fdF_j
\end{equation}
Thm, $F$ differentiable on $[a,b]$, $F'$ continuous on $[a,b]$, then
\begin{equation}
\int_a^b fdF = \int_a^b f(x) F'(x) dx
\end{equation}
%分解成连续部分和跳跃部分
jump function $F_d = \sum c_j J_{u_j}$, where $u_j$ are jump points, $c_j$ is the jump at $u_j$, i.e. $c_j = F(u_j^+) - F(u_j^-)$.\\
Thm, right-continuous increasing $F=F_c+F_d$, where $F_c$ continuous increasing, $F_d$ jump.\\
lemma, $\exists P$, s.t. $F(t_k^-) - F(t_{k-1}^+)<\epsilon$.\\
Thm, continuous $f$ is $F$-integrable.\\
Thm, monotonic $f$ is $F$-integrable.\\
Thm, piecewise continuous or bounded piecewise monotonic $f$, $F$-integrable.\\
prop, if $f$ $F$-integrable, $g(x)=f(x)$ except finite points, then $g$ is $F$-integrable. may not equal\\
Thm[Integration by Parts], define $F_i^*(t) = [F_i(t^-) + F_i(t^+)]/2$, then
\begin{equation}
\int_a^b F_1^*dF_2 + \int_a^b F_2^*dF_1 = F_1(b)F_2(b) - F_1(a)F_2(a)
\end{equation}
%Riemann-Stieltjes Integrals
\begin{equation}
\tilde{S}_F(f,P) = \sum_{k=1}^n f(x_k)[F(t_k) - F(t_{k-1})]
\end{equation}
Thm, if $f$ is Darboux-Stieltjes integrable, then $F$-integrable, and integrals agree.\\
%F-integrable 的"Cauchy"定义
def, $F\text{-mesh}(P) = \max \{F(t_k^-) - F(t_{k-1}^+):k=1,2,...,n\}$\\
lemma, if $\delta>0$, then $\exists P$ s.t. $F$-mesh$(P)<\delta$.\\
Thm, bounded $f$ on $[a,b]$ is $F$-integrable $\Leftrightarrow$ $\forall\epsilon>0$, $\exists \delta>0$ s.t.
\begin{equation}
	F\text{-mesh}(P)<\delta \text{ implies } U_F(f,P) - L_F(f,P)<\epsilon,\forall P
\end{equation}
Def, Riemman-Stieltjes sum: where $x_k\in(t_{k-1}, t_k)$
\begin{equation}
	J_F(f,P) + \sum_{k=1}^n f(x_k) [F(t_k^-) - F(t_{k-1}^+)]
\end{equation}
%Riemman-Stieltjes integrable, 收敛到r定义
Riemman-Stieltjes integrable if $\forall \epsilon>0, \exists \delta>0$ s.t. $|S-r|<\epsilon$.\\
write $r$ as $\mathcal{RS}\int_a^b fdF$.\\
Thm, bounded $f$ on $[a,b]$ is $F$-integrable $\Leftrightarrow$ Riemman-Stieltjes integrable. in which case integrals equal.
\begin{equation}
	\int_a^b fdF = r = \mathcal{RS}\int_a^b fdF
\end{equation}
notation, $\overline{U}_F(f)=\inf \overline{U}_F(f,P)$, for closed interval $[t_{k-1},t_k]$.\\
prop, if $f$ bounded on $[a,b]$ then $\overline{U}_F(f) = U_F(f)$. Thus $f$ integrable using closed intervals iff integrable using open intervals, in this case integral equal.\\
Corollary, for early normal case $F(x)=x$, Darboux can use open intervals instead of closed interval.\\
Thm, $F$ differentiable on $[a,b]$, $F'$ Riemann integrable on $[a,b]$, then bounded $f$ on $[a,b]$ is $F$-integrable iff $fF'$ is Riemann integrable, can get:
\begin{equation}
	\int_a^b fdF = \int_a^b f(x)F'(x)dx
\end{equation}
lemma, $F$ differentiable on $[a,b]$, $F'$ Riemann integrable on $[a,b]$, bounded $f$ on $[a,b]$. $B>0$ bounded $|f|$. If $U^{\flat}(F',P) - L^{\flat}(F',P) <\epsilon/B$. Then
\begin{align}
|U_F(f,P) - U^{\flat}(fF',P)| &\leq \epsilon \\
|L_F(f,P) - L^{\flat}(fF',P)| &\leq \epsilon
\end{align}
corollay, $F,f$ as above, $U_F(f) = U(fF')$.\\
lemma, $a_k\geq 0$, $B_k\subset \mathbb{R}$ nonempty bounded, if $\exists K$ that $\sum_{k=1}^n  a_kb_k\leq K$, for all choices of $b_k$ in $B_k$. Then $\sum_{k=1}^n a_k\sup B_k\leq K$.\\
Thm, if $f$ $F$-integrable on each interval $[a,b]$ and $f(x)\geq 0,\forall x\in \mathbb{R}$, then $f$ is $F$-integrable on $\mathbb{R}$ or $\int_{-\infty}^\infty fdF = +\infty$.\\
Thm, suppose $-\infty < F(-\infty)<F(\infty)<\infty$. $f$ bounded on R, $F$-integrable on each interval $[a,b]$. Then $f$ $F$-integrable on R.\\
distribution function, i.e. $F(-\infty)=0$ and $F(\infty)=1$.\\

\end{document}