%%%%%%%%%%%%%%%%%%%%%%%%%%%%%%%%%%%%%%%%%
% Short Sectioned Assignment
% LaTeX Template
% Version 1.0 (5/5/12)
%
% This template has been downloaded from:
% http://www.LaTeXTemplates.com
%
% Original author:
% Frits Wenneker (http://www.howtotex.com)
%
% License:
% CC BY-NC-SA 3.0 (http://creativecommons.org/licenses/by-nc-sa/3.0/)
%
%%%%%%%%%%%%%%%%%%%%%%%%%%%%%%%%%%%%%%%%%

%----------------------------------------------------------------------------------------
%	PACKAGES AND OTHER DOCUMENT CONFIGURATIONS
%----------------------------------------------------------------------------------------

\documentclass[paper=a4, fontsize=11pt]{scrartcl} % A4 paper and 11pt font size

\usepackage[T1]{fontenc} % Use 8-bit encoding that has 256 glyphs
\usepackage{fourier} % Use the Adobe Utopia font for the document - comment this line to return to the LaTeX default
\usepackage[english]{babel} % English language/hyphenation
\usepackage{amsmath,amsfonts,amsthm} % Math packages

\usepackage{sectsty} % Allows customizing section commands
\allsectionsfont{\centering \normalfont\scshape} % Make all sections centered, the default font and small caps
\usepackage{url}
\usepackage{fancyhdr} % Custom headers and footers
\pagestyle{fancyplain} % Makes all pages in the document conform to the custom headers and footers
\fancyhead{} % No page header - if you want one, create it in the same way as the footers below
\fancyfoot[L]{} % Empty left footer
\fancyfoot[C]{} % Empty center footer
\fancyfoot[R]{\thepage} % Page numbering for right footer
\renewcommand{\headrulewidth}{0pt} % Remove header underlines
\renewcommand{\footrulewidth}{0pt} % Remove footer underlines
\setlength{\headheight}{13.6pt} % Customize the height of the header

\numberwithin{equation}{section} % Number equations within sections (i.e. 1.1, 1.2, 2.1, 2.2 instead of 1, 2, 3, 4)
\numberwithin{figure}{section} % Number figures within sections (i.e. 1.1, 1.2, 2.1, 2.2 instead of 1, 2, 3, 4)
\numberwithin{table}{section} % Number tables within sections (i.e. 1.1, 1.2, 2.1, 2.2 instead of 1, 2, 3, 4)

\setlength\parindent{0pt} % Removes all indentation from paragraphs - comment this line for an assignment with lots of text


\title{MATH 115: Functions of a Real Variable}

\author{sogapalag}

\date{\normalsize\today}

\begin{document}

\maketitle

\section{Introduction}
Peano Axioms, basis of mathematical induction.\\
algebraic number, integer polynomial $p(x)=0$. $c_n\neq 0,\ n\geq 1$.\\
Rational Zeros Theorem: c divides $c_0$, d divedes $c_n$.\\
corollary, $x^n+c_{n-1}x^{n-1}+...+c_1x+c_0=0$, Any rational solution of this equation must be an integer that divides $c_0$\\
Field and order def and properties.\\
Def, max, min, upper bound, lower bound;\\
supremum sup S, if S is bounded above and S has a least upper bound;\\
infimum inf S, If S is bounded below and S has a greatest lower bound.\\
Completeness Axiom: Every nonempty subset S of R that is bounded above has a least upper bound. In other words, sup S exists and is a real number.\\
$-\infty,+\infty$ symbol for unbounded set. e.g. $\sup S=+\infty$\\
Archimedean Property. if $a,b>0$, then for some positive integer n, $na>b$.\\
Denseness of $\mathbb{Q}$, if $a,b\in\mathbb{R}$ and $a<b$, then $\exists r\in\mathbb{Q}$, s.t. $a<r<b$.\\

\section{Sequences}
%s_n足够逼近s
def of sequence, limit, converge, diverge. $\epsilon, N$ notation.\\
Thm, Convergent sequences are bounded.\\
Thm, $\lim(ks_n) = k\lim s_n$.\\
Thm, $\lim(s_n+t_n)=\lim s_n+ \lim t_n$.\\
Thm, $\lim(s_nt_n)=(\lim s_n) (\lim t_n)$.\\
lemma, if $s_n, s\neq 0$, then $\lim (1/s_n) = 1/s$.\\
Thm, $\lim(t_n/s_n)=t/s$.\\
Thm, positive real numbers, $\lim s_n=+\infty$ iff $\lim(1/s_n) =0$.\\
monotone sequence; increasing or decreasing\\
%单调确界隐含着逼近
Thm, All bounded monotone sequences converge.\\
Def, $\lim \sup s_n = \lim_{N\rightarrow\infty} sup\{s_n:n>N\}$.\\
Thm, if $\lim s_n$ defined, then $\lim \inf s_n=\lim s_n =\lim \sup s_n$; if $\lim \inf s_n=\lim \sup s_n$, then $\lim s_n$ is defined and $\lim s_n =\lim \sup s_n$.\\
%s_n与s_m足够逼近
Cauchy sequence, $\forall\epsilon, \exists N$, s.t. $m,n>N$ implies $|s_n-s_m|<\epsilon$.\\
lemma, Convergent sequences are Cauchy sequences.\\
lemma, Cauchy sequences are bounded.\\
Thm, Convergent sequences iff Cauchy sequences.\\
def, subsequence, 'select', 'in order', 'infinite';\\
%可挑出(甚至单调)子序列 当逼近集无限时收敛 or 无界时无穷极限
Thm, there is a subsequence of $(s_n)$ converging to $t$ iff set $\{n\in\mathbb{N}: |s_n-t|<\epsilon\}$ is infinite for all $\epsilon>0$. if unbounded above(below), it has a subsequence has limit $\infty$. in each case, the subsequence can be taken to be monotonic.\\
Thm, if $(s_n)$ converges, then every subsequence converges to the same limit.\\
%可挑出单调子序列
Thm, Every $(s_n)$ has a monotonic subsequence.\\
%有界,可挑出收敛子序列
%这个定理可从前面证得单调收敛,不过这个定理更广义
Bolzano-Weierstrass Theorem. Every bounded sequence has a convergent subsequence.\\
%可挑出单调收敛子序列,极限为确界极限
Thm, any $(s_n)$, There exists a monotonic subsequence whose limit is $\lim\sup s_n$, and there exists a monotonic subsequence whose limit is $\lim\inf s_n$.\\
Thm, S denote the set of subsequential limits, then $S$ is nonempty; $\sup S=\lim\sup s_n$ and $\inf S=\lim\inf s_n$; $\lim s_n$ exists iff $S$ has exactly one element.\\
Thm, S as above, suppose $(t_n)$ is a sequence in $S\cap \mathbb{R}$ and $t=\lim t_n$. Then $t\in S$.\\
Thm, if $\lim s_n=s>0$, then $\lim\sup s_nt_n = s\lim\sup t_n$, allow $\infty$.\\
%确界极限的定义蕴含着确界向内单调逼近,于是变化率严格小于,常量极限下,于是变\leq
Thm, $(s_n)$ any sequence of nonzero real numbers. Then
\begin{equation}
\lim\inf|\frac{s_{n+1}}{s_n}| \leq \lim\inf |s_n|^{1/n} \leq \lim\sup |s_n|^{1/n} \leq \lim\sup |\frac{s_{n+1}}{s_n}|
\end{equation}
Corollary, if $\lim |\frac{s_{n+1}}{s_n}|$ exists (equals L), then $\lim|s_n|^{1/n}$ exists (equals L).\\
Series.\\
$\sum_{k=0}^n ar^k = a\frac{1-r^{n+1}}{1-r}$, $r\neq 1$; $\sum_{n=0}^\infty ar^n = \frac{a}{1-r}$ if $|r|<1$\\
$\sum \frac{1}{n^p}$ converges iff $p>1$.\\
Def, a series $\sum a_n$ satisfies the Cauchy criterion if its sequence $(s_n)$ of partial sums is a Cauchy sequence.\\
Thm, A series converges if and only if it satisfies the Cauchy criterion.\\
Corollary, if a series $\sum a_n$ converges, then $\lim a_n = 0$.\\
%注意被比较的极数 a_n需要非负
Comparison Test, $a_n\geq 0, \forall n$: if $\sum a_n$ converges and $|b_n|\leq a_n,\forall n$, then $\sum b_n$ converges; if $\sum a_n = +\infty$ and $b_n\geq a_n, \forall n$, then $\sum b_n=+\infty$.\\
corollary, Absolutely convergent series are convergent.\\
%保证了存在epsilon,使得后续变化率<1-epsilon
Ratio Test, $\sum a_n$ nonzero terms: converges absolutely if $\lim\sup|a_{n+1}/a_n|<1$; diverges if $\lim\inf|a_{n+1}/a_n|>1$.\\
%更紧,所以当更松的增长率极限为1时,使得确界极限夹逼,该TEST也失效
Root Test, let $\alpha = \lim\sup |a_n|^{1/n}$: converges absolutely if $\alpha<1$; diverges if $\alpha >1$.\\
Integral Test.
consider $\lim \int_1^n f(x)dx$. advisa nonnegative, decreasing f.\\
Alternating Series Theorem: if $a_n$ decreasing, and $\lim a_n=0$, then the alternating series $\sum (-1)^{n+1} a_n$ converges.\\
Thm, A real number x has exactly one decimal expansion or else x has two decimal expansions, one ending in a sequence of all 0's and the other ending in a sequence of all 9's.\\
Thm, rational iff its decimal expansion is repeating.

\section{Continuity}
%默认定义在R上
%连续,函数值随着x逼近而逼近
continuous, $\epsilon$-$\delta$ notation.\\
Thm, $|f|,kf$ continuous at $x_0$ if $f$ does.\\
Thm, $f+g, fg, f/g$ so does.\\
Thm, if $f$ continuous at $x_0$, g at $f(x_0)$, then $g\circ f$ at $x_0$.\\
bounded, $|f(x)|\leq M$.\\
Thm, if f continuous on $[a,b]$, then bounded. moreover, maximum and minimum on $[a,b]$.\\
%画连续线(即不抬笔)必穿过介值
Intermediate Value Theorem: if continuous on I: $y$ between $(a),f(b)$, $\exists x\in(a,b)$, s.t. $f(x)=y$.\\
corollay, continuous on I, then $f(I)$ is interval or single point.\\
Thm, f continuous strictly increasing on I, then $f^{-1}$ continuous strictly increasing on $J=f(I)$.\\
%一直朝上画,y投影铺满了区间,故必连续
Thm, if g strictly increasing on J, s.t. $g(J)$ is interval. Then g continuous.\\
%连续画线若折返则有同值
Thm, Let f be a one-to-one continuous function on an interval I. Then f is strictly increasing or strictly decreasing.\\
%值随着x,y逼近而逼近; 与普通连续区别,delta尺子不依赖点,只依赖epsilon
%如1/x, 对于靠近0的点,总是可以普通连续逼近;但没有不变的delta尺子使得任意两点的度量小于epsilon
uniformly continuous;\\
Thm, continuous on $[a,b]$ $\Rightarrow$ uniformly continuous on $[a,b]$.\\
Thm, If f is uniformly continuous on a set S and $(s_n)$ is a Cauchy sequence in S, then $(f(s_n))$ is a Cauchy sequence.\\
%确定斜率(导数)非无穷了,尺子比例也限定了
%注,导数无穷,尺子比依然可能限定,如sqrt{x}
Thm, f on $(a,b)$ is uniformly continuous on $(a,b)$ iff it can be extended to a continuous function $\tilde{f}$ on $[a,b]$.\\
Thm, Let f be a continuous function on I. Let $I^\circ$ removing endpoints, if f differentiable on $I^\circ$ and $f'$ bounded on $I^\circ$, then $f$ uniformly continuous on I.\\
Mean Value Theorem: continuous on $[a,b]$, differentiable on $(a,b)$ then $\exists x\in(a,b)$, s.t. $f'(x) = \frac{f(b)-f(a)}{b-a}$\\
limit, as x tends to a along S of $f(x)$, i.e. $\lim_{x\rightarrow a^S} f(x)$.\\
by this def, $f$ is continuous at $a$ iff $\lim_{x\rightarrow a^S} f(x) = f(a)$.\\
left(right)-hand limit, $\lim_{x\rightarrow a^-} f(x)$, $\lim_{x\rightarrow a^+} f(x)$.\\
Thm, f defined on $J/\{a\}$ for some open interval $J$ containing a. Then $\lim_{x\rightarrow}f(x)$ exists iff $\lim_+$ and $\lim_-$ both exist and equal, i.e. THREE limits equal.\\
Prop, on $[a,b], $$\gamma(t) = (f_j(t),...)$ is continuous iff each $f_j$ is continuous.\\
%开集蕴涵着边界的接近和拓扑可区分性,故定义了连续
Thm, metric space, a function is continuous iff $f^{-1}(U)$ is open $\forall $ open $U$.\\
Thm, if f continuous, E compact. Then $f(E)$ is compact, $f$ uniformly continuous on E. 
\end{document}