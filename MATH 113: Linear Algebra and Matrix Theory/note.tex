%%%%%%%%%%%%%%%%%%%%%%%%%%%%%%%%%%%%%%%%%
% Short Sectioned Assignment
% LaTeX Template
% Version 1.0 (5/5/12)
%
% This template has been downloaded from:
% http://www.LaTeXTemplates.com
%
% Original author:
% Frits Wenneker (http://www.howtotex.com)
%
% License:
% CC BY-NC-SA 3.0 (http://creativecommons.org/licenses/by-nc-sa/3.0/)
%
%%%%%%%%%%%%%%%%%%%%%%%%%%%%%%%%%%%%%%%%%

%----------------------------------------------------------------------------------------
%	PACKAGES AND OTHER DOCUMENT CONFIGURATIONS
%----------------------------------------------------------------------------------------

\documentclass[paper=a4, fontsize=11pt]{scrartcl} % A4 paper and 11pt font size

\usepackage[T1]{fontenc} % Use 8-bit encoding that has 256 glyphs
\usepackage{fourier} % Use the Adobe Utopia font for the document - comment this line to return to the LaTeX default
\usepackage[english]{babel} % English language/hyphenation
\usepackage{amsmath,amsfonts,amsthm} % Math packages

\usepackage{sectsty} % Allows customizing section commands
\allsectionsfont{\centering \normalfont\scshape} % Make all sections centered, the default font and small caps

\usepackage{fancyhdr} % Custom headers and footers
\pagestyle{fancyplain} % Makes all pages in the document conform to the custom headers and footers
\fancyhead{} % No page header - if you want one, create it in the same way as the footers below
\fancyfoot[L]{} % Empty left footer
\fancyfoot[C]{} % Empty center footer
\fancyfoot[R]{\thepage} % Page numbering for right footer
\renewcommand{\headrulewidth}{0pt} % Remove header underlines
\renewcommand{\footrulewidth}{0pt} % Remove footer underlines
\setlength{\headheight}{13.6pt} % Customize the height of the header

\numberwithin{equation}{section} % Number equations within sections (i.e. 1.1, 1.2, 2.1, 2.2 instead of 1, 2, 3, 4)
\numberwithin{figure}{section} % Number figures within sections (i.e. 1.1, 1.2, 2.1, 2.2 instead of 1, 2, 3, 4)
\numberwithin{table}{section} % Number tables within sections (i.e. 1.1, 1.2, 2.1, 2.2 instead of 1, 2, 3, 4)

\setlength\parindent{0pt} % Removes all indentation from paragraphs - comment this line for an assignment with lots of text


\title{MATH 113: Linear Algebra and Matrix Theory}

\author{sogapalag}

\date{\normalsize\today}

\begin{document}

\maketitle 
\section{Vector Space}
Field $F$, set $S$, $V=F^{S}$, denotes the set of functions from $S$ to $F$.\\
Subset $U$ is Subspace, iff $0\in U$ and closed under addition and scalar multiplication.\\
$U_1 + \dots + U_m$ is smallest subspaces of $V$  containing $U_1$,...,$U_m$.\\
direct sum, $u_1 + \dots + u_m$ expressed in only one way, use $\bigoplus$ notation, $U_1 \bigoplus \dots \bigoplus U_m$. iff only way write $u_1 + \dots + u_m = 0$, is each $u_j = 0$.\\
for Two subspace, $U+W$ is direct sum iff $U\bigcap W = \{0\}$.\\
odd, even functions, $R^R = U_e \bigoplus U_o$, proof $f_e(x) = \frac{1}{2} [f(x) +  f(-x)]$ , $f_o(x) = \frac{1}{2} [f(x) -  f(-x)]$.


\section{Finite Dimensional Vector Space}
Def, $span(v_1,...,v_m) = \{a_1v_1+\dots +a_mv_m: a_1,...,a_m\in F\}$. $span() = \{0\}$\\
Thm, span is the smallest containing subspace.\\
A vector space is called finite-dimensional if some list of vectors in it spans the space.\\
polynomial $P(F)$ is a subspace of $F^F$.\\
Def, degree of a polynomial, $deg\ p = m$, if $p(z)=a_0+a_1z+\dots+ a_mz^m$, $a_m\neq 0$. $p = 0$ with degree $-\infty$.\\
Def, $P_m(F)$ denotes the set of all polynomials with coefficients in $F$ and degree at most $m$. note $P_m(F)=span(1,z,...,z^m)$\\
A vector space is called infinite-dimensional if it is not finite-dimensional.\\
linearly independent, $a_1v_1+\dots+a_mv_m = 0$ only when $a_1=\dots=a_m=0$. Iff $\forall v\in span(v_1,...,v_m)$, only one representation as a linear combination of $v_1,...,v_m$.\\
Length of linearly independent list $<=$ length of spanning list.\\
Every subspace of a finite-dimensional vector space is finite-dimensional.

Def, A basis of V is a list of vectors in V that is linearly independent and spans V.\\
iff $\forall v\in V$, unique form, $v=a_1v_1+\dots+a_nv_n$. \\
Every finite-dimensional vector space has a basis.\\
U is subspace of finite-dimensional V, Then there is a subspace W of V such that $V=U\oplus W$.\\
The dimension of a finite-dimensional vector space is the length of any basis of the vector space.\\
subspace, $dim(U_1+U_2) = dim\ U_1 +dim\ U_2 - dim(U_1\cap U_2)$.


\section{Linear Maps}
Def, linear map $T:V\rightarrow W$ with:\\
additivity $T(u+v)=Tu+Tv, \forall u,v\in V$;\\
homogeneity $T(\lambda v) = \lambda (Tv), \forall \lambda\in F, v\in V$.\\
unique linear map for basis, $Tv_j = w_j,\forall j$, dim V = dim W.\\
$L(V,W)$ is a vector space.\\
null space of T, $null\ T = \{v\in V: Tv=0\}$, is a subspace of V.\\
injective: maps distinct inputs to distinct outputs.\\
T is injective iff $null\ T =\{0\}$\\
range T is a subspace of W.\\
surjective: range T equals W.\\
Fundamental Theorem of Linear Maps: V, range T is finite dim, then 
\begin{equation}
dim\ V = dim\ null\ T + dim\ range\ T
\end{equation}
if $dim\ V > dim\ W$, no linear map is injective.\\
if $dim\ V < dim\ W$, no linear map is surjective.\\
Def matrix of T,  $Tv_k = A_{1,k}w_1+\dots+A_{m,k} w_m$\\
notation, set of all m-by-n matrices with entries in F, denoted by $F^{m,n}$, $dim\ F^{m,n} = mn$\\
$M(ST)$ with entries $\sum_{r=1}^n A_{j,r}C_{r,k}$. $(AC)_{j,k}=A_{j,\cdot} C_{\cdot,k}$
\end{document}