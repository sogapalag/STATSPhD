%%%%%%%%%%%%%%%%%%%%%%%%%%%%%%%%%%%%%%%%%
% Short Sectioned Assignment
% LaTeX Template
% Version 1.0 (5/5/12)
%
% This template has been downloaded from:
% http://www.LaTeXTemplates.com
%
% Original author:
% Frits Wenneker (http://www.howtotex.com)
%
% License:
% CC BY-NC-SA 3.0 (http://creativecommons.org/licenses/by-nc-sa/3.0/)
%
%%%%%%%%%%%%%%%%%%%%%%%%%%%%%%%%%%%%%%%%%

%----------------------------------------------------------------------------------------
%	PACKAGES AND OTHER DOCUMENT CONFIGURATIONS
%----------------------------------------------------------------------------------------

\documentclass[paper=a4, fontsize=11pt]{scrartcl} % A4 paper and 11pt font size

\usepackage[T1]{fontenc} % Use 8-bit encoding that has 256 glyphs
\usepackage{fourier} % Use the Adobe Utopia font for the document - comment this line to return to the LaTeX default
\usepackage[english]{babel} % English language/hyphenation
\usepackage{amsmath,amsfonts,amsthm} % Math packages

\usepackage{sectsty} % Allows customizing section commands
\allsectionsfont{\centering \normalfont\scshape} % Make all sections centered, the default font and small caps

\usepackage{fancyhdr} % Custom headers and footers
\pagestyle{fancyplain} % Makes all pages in the document conform to the custom headers and footers
\fancyhead{} % No page header - if you want one, create it in the same way as the footers below
\fancyfoot[L]{} % Empty left footer
\fancyfoot[C]{} % Empty center footer
\fancyfoot[R]{\thepage} % Page numbering for right footer
\renewcommand{\headrulewidth}{0pt} % Remove header underlines
\renewcommand{\footrulewidth}{0pt} % Remove footer underlines
\setlength{\headheight}{13.6pt} % Customize the height of the header

\numberwithin{equation}{section} % Number equations within sections (i.e. 1.1, 1.2, 2.1, 2.2 instead of 1, 2, 3, 4)
\numberwithin{figure}{section} % Number figures within sections (i.e. 1.1, 1.2, 2.1, 2.2 instead of 1, 2, 3, 4)
\numberwithin{table}{section} % Number tables within sections (i.e. 1.1, 1.2, 2.1, 2.2 instead of 1, 2, 3, 4)

\setlength\parindent{0pt} % Removes all indentation from paragraphs - comment this line for an assignment with lots of text


\title{MATH 113: Linear Algebra and Matrix Theory}

\author{sogapalag}

\date{\normalsize\today}

\begin{document}

\maketitle 
\section{Vector Space}
Field $F$, set $S$, $V=F^{S}$, denotes the set of functions from $S$ to $F$.\\
Subset $U$ is Subspace, iff $0\in U$ and closed under addition and scalar multiplication.\\
$U_1 + \dots + U_m$ is smallest subspaces of $V$  containing $U_1$,...,$U_m$.\\
direct sum, $u_1 + \dots + u_m$ expressed in only one way, use $\bigoplus$ notation, $U_1 \bigoplus \dots \bigoplus U_m$. iff only way write $u_1 + \dots + u_m = 0$, is each $u_j = 0$.\\
for Two subspace, $U+W$ is direct sum iff $U\bigcap W = \{0\}$.\\
odd, even functions, $R^R = U_e \bigoplus U_o$, proof $f_e(x) = \frac{1}{2} [f(x) +  f(-x)]$ , $f_o(x) = \frac{1}{2} [f(x) -  f(-x)]$.


\section{Finite Dimensional Vector Space}
Def, $span(v_1,...,v_m) = \{a_1v_1+\dots +a_mv_m: a_1,...,a_m\in F\}$. $span() = \{0\}$\\
Thm, span is the smallest containing subspace.\\
A vector space is called finite-dimensional if some list of vectors in it spans the space.\\
polynomial $P(F)$ is a subspace of $F^F$.\\
Def, degree of a polynomial, $deg\ p = m$, if $p(z)=a_0+a_1z+\dots+ a_mz^m$, $a_m\neq 0$. $p = 0$ with degree $-\infty$.\\
Def, $P_m(F)$ denotes the set of all polynomials with coefficients in $F$ and degree at most $m$. note $P_m(F)=span(1,z,...,z^m)$\\
A vector space is called infinite-dimensional if it is not finite-dimensional.\\
linearly independent, $a_1v_1+\dots+a_mv_m = 0$ only when $a_1=\dots=a_m=0$. Iff $\forall v\in span(v_1,...,v_m)$, only one representation as a linear combination of $v_1,...,v_m$.\\
Length of linearly independent list $\leq$ length of spanning list.\\
Every subspace of a finite-dimensional vector space is finite-dimensional.

Def, A basis of V is a list of vectors in V that is linearly independent and spans V.\\
iff $\forall v\in V$, unique form, $v=a_1v_1+\dots+a_nv_n$. \\
Every finite-dimensional vector space has a basis.\\
U is subspace of finite-dimensional V, Then there is a subspace W of V such that $V=U\oplus W$.\\
The dimension of a finite-dimensional vector space is the length of any basis of the vector space.\\
subspace, $dim(U_1+U_2) = dim\ U_1 +dim\ U_2 - dim(U_1\cap U_2)$.


\section{Linear Maps}
Def, linear map $T:V\rightarrow W$ with:\\
additivity $T(u+v)=Tu+Tv, \forall u,v\in V$;\\
homogeneity $T(\lambda v) = \lambda (Tv), \forall \lambda\in F, v\in V$.\\
unique linear map for basis, $Tv_j = w_j,\forall j$, dim V = dim W.\\
$L(V,W)$ is a vector space.\\
%维度蜷缩, null
null space of T, $null\ T = \{v\in V: Tv=0\}$, is a subspace of V.\\
injective: maps distinct inputs to distinct outputs.\\
T is injective iff $null\ T =\{0\}$\\
%维度投影, range
range T is a subspace of W.\\
surjective: range T equals W.\\
%总维度数等于蜷缩起来的维度数加上投影的维度数
Fundamental Theorem of Linear Maps: V, range T is finite dim, then 
\begin{equation}
dim\ V = dim\ null\ T + dim\ range\ T
\end{equation}
if $dim\ V > dim\ W$, no linear map is injective.\\
if $dim\ V < dim\ W$, no linear map is surjective.\\
%投影变换,对应两组基 Matrix
Def matrix of T,  $Tv_k = A_{1,k}w_1+\dots+A_{m,k} w_m$\\
notation, set of all m-by-n matrices with entries in F, denoted by $F^{m,n}$, $dim\ F^{m,n} = mn$\\
$M(ST)$ with entries $\sum_{r=1}^n A_{j,r}C_{r,k}$. $(AC)_{j,k}=A_{j,\cdot} C_{\cdot,k}$\\
Def, invertible, $ST=I_V, TS=I_W$\\
An invertible linear map has a unique inverse., notation $T^{-1}$\\
invertible $\Leftrightarrow$ injective and surjective.\\
Def, an isomorphism is an invertible linear map. Two vector spaces are called isomorphic if there is an isomorphism from one vector space onto the other one.\\
%同维度数则同构
Two finite-dimensional vector spaces over F are isomorphic if and only if they have the same dimension. implies isomorphic to $F^n$.\\
$L(V,W)$ and $F^{m,n}$ are isomorphic. finite $dim\ L = (dim\ V)(dim\ W)$.\\
Def, matrix of vector, $M(v) = (c_1,...,c_n)$ coeff of basis.\\
then ${M(T)}_{\cdot,k} = M(Tv_k)$, here with resopect to basis of $W$.\\
$M(Tv) = M(T)M(v)$, means, $Tv$ coeff under basis of W equals to coeff of v under basis of V multiplied by matrix A.\\
Def, operators, $L(V) = L(V,V)$.\\
finite-dim V, for operator T: invertible $\Leftrightarrow$ injective $\Leftrightarrow$ surjective.\\
Def, product $V_1 \times\dots\times V_m = \{(v_1,...,v_m): v_j\in V_j \forall j\}$; is vector space over $F$.\\
$R^2\times R^3$, $R^5$, not equal but isomorphic.\\
finite $dim(V_1 \times\dots\times V_m) = dim\ V_1 + \dots + dim\ V_m$.\\
$U_j$ subspace of $V$, define a linear map $\Gamma: U_1 \times \dots\times U_m \rightarrow U_1+\dots +U_m$ by $\Gamma(u_1,...,u_m)=u_1+\dots+u_m$. Then direct sum iff $\Gamma$ is injective.\\
%直和,维度互不干涉
direct sum $\Leftrightarrow$ $dim(U_1+\dots +U_m) = dim\ U_1+ \dots + dim\ U_m$.\\
affine subset, $v+U =\{v+u: u\in U\}$, is said to be parallel to $U$.\\
quotient space $V/U = \{v+U: v\in V\}$.\\
Two affine subsets parallel to U are equal or disjoint: $v-w\in U$ $\Leftrightarrow$ $v+U=w+U$ $\Leftrightarrow$ $(v+U)\cap (w+U)\neq\emptyset$.
%将U维度蜷缩,剩余维度的扩张 quotient
quotient space $V/U$ with the operations of addition and scalar multiplication as defined as $(v+U) +(w+U) = (v+w) +U, \lambda(v+U) = (\lambda v) +U$, is vector space.\\
quotient map $\pi: V\rightarrow V/U$ defined by $\pi(v)=v+U$.\\
Finite, $dim\ V/U = dim\ V - dim\ U$.\\
%将null蜷缩与广义蜷缩对应
Define $\tilde{T}: V/(null\ T) \rightarrow W$ by $\tilde{T}(v+null\ T)= Tv$. Then $\tilde{T}$ is a linear map, injective, $range\ \tilde{T} = range\ T$, $V/(null\ T)$ is isomorphic to $range\ T$.\\
linear functional, $\in L(V,F)$.\\
%单维度投影空间同构于原空间,对偶基,使得坐标等值对应
dual space of V, denoted $V' = L(V,F)$\\
dual basis $\varphi_j(v_k)=1$ if $k=j$, otherwise $0$. is basis of $V'$\\
%单维度投影空间之间的投影,V->W 存活维度的反射
dual map $T'\in L(W', V')$ defined by $T'(\varphi) = \varphi\circ T$ for $\varphi\in W'$.\\
$(S+T)'=S'+T'$, $(\lambda T)'=\lambda T'$, $(ST)'=T'S'$.\\
%使得U所有维度蜷缩的投影空间,即投影空间下缺少的维度,蜷缩成蜷缩投影
$U\subset V$, annihilator $U^0=\{\varphi\in V': \varphi(u)=0\ for\ all\ u\in U\}$, is subspace of $V'$\\
finite, $dim\ U+dim\ U^0 = dim\ V$.\\
finite, $null\ T' = (range\ T)^0$,  $dim\ null\ T'= dim\ null\ T+dim\ W -dim\ V$.\\
%显然,不缺少维度
finite, $T$ is surjective iff $T'$ is injective.\\
finite, $T$ is injective iff $T'$ is surjective.\\
%使得蜷缩维度蜷缩起来的投影
finite, $dim\ range\ T' = dim\ range\ T = rank$, $range\ T' =(null\ T)^0$\\
transpose properties, $(A+C)^t = A^t+C^t,\ (\lambda A)^t=\lambda A^t,\ (AC)^t=C^tA^t$\\
$M(T')=(M(T))^t$

\section{Polynomials}
complex conjugate $\bar{z}=Re\ z - (Im\ z)i$, properties, $z+\bar{z}=2Re\ z$, $z-\bar{z}=2i(Im\ z)$, $z\bar{z}=|z|^2$, $|wz|=|w||z|$, $|w+z|\leq|w|+|z|$\\
Uniqueness of Coefficients for Polynomials.\\
Division Algorithm for Polynomials, $p,s\in P(F)$, $s\neq 0$, then $\exists q,r\in P(F)$ unique, s.t. $p=sq+r$ and $deg\ r < deg\ s$.\\
$p(\lambda)=0$ iff $z-\lambda$ is a factor of $p$.\\
Fundamental Theorem of Algebra: Every nonconstant polynomial with complex coefficients has a zero.\\
nonconstant polynomial $p\in P(C)$ has a unique factorization, $p(z)=c(z-\lambda_1)\dots(z-\lambda_m)$.\\
real coefficients, if $\lambda$ is root, so is $\bar{\lambda}$.\\
nonconstant polynomial $p\in P(R)$ has a unique factorization, $p(x)=c(x-\lambda_1)\dots(x-\lambda_m)(x^2+b_1x+c_1)\dots(x^2+b_Mx+c_M)$, $b_j^2<4c_j$
\end{document}