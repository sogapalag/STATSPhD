%%%%%%%%%%%%%%%%%%%%%%%%%%%%%%%%%%%%%%%%%
% Short Sectioned Assignment
% LaTeX Template
% Version 1.0 (5/5/12)
%
% This template has been downloaded from:
% http://www.LaTeXTemplates.com
%
% Original author:
% Frits Wenneker (http://www.howtotex.com)
%
% License:
% CC BY-NC-SA 3.0 (http://creativecommons.org/licenses/by-nc-sa/3.0/)
%
%%%%%%%%%%%%%%%%%%%%%%%%%%%%%%%%%%%%%%%%%

%----------------------------------------------------------------------------------------
%	PACKAGES AND OTHER DOCUMENT CONFIGURATIONS
%----------------------------------------------------------------------------------------

\documentclass[paper=a4, fontsize=11pt]{scrartcl} % A4 paper and 11pt font size

\usepackage[T1]{fontenc} % Use 8-bit encoding that has 256 glyphs
\usepackage{fourier} % Use the Adobe Utopia font for the document - comment this line to return to the LaTeX default
\usepackage[english]{babel} % English language/hyphenation
\usepackage{amsmath,amsfonts,amsthm} % Math packages

\usepackage{sectsty} % Allows customizing section commands
\allsectionsfont{\centering \normalfont\scshape} % Make all sections centered, the default font and small caps
\usepackage{url}
\usepackage{fancyhdr} % Custom headers and footers
\pagestyle{fancyplain} % Makes all pages in the document conform to the custom headers and footers
\fancyhead{} % No page header - if you want one, create it in the same way as the footers below
\fancyfoot[L]{} % Empty left footer
\fancyfoot[C]{} % Empty center footer
\fancyfoot[R]{\thepage} % Page numbering for right footer
\renewcommand{\headrulewidth}{0pt} % Remove header underlines
\renewcommand{\footrulewidth}{0pt} % Remove footer underlines
\setlength{\headheight}{13.6pt} % Customize the height of the header

\numberwithin{equation}{section} % Number equations within sections (i.e. 1.1, 1.2, 2.1, 2.2 instead of 1, 2, 3, 4)
\numberwithin{figure}{section} % Number figures within sections (i.e. 1.1, 1.2, 2.1, 2.2 instead of 1, 2, 3, 4)
\numberwithin{table}{section} % Number tables within sections (i.e. 1.1, 1.2, 2.1, 2.2 instead of 1, 2, 3, 4)

\setlength\parindent{0pt} % Removes all indentation from paragraphs - comment this line for an assignment with lots of text


\title{HW1}

\author{sogapalag}

\date{\normalsize\today}

\begin{document}

\maketitle 
\url{http://web.stanford.edu/~mkemeny/113lectures/Homework1.pdf}

1.\\
$\Rightarrow$\\
if $n$ is not a prime, means, $\exists 1<x,y<n$, s.t. $\bar{x}\cdot\bar{y} = \bar{xy} = \bar{n} = \bar{0}$, since $\mathbb{Z}_n$ is field, $\exists 0<x^{-1},y^{-1}<n$ that $\bar{1} = \bar{x^{-1}}\cdot\bar{y^{-1}}\cdot \bar{x}\cdot\bar{y} = \bar{0}$, which conflicts to $\bar{1}\neq \bar{0}$.\\
$\Leftarrow$\\
additive is trivial; by Fermat's little theorem, $x^{-1} \equiv x^{n-2} (\mod n)$. \qed

2.\\
closed unbder $+$ and $\cdot$ is trivial, 0 is trivial, for 1, that\\
\begin{equation}
\frac{1}{a+b\sqrt{2}} = \frac{a - b\sqrt{2}}{a^2 - 2b^2} = \frac{a}{a^2 - 2b^2}  - \frac{b\sqrt{2}}{a^2 - 2b^2} \in \mathbb{Q}(\sqrt{2})
\end{equation}\qed

3.\\
by the definition of 1.24 example, and according to distributive property of the field, can easily prove.\qed

1.C.1\\
yes;yes;no;yes.\\
1.C.7\\
$U=\mathbb{Z}$, that not close under scalar multiplication over $R$.\\
1.C.11\\
$u\in U = U_1\bigcap\dots\bigcap U_m$ means,\\
$0\in U_1,...,U_m$ that $0\in U$;\\
$-u\in U_1,...,U_m$ that $-u\in U$;\\
$\lambda u\in U_1,...,U_m$ that $\lambda u\in U$.\qed

1.C.12\\
$\Rightarrow$\\
suppose $\exists u_1\in U_1 - U_2, u_2\in U_2 -U_1$, then $u_1+u_2\in U_1\cup U_2$, either $u_2\in U_1$ or $u_1\in U_2$, which conflicts.\\
$\Leftarrow$\\
trivial.\qed

Bonus\\
since $|\mathbb{Z}| = |\mathbb{Q}|$, $\exists f: \mathbb{Z}\rightarrow\mathbb{Q}$ bijection. Define $\oplus$, $\otimes$ on $\mathbb{Z}$:
\begin{equation}
	a\oplus b = f^{-1}(f(a)+f(b))\\
	a\otimes b = f^{-1}(f(a)\cdot f(b))
\end{equation}
then $\mathbb{Z}$ is isomorphic to $\mathbb{Q}$, into a field.\\
which tells us, any bijection can turn to isomorphic, cause we just use one set to just be symbolic to another set, under another operations, within another 0 and 1, discard its own "natural" operations. 
\end{document}