%%%%%%%%%%%%%%%%%%%%%%%%%%%%%%%%%%%%%%%%%
% Short Sectioned Assignment
% LaTeX Template
% Version 1.0 (5/5/12)
%
% This template has been downloaded from:
% http://www.LaTeXTemplates.com
%
% Original author:
% Frits Wenneker (http://www.howtotex.com)
%
% License:
% CC BY-NC-SA 3.0 (http://creativecommons.org/licenses/by-nc-sa/3.0/)
%
%%%%%%%%%%%%%%%%%%%%%%%%%%%%%%%%%%%%%%%%%

%----------------------------------------------------------------------------------------
%	PACKAGES AND OTHER DOCUMENT CONFIGURATIONS
%----------------------------------------------------------------------------------------

\documentclass[paper=a4, fontsize=11pt]{scrartcl} % A4 paper and 11pt font size

\usepackage[T1]{fontenc} % Use 8-bit encoding that has 256 glyphs
\usepackage{fourier} % Use the Adobe Utopia font for the document - comment this line to return to the LaTeX default
\usepackage[english]{babel} % English language/hyphenation
\usepackage{amsmath,amsfonts,amsthm} % Math packages

\usepackage{sectsty} % Allows customizing section commands
\allsectionsfont{\centering \normalfont\scshape} % Make all sections centered, the default font and small caps
\usepackage{url}
\usepackage{fancyhdr} % Custom headers and footers
\pagestyle{fancyplain} % Makes all pages in the document conform to the custom headers and footers
\fancyhead{} % No page header - if you want one, create it in the same way as the footers below
\fancyfoot[L]{} % Empty left footer
\fancyfoot[C]{} % Empty center footer
\fancyfoot[R]{\thepage} % Page numbering for right footer
\renewcommand{\headrulewidth}{0pt} % Remove header underlines
\renewcommand{\footrulewidth}{0pt} % Remove footer underlines
\setlength{\headheight}{13.6pt} % Customize the height of the header

\numberwithin{equation}{section} % Number equations within sections (i.e. 1.1, 1.2, 2.1, 2.2 instead of 1, 2, 3, 4)
\numberwithin{figure}{section} % Number figures within sections (i.e. 1.1, 1.2, 2.1, 2.2 instead of 1, 2, 3, 4)
\numberwithin{table}{section} % Number tables within sections (i.e. 1.1, 1.2, 2.1, 2.2 instead of 1, 2, 3, 4)

\setlength\parindent{0pt} % Removes all indentation from paragraphs - comment this line for an assignment with lots of text


\title{HW3}

\author{sogapalag}

\date{\normalsize\today}

\begin{document}

\maketitle 
\url{http://web.stanford.edu/~mkemeny/113lectures/Homework3.pdf}

2.B.4\\
(a), $(1, 0, 0, 0,0)$, $(0,0,1,0,0)$ and $(0,0,0,1,0)$;\\
(b)(c), add $(0, 1, 0, 0,0)$ and $(0, 0, 0, 0,1)$.\\
2.B.7\\
note $span(v_1, v_2) $ and $span(v_3, v_4)$ intersect $\{0\}$, both $dim = 2$.\qed

2.C.3\\
subspace with dimension $0,1,2,3$, and with corresponding number bases.\qed

2.C.8\\
(a) $(0,x,0,0,0)$ and $(0,0,0,x^3,0)$\\
(b)(c), trivial.\\
2.C.12\\
$dim\ U\cap W = dim\ U +dim\ W - dim(U+W) >= 10- dim(R^9) = 1$.\qed

2.C.14\\
trivial, dimensions of intersects $>=0$.\qed

2.C.17\\
note $(U_1 +U_2)\cap U_3 = (U_1\cap U_3) + (U_2\cap U_3)$.\qed

3.C.2\\
$P_2(R)$ basis: $(1,0,0)$, $(0,2x,0)$ and $(0,0,3x^2)$\\
$P_3(R)$ basis: $(0,x,0,0)$, $(0,0,x^2,0)$, $(0,0,0,x^3)$ and $(1,0,0,0)$\\

3.C.3\\
there is basis $w_j$ of $range\ T$, and $Tv_j = w_j$, that $v_j$ is linear independent; then expand basis $v_j$ and $w_j$ to basis of $V$ and $W$.\qed

3.C.6\\
$\Rightarrow$\\
$Tv_j=w$, for a basis $\{v_j\}$, $w$ is basis of $range\ T$; expand $w$ into basis of $W$, i.e. $\{w, w_1,...,w_{m-1}\}$, let $w_m = w-w_1-...-w_{m-1}$, that $\{w_j\}$ still a basis of $W$, then matrix of $T$ under basis $v_j$ and $w_j$, which all entries is 1.\\
$\Leftarrow$\\
$Tv = \sum_j c_jTv_j = \sum_j c_j w$.\qed

Bonus\\
let $dim\ V = n$, then there is basis $v_j$ of $V$, that $\forall v\in V$, $v= \sum_j c_j v_j$, where $c_j\in F_p$, since $|c|$ is finite, that $|V|$ is finite. for every $u=v$, that $(c_u-c_v) v_j = 0$, means $c_u=c_v$ under $F_p$, so every distinc $c$ create distinc $v$, $|V| = p^n$.\qed 
\end{document}