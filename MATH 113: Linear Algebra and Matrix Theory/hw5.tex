%%%%%%%%%%%%%%%%%%%%%%%%%%%%%%%%%%%%%%%%%
% Short Sectioned Assignment
% LaTeX Template
% Version 1.0 (5/5/12)
%
% This template has been downloaded from:
% http://www.LaTeXTemplates.com
%
% Original author:
% Frits Wenneker (http://www.howtotex.com)
%
% License:
% CC BY-NC-SA 3.0 (http://creativecommons.org/licenses/by-nc-sa/3.0/)
%
%%%%%%%%%%%%%%%%%%%%%%%%%%%%%%%%%%%%%%%%%

%----------------------------------------------------------------------------------------
%	PACKAGES AND OTHER DOCUMENT CONFIGURATIONS
%----------------------------------------------------------------------------------------

\documentclass[paper=a4, fontsize=11pt]{scrartcl} % A4 paper and 11pt font size

\usepackage[T1]{fontenc} % Use 8-bit encoding that has 256 glyphs
\usepackage{fourier} % Use the Adobe Utopia font for the document - comment this line to return to the LaTeX default
\usepackage[english]{babel} % English language/hyphenation
\usepackage{amsmath,amsfonts,amsthm} % Math packages

\usepackage{sectsty} % Allows customizing section commands
\allsectionsfont{\centering \normalfont\scshape} % Make all sections centered, the default font and small caps
\usepackage{url}
\usepackage{fancyhdr} % Custom headers and footers
\pagestyle{fancyplain} % Makes all pages in the document conform to the custom headers and footers
\fancyhead{} % No page header - if you want one, create it in the same way as the footers below
\fancyfoot[L]{} % Empty left footer
\fancyfoot[C]{} % Empty center footer
\fancyfoot[R]{\thepage} % Page numbering for right footer
\renewcommand{\headrulewidth}{0pt} % Remove header underlines
\renewcommand{\footrulewidth}{0pt} % Remove footer underlines
\setlength{\headheight}{13.6pt} % Customize the height of the header

\numberwithin{equation}{section} % Number equations within sections (i.e. 1.1, 1.2, 2.1, 2.2 instead of 1, 2, 3, 4)
\numberwithin{figure}{section} % Number figures within sections (i.e. 1.1, 1.2, 2.1, 2.2 instead of 1, 2, 3, 4)
\numberwithin{table}{section} % Number tables within sections (i.e. 1.1, 1.2, 2.1, 2.2 instead of 1, 2, 3, 4)

\setlength\parindent{0pt} % Removes all indentation from paragraphs - comment this line for an assignment with lots of text


\title{HW5}

\author{sogapalag}

\date{\normalsize\today}

\begin{document}

\maketitle 
\url{http://web.stanford.edu/~mkemeny/113lectures/Homework5.pdf}

3.F.1\\
if $f(v)=c\neq 0$, then $c$ is basis of $F$.\qed

3.F.4\\
note $U\subset V$, and $U\neq V$, i.e. basis $u_j$ can expand, with some $v\notin U$, consider dual space of V under these dual basis, then $\varphi_v$ is the one.\\
Another solution, $dim\ U^0 = dim\ V - dim\ U > 0$.\qed

3.F.6\\
By Linear Dependence Lemma, relable $v_1,...,v_k$, that $k=dim\ span(v_1,...,v_m)$, $k\leq n,m$; consider $\pi: F^k\rightarrow F^m$, which corresponding calculate $v_{k+1},...,v_m$; then $range\ \Gamma\subset range\ \pi$, i.e. $r = dim\ range\ Gamma \leq k$, and under dual basis of $v_1,...,v_k$, we know $range \pi\subset range\ \Gamma$, i.e. $r=k$\\
(a) left iff $k=n$, right iff $r=n$.\qed\\
(b) left iff $k=m$, right iff $r=m$.\qed

3.F.7\\
trivial.

3.F.11\\
rank is 1, iff $dim\ range\ T=1$, which basis $w=\sum_jc_jw_j$, choose this $c$ under $w_j$.\qed

2.\\
define $\alpha((x_1,y_1),(x_2,y_2)) = x_1y_2 - x_2y_1$.\\

3.\\
(a) swap(1,3), swap(3,4), swap(4,2);\\
(b) 3.\\
(c) -1.\\
4.recall det.\\
bonus, $n!$.
\end{document}