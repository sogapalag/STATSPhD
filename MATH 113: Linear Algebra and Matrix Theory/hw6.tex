%%%%%%%%%%%%%%%%%%%%%%%%%%%%%%%%%%%%%%%%%
% Short Sectioned Assignment
% LaTeX Template
% Version 1.0 (5/5/12)
%
% This template has been downloaded from:
% http://www.LaTeXTemplates.com
%
% Original author:
% Frits Wenneker (http://www.howtotex.com)
%
% License:
% CC BY-NC-SA 3.0 (http://creativecommons.org/licenses/by-nc-sa/3.0/)
%
%%%%%%%%%%%%%%%%%%%%%%%%%%%%%%%%%%%%%%%%%

%----------------------------------------------------------------------------------------
%	PACKAGES AND OTHER DOCUMENT CONFIGURATIONS
%----------------------------------------------------------------------------------------

\documentclass[paper=a4, fontsize=11pt]{scrartcl} % A4 paper and 11pt font size

\usepackage[T1]{fontenc} % Use 8-bit encoding that has 256 glyphs
\usepackage{fourier} % Use the Adobe Utopia font for the document - comment this line to return to the LaTeX default
\usepackage[english]{babel} % English language/hyphenation
\usepackage{amsmath,amsfonts,amsthm} % Math packages

\usepackage{sectsty} % Allows customizing section commands
\allsectionsfont{\centering \normalfont\scshape} % Make all sections centered, the default font and small caps
\usepackage{url}
\usepackage{fancyhdr} % Custom headers and footers
\pagestyle{fancyplain} % Makes all pages in the document conform to the custom headers and footers
\fancyhead{} % No page header - if you want one, create it in the same way as the footers below
\fancyfoot[L]{} % Empty left footer
\fancyfoot[C]{} % Empty center footer
\fancyfoot[R]{\thepage} % Page numbering for right footer
\renewcommand{\headrulewidth}{0pt} % Remove header underlines
\renewcommand{\footrulewidth}{0pt} % Remove footer underlines
\setlength{\headheight}{13.6pt} % Customize the height of the header

\numberwithin{equation}{section} % Number equations within sections (i.e. 1.1, 1.2, 2.1, 2.2 instead of 1, 2, 3, 4)
\numberwithin{figure}{section} % Number figures within sections (i.e. 1.1, 1.2, 2.1, 2.2 instead of 1, 2, 3, 4)
\numberwithin{table}{section} % Number tables within sections (i.e. 1.1, 1.2, 2.1, 2.2 instead of 1, 2, 3, 4)

\setlength\parindent{0pt} % Removes all indentation from paragraphs - comment this line for an assignment with lots of text


\title{HW6}

\author{sogapalag}

\date{\normalsize\today}

\begin{document}

\maketitle 
\url{http://web.stanford.edu/~mkemeny/113lectures/Homework6.pdf}

Mid1.1\\
(a) $\Rightarrow$ if not, some $v=\sum c_jv_j$ in null T; $\Leftarrow$, $Tv=T(c_jv_j)$, if $v\neq c_jv_j$, that $v-c_jv_j$ results $Tv_j$ dependent.\qed\\
(b) $\Rightarrow$ formula $v$ on $Tv$; $\Leftarrow$, close formula $c_jTv_j$.\qed

Mid1.2\\
(a) yes, e.g. $T(x)=x^3$.\\
(b) no, dimension is less.\\
(c) no, $x$ is independent to given basis.\\
(d) yes, indeed $T(x^2+2)$.\\

Mid1.3\\
(a) if diagonalizable, then nonzero eigenvalue correponding SAME eigenvector that $Tv_i=\lambda_i v_i$, $T^2v_i = \lambda_i^2 v_i$, i.e. Image equals, conflict.\qed

Mid1.5\\
suppose $T^2\neq 0$, then $T\neq 0$, since $T^3=0$, that dim of Im($T^2$) can only be 1. Thus $Im(T)=Ker(T^2)$, $Im(T^2)=Ker(T)$, then for any $v$ of combs, either $T(Tv)=0$ or $T^2(v)=0$, thus $T^2=0$ conflicts.\qed

Mid2.3\\
(a)$\forall u\in range\ ST$, that $u=ST(v) = S(Tv)\in range\ S$.\qed\\
(b) (III), surjective can be obviously; non-surjective also can be, e.g. $V=U$, not equal but subspace of $W$, T,S be $i_V$, for S, extra be ker.\qed

Bonus\\
(a)
\begin{equation}\begin{split}
(\psi\circ\phi)(v) &= \psi(\phi v) \\
&= \sum_i \alpha(\phi(b_1),...,\phi(b_n)) a_ib_i \\
&= \alpha(\phi(b_1),...,\phi(b_n))\sum_ia_ib_i \\
&= (det\ \phi) v
\end{split}
\end{equation}
(b)
\begin{equation}\begin{split}
(\phi\circ\psi)(v) &= \phi(\psi v) \\
&= \phi\sum_i \alpha(\phi(b_1),...,\phi(b_n)) a_ib_i \\
&= \alpha(\phi(b_1),...,\phi(b_n))\sum_i\phi(a_ib_i) \\
&= (det\ \phi) \sum_ia_i\phi(b_i) \\
&= (det\ \phi) v
\end{split}
\end{equation}
\end{document}