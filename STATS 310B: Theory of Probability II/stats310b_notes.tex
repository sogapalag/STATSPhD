%%%%%%%%%%%%%%%%%%%%%%%%%%%%%%%%%%%%%%%%%
% Short Sectioned Assignment
% LaTeX Template
% Version 1.0 (5/5/12)
%
% This template has been downloaded from:
% http://www.LaTeXTemplates.com
%
% Original author:
% Frits Wenneker (http://www.howtotex.com)
%
% License:
% CC BY-NC-SA 3.0 (http://creativecommons.org/licenses/by-nc-sa/3.0/)
%
%%%%%%%%%%%%%%%%%%%%%%%%%%%%%%%%%%%%%%%%%

%----------------------------------------------------------------------------------------
%	PACKAGES AND OTHER DOCUMENT CONFIGURATIONS
%----------------------------------------------------------------------------------------

\documentclass[paper=a4, fontsize=11pt]{scrartcl} % A4 paper and 11pt font size

\usepackage[T1]{fontenc} % Use 8-bit encoding that has 256 glyphs
\usepackage{fourier} % Use the Adobe Utopia font for the document - comment this line to return to the LaTeX default
\usepackage[english]{babel} % English language/hyphenation
\usepackage{amsmath,amsfonts,amsthm} % Math packages

\usepackage{sectsty} % Allows customizing section commands
\allsectionsfont{\centering \normalfont\scshape} % Make all sections centered, the default font and small caps
\usepackage{url}
\usepackage{fancyhdr} % Custom headers and footers
\pagestyle{fancyplain} % Makes all pages in the document conform to the custom headers and footers
\fancyhead{} % No page header - if you want one, create it in the same way as the footers below
\fancyfoot[L]{} % Empty left footer
\fancyfoot[C]{} % Empty center footer
\fancyfoot[R]{\thepage} % Page numbering for right footer
\renewcommand{\headrulewidth}{0pt} % Remove header underlines
\renewcommand{\footrulewidth}{0pt} % Remove footer underlines
\setlength{\headheight}{13.6pt} % Customize the height of the header

\numberwithin{equation}{section} % Number equations within sections (i.e. 1.1, 1.2, 2.1, 2.2 instead of 1, 2, 3, 4)
\numberwithin{figure}{section} % Number figures within sections (i.e. 1.1, 1.2, 2.1, 2.2 instead of 1, 2, 3, 4)
\numberwithin{table}{section} % Number tables within sections (i.e. 1.1, 1.2, 2.1, 2.2 instead of 1, 2, 3, 4)

\setlength\parindent{0pt} % Removes all indentation from paragraphs - comment this line for an assignment with lots of text
\def \cov {\text{Cov}}
\def \var {\text{Var}}

\title{STATS 310B: Theory of Statistics II}
% Amir-Dembo
\author{sogapalag}

\date{\normalsize\today}

\begin{document}

\maketitle

version of C.E. (almost) uniquess.\\
% 注意!!以下所有定理的要求有暂时忽略(比如有的是需要有限,可积),细节自查
% 两个测度的相对变化率(求导),唯一(almost)
Radon-Nikodym theorem: almost uniquess of two measure derivative, $\frac{dv}{d\mu}$\\
% 两个测度被某个A划分,即u1(A)=0, u2(A^c)=0
mutually singular\\
% 一个测度可以唯一分解为另一测度的垂直部分,和
Lebesgue decomposition(unique): $\nu=\nu_{ac}+\nu_s$, where $\nu_s\perp \mu$, $\nu_{ac}=f\mu$.\\
% 使得P(A|B)有意义(即便P(B)=0,(measure zero)如落在某点上)
regular conditional probability distribution\\
% 一系列不减的, F的子sigma域
filtration, $\{\mathcal{F}_n\}$\\
% 即随机过程 X_n在每个F_n可measure(有定义)
$\mathcal{F}_n$-adapted, $\sigma(X_n)\subset \mathcal{F}_n,\forall n$.
% 即(X_0,...,X_n)生成域
canonical(minimal) filtration\\
% 即每个状态条件期望与当前状态一致,不变,(无法获取收益)
martingale(MG), $E[X_{n+1}|\mathcal{F}_n]=X_n$, $\forall n$, a.s.\\
% 即若随机过程可拆分,则鞅iff每一步D条件期望为0
if $X_n=\sum_{k=0}^n D_k$, MG iff $E[D_{n+1}|\mathcal{F}_n]=0$.\\
% SRW(simple random walk),P(+1)=p; symmetric, law(D)=law(-D). 
% 即随机过程的积为鞅 iff 每个状态(条件)期望为1 a.s.
product martingale, $M_n=\prod Y_k$, iff $E[Y_k]=1$ indep $\{Y_k\}$. general $E[Y_{k+1}|Y_1,...,Y_k]=1$ a.s.\\
% 即依据n步仅有的信息可测n步内,(不依赖未来信息),直觉上即停时是明确的
stopping time, $\{\omega:\tau(\omega)\leq n\}\in \mathcal{F}_n$, forall finite $n$.\\
Sub-martingales, super-martingales; 5.1.22, convex $\Phi(X_{n})$\\
% 随机过程可唯一分解为 MG和predictable
Doob's decomposition\\
% 半鞅 P(maxX >x) 的bound
Doob's inequality, (generalize Kolmogorov's maximal inequality)\\
% 半鞅,鞅 E[(max)^p] 的bound
Lp maximal inequalities\\
% E[U] up-cross 区间[a,b]的期望次数 bound
Doob's up-crossing inequality\\
% sup-MG 的收敛; 即对于半鞅,对应的>or<那一侧E{+or-}<infty, 则它收敛(a.s.)
Doob's convergence theorem\\
% Corollary 5.3.4 U.I. sub-MG, P(max>x)的bound
% Proposition 5.3.5 MG, sup|-|<c (a.s.), 则X_n极限a.s.存在(or infty)(即不会多极限跳跃)
% 半鞅一致可积(U.I.) iff   L1 -> ; also a.s. ->
% Doob's martingale, X_n= E[X|F_n];  <=> MG U.I.
% sup|X_m| integrable, X_n->(a.s.) => E[X_n|F_n] -> (a.s.; L1)
% 即X的收敛 可得出期望的收敛
Levy's upward theorem\\
% 即对于A\in F_\infty, A事件的结果将会逐渐确定;general Kolm, Kolm 要求P-m.i.
Levy's 0-1 law\\
% MG, sup E|X_n|^p<infty, p>1 则X_n -> (a.s.;Lp)
Doob's Lp martingale convergence\\
% 5.3.33; a.e. omega, S/Z ->1 
Borel-Cantelli III\\
% 定理的常用推论,即对于MG,任意停时, E[X_tau]=E[X_0]
Doob's optional stopping\\
% 即random tree, 第n代 由 第(n-1)代随机生成枝(i.i.d); sub,,sup -critical, (m_N=EN <,=,>1)
Branching process\\
% Lemma 5.5.3, P(N=0)>0, then P(Z_n->infty, or Z_n=0 for n large enough)=1
% 即以概率1繁殖到无穷或灭绝;直觉上有限Z无法以1概率维持,然后由Levy's 0-1 law可得。
% Lemma 5.5.4, {m^{-n}Z_n} is MG, {rho^{Z_n}} is MG for rho, s=L(s).
% prop 5.5.5, p_ex =1 for m<=1; p_ex =rho for m>1
\end{document}