%%%%%%%%%%%%%%%%%%%%%%%%%%%%%%%%%%%%%%%%%
% Short Sectioned Assignment
% LaTeX Template
% Version 1.0 (5/5/12)
%
% This template has been downloaded from:
% http://www.LaTeXTemplates.com
%
% Original author:
% Frits Wenneker (http://www.howtotex.com)
%
% License:
% CC BY-NC-SA 3.0 (http://creativecommons.org/licenses/by-nc-sa/3.0/)
%
%%%%%%%%%%%%%%%%%%%%%%%%%%%%%%%%%%%%%%%%%

%----------------------------------------------------------------------------------------
%	PACKAGES AND OTHER DOCUMENT CONFIGURATIONS
%----------------------------------------------------------------------------------------

\documentclass[paper=a4, fontsize=11pt]{scrartcl} % A4 paper and 11pt font size

\usepackage[T1]{fontenc} % Use 8-bit encoding that has 256 glyphs
\usepackage{fourier} % Use the Adobe Utopia font for the document - comment this line to return to the LaTeX default
\usepackage[english]{babel} % English language/hyphenation
\usepackage{amsmath,amsfonts,amsthm} % Math packages

\usepackage{sectsty} % Allows customizing section commands
\allsectionsfont{\centering \normalfont\scshape} % Make all sections centered, the default font and small caps
\usepackage{url}
\usepackage{fancyhdr} % Custom headers and footers
\pagestyle{fancyplain} % Makes all pages in the document conform to the custom headers and footers
\fancyhead{} % No page header - if you want one, create it in the same way as the footers below
\fancyfoot[L]{} % Empty left footer
\fancyfoot[C]{} % Empty center footer
\fancyfoot[R]{\thepage} % Page numbering for right footer
\renewcommand{\headrulewidth}{0pt} % Remove header underlines
\renewcommand{\footrulewidth}{0pt} % Remove footer underlines
\setlength{\headheight}{13.6pt} % Customize the height of the header

\numberwithin{equation}{section} % Number equations within sections (i.e. 1.1, 1.2, 2.1, 2.2 instead of 1, 2, 3, 4)
\numberwithin{figure}{section} % Number figures within sections (i.e. 1.1, 1.2, 2.1, 2.2 instead of 1, 2, 3, 4)
\numberwithin{table}{section} % Number tables within sections (i.e. 1.1, 1.2, 2.1, 2.2 instead of 1, 2, 3, 4)

\setlength\parindent{0pt} % Removes all indentation from paragraphs - comment this line for an assignment with lots of text
\def \cov {\text{Cov}}
\def \var {\text{Var}}

\title{HW of STATS 310B: Theory of Statistics II}
% Amir-Dembo
\author{sogapalag}

\date{\normalsize\today}

\begin{document}

\maketitle

\begin{itemize}
	\item[5.1.14](Reflection principle)
	\begin{itemize}
		\item[(a)] by symmetry $P(>0)=P(<0)$, thus $P(\geq 0)\geq 1/2$.\qed
		% 注意这里还只是对称(不一定SRW),在k, >x之后,S_n-S_k<0,也是仍有可能>x
		\item[(b)] clearly $(\tau=k, S_n-S_k\geq 0)$ are disjoint and sufficient to $S_n\geq x$, so left inequality holds; right followed by (a).\qed
		\item[(c)] clearly $\max>x$ iff $\tau\leq n$, then followed by (b).\qed
		\item[(d)] for symm SRW, left in (b) is '=', (and $S_n\geq x$ can be treated as $S_n>x-1$) note $P(\geq 0)=(1+P(=0))/2$, thus
		\begin{align}
			P(S_n\geq x) &= \sum P(\tau) \frac{1+P(=0)}{2}\\
				&= \frac{1}{2}(P(\max\geq x)+ P(S_n=x))
		\end{align}
		by hint let even-th sign change $+1\stackrel{-2}\rightarrow -1$, instead with reflection $+2$, thus hint equality holds. then 
		\begin{align}
			P(Z_{2n+1}\geq r) &= P(Z_{2n+1}\geq r | S_1=-1)\\
				&= P(\max_{k=1}^{2n+1} S_k\geq 2r-1|S_1=-1)\\
				&= P(\max_{j=1}^{2n} S_j\geq 2r)\\
				&= 2P(S_{2n}\geq 2r) - P(S_{2n}=2r)\\
				&= 2P(S_{2n}\geq 2r+2)+ P(S_{2n}=2r)\\
				&= 2P(S_{2n+1}\geq 2r+1)\\
				&= P(|S_{2n+1}|\geq 2r+1)
		\end{align}
		i.e. $Z_{2n+1}\stackrel{\mathcal{D}}{=} (|S_{2n+1}|-1)/2$.\qed
	\end{itemize}
	% 5.3.24 Pólya’s urn, r,b,c_k 每一步加上额外同颜色
	\item[5.3.26]
	\begin{itemize}
		\item[(a)] by Doob's $L_p$ martingale convergence.\qed
		\item[(b)] probability is trivial, since each of ${n \choose l}$ is has the form of RHS. and as $n\rightarrow \infty$, $M_\infty= x=l/n$, thus
		\begin{align}
			f(x)\propto P(.)&\propto \prod^l (1+\frac{\beta-1}{i})\prod^{n-l} (1+\frac{\alpha-1}{j})
		\end{align}
		and note 
		\begin{align}
			\prod^l (1+\frac{\beta-1}{i}) &= \exp(\sum^l \log (1+\frac{\beta-1}{i}))\\
			&\approx \exp(\sum^l \frac{\beta-1}{i}))\\
			&\approx \exp(\log(l)(\beta-1)))\\
			&= l^{\beta-1}
		\end{align}
		thus
		\begin{align}
			f(x) \propto x^{\beta-1}(1-x)^{\alpha-1}
		\end{align}
		is beta distribution.\qed
		\item[(c)] by Doob's maximal inequality, and $M_{\infty}$ is uniform which $E[M_{\infty}]=1/2$.\qed
	\end{itemize}
	\item[5.3.28] since $E[X_{n+1}|.]= (1-b_n)X_n + b_nE[B_n|.]= X_n$, is MG, and obviously bounded integrable, thus $X_n\stackrel{a.s.}{\rightarrow} X_{\infty}$. and by Levy's 0-1 law, $E[X_n|F_n]\stackrel{a.s.}{\rightarrow} E[X_\infty|F_\infty]$. and note only $P(0|0)=P(1|1)=1$, thus $X_{\infty}\in \{0,1\}$. and $E[X_\infty]=E[..E[X_{n+1}|.]..] = E[X_0]=X_0$, implies $P(X_\infty=1|F_0)=X_0$.\qed
\end{itemize}
\end{document}