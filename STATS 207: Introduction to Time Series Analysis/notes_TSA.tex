%%%%%%%%%%%%%%%%%%%%%%%%%%%%%%%%%%%%%%%%%
% Short Sectioned Assignment
% LaTeX Template
% Version 1.0 (5/5/12)
%
% This template has been downloaded from:
% http://www.LaTeXTemplates.com
%
% Original author:
% Frits Wenneker (http://www.howtotex.com)
%
% License:
% CC BY-NC-SA 3.0 (http://creativecommons.org/licenses/by-nc-sa/3.0/)
%
%%%%%%%%%%%%%%%%%%%%%%%%%%%%%%%%%%%%%%%%%

%----------------------------------------------------------------------------------------
%	PACKAGES AND OTHER DOCUMENT CONFIGURATIONS
%----------------------------------------------------------------------------------------

\documentclass[paper=a4, fontsize=11pt]{scrartcl} % A4 paper and 11pt font size

\usepackage[T1]{fontenc} % Use 8-bit encoding that has 256 glyphs
\usepackage{fourier} % Use the Adobe Utopia font for the document - comment this line to return to the LaTeX default
\usepackage[english]{babel} % English language/hyphenation
\usepackage{amsmath,amsfonts,amsthm,bm} % Math packages

\usepackage{sectsty} % Allows customizing section commands
\allsectionsfont{\centering \normalfont\scshape} % Make all sections centered, the default font and small caps
\usepackage{url}
\usepackage{fancyhdr} % Custom headers and footers
\pagestyle{fancyplain} % Makes all pages in the document conform to the custom headers and footers
\fancyhead{} % No page header - if you want one, create it in the same way as the footers below
\fancyfoot[L]{} % Empty left footer
\fancyfoot[C]{} % Empty center footer
\fancyfoot[R]{\thepage} % Page numbering for right footer
\renewcommand{\headrulewidth}{0pt} % Remove header underlines
\renewcommand{\footrulewidth}{0pt} % Remove footer underlines
\setlength{\headheight}{13.6pt} % Customize the height of the header

\numberwithin{equation}{section} % Number equations within sections (i.e. 1.1, 1.2, 2.1, 2.2 instead of 1, 2, 3, 4)
\numberwithin{figure}{section} % Number figures within sections (i.e. 1.1, 1.2, 2.1, 2.2 instead of 1, 2, 3, 4)
\numberwithin{table}{section} % Number tables within sections (i.e. 1.1, 1.2, 2.1, 2.2 instead of 1, 2, 3, 4)

\setlength\parindent{0pt} % Removes all indentation from paragraphs - comment this line for an assignment with lots of text
\def \cov {\text{Cov}}
\def \var {\text{Var}}
\def \arg {\text{arg}}
\def \logit {\text{logit}}

\title{Notes of STATS 207: Introduction to Time Series Analysis}
% Time Series Analysis and Its Applications With R Examples. 4th ed. Robert H. Shumway  David S. Stoffer
\author{sogapalag}

\date{\normalsize\today}

\begin{document}

\maketitle
% 一般指weak S., 即mean, autocov与时间t无关(只与时间间隔h有关);满足autocov的,称为trend S.
Stationarity\\
% 对于S.的 x,y的cross-cov满足lag, h
Joint Stationarity\\
% Wold Decomposition
Linear process\\
% Linear regression
% t-test; F-test; aov; AIC;BIC;AICc
% lm, ksmooth, lowess
% AR(p), x_t ~ x_{t-1}+...+x_{t-p}
% B: backshift.    phi(B)x_t = w_t
Autoregressive(AR); AR($p$) operator $\phi(B)=(1-\phi_1B-\dots-\phi_p B^p)$. $\phi(B)x_t=w_t$.\\
% AR(1) ~ solution LP x_t=\sum_j phi^j w_{t-j}
% solution技巧 展开1/(1-phiB)
For AR(1) $|\phi|<1$ S. solution
\begin{align}
	x_t = \sum_{j=0}^\infty \phi^j w_{t-j}\\
	\gamma(\pm h) = \frac{\sigma_w^2 \phi^h}{1-\phi^2}, h\geq 0\\
	\rho(h) = \gamma(h)/\gamma(0) = \phi^h\\
	\rho(h) = \phi \rho(h-1), h=1,2,...
\end{align}
$|\phi|>1$ S. solution $x_t=-\sum_{j=1}^\infty \phi^{-j}w_{t+j}$. (useless, future info)\\
% MA(q), x_t = w_t + t_1w_{t-1} +...+ t_q w_{t-q}
% x_t = t(B) w_t
moving average(MA); MA($q$) operator $\theta(B)=1+\theta_1 B+\dots+\theta_q B^q$. $x_t=\theta(B)w_t$.\\
% MA,non-uniqueness; 即有不同的(t,sigma)使得acf相同;所以无法区分,默认选择可以casual AR; (invertible). 即x,w看作符号对换, phi ~ -t
% ARMA def 3.5
autoregressive moving average(ARMA); ARMA($p,q$), i.e. $\phi(B)x_t=\theta(B)w_t$.\\
% parameter redundant, 一个例子即x_t=w_t两边同乘上,看上去cor影响很大,实际上只是noise。
% 冗余,即需要消因子
def, AR,MA polynomial, $z\in\mathbb{C}$, that
\begin{align}
	\phi(z) = 1 -\phi_1 z - \dots - \phi_p z^p, \phi_p\neq 0\\
	\theta(z) = 1 + \theta_1 z + \dots + \theta_q z^q, \theta_q \neq 0
\end{align}
% 即不依赖未来的单边LP
def, casual, if time-series can be written as one-side linear process
\begin{align}
	x_t = \sum_{j=0}^\infty \psi_j w_{t-j} =\psi(B)w_t
\end{align}
where $\psi(B)=\sum_{j=0}^\infty \psi_jB^j$, $\sum_{j=0}^\infty |\psi_j|<\infty$, wlog set $\psi_0=1$.\\
% 单位圆内无根
prop, ARMA($p,q$) is casual iff $\phi(z)\neq 0,\forall|z|\leq 1$. coeff solved by $\psi(z) = \theta(z)/\phi(z)$.\\
def, invertible, if can written as
\begin{align}
	\pi(B)x_t = \sum_{j=0}^\infty \pi_j x_{t-j} = w_t
\end{align}
where $\pi(B)=\sum_{j=0}^\infty\pi_jB^j$, $\sum_{j=0}^\infty |\pi_j|<\infty$; wlog set $\pi_0=1$.\\
% 类似地
prop, invertible iff $\theta(z)\neq 0,\forall |z|\leq 1$.\\
% 某种情况视作给定 中间的变量时,的corr
partial autocorrelation function(PACF)
\begin{align}
	\phi_{11}=\text{corr}(x_{t+1}, x_t)\\
	\phi_{hh}=\text{corr}(x_{t+h}-\widehat{x}_{t+h}, x_t - \widehat{x}_t), h\geq 2
\end{align}
where $\widehat{x}$ is regression that
\begin{align}
	\widehat{x}_{t+h} = \beta_1 x_{t+h-1}+\dots +\beta_{h-1}x_{t+1}
\end{align}
% (3.54) S.回归,反演性
% prop3.3, BLP (3.60) 最佳线性预测;(3.64)
for one-step prediction, coeff and error
\begin{align}
	\phi_n = \Gamma_n^{-1}\gamma_n\\
	E[(-)^2] = \gamma(0) - \gamma_n'\Gamma_n^{-1}\gamma_n
\end{align}
% iteratively计算coeff,error 解决避免求矩阵inverse
The Durbin-Levinson Algorithm\\
% iteratively计算predict, error
The Innovations Algorithm\\
% example3.23 直觉上stationary长预测显然几乎没额外信息,故-> mu, gamma(0)
% def3.10 Yule-Walker estimator 即AR(p)系数,error的估计形式
% 估计 的表现
% large sample result, Y-W, PACF 渐近分布AN
% 对于small sample, 可以bootstrap出参数分布
% ex3.30 The Newton–Raphson and Scoring Algorithms; iteratively计算似然MLE
% 即差分d次后 为 ARMA(p,q) 
def3.11, ARIMA$(p,d,q)$.\\
% 注意过度差分
% Q-statistic, 即model后的residuals应该要iid的mean-zero的,(Gaussian不一定)
% (seasonal) def,3.12, SARIMA; 即额外做周期差分
% DFT
% 4.1,4.2 (S.)ACF的谱分析(谱密度)(对称)
for stationary process, $\sum|\gamma|<\infty$, that
\begin{align}
	\gamma(h) = \int_{-\frac{1}{2}}^{\frac{1}{2}} e^{2\pi i \omega h} f(\omega)d\omega, h=0,\pm 1,\pm 2,...\\
	f(\omega) = \sum_{h=-\infty}^\infty \gamma(h)e^{-2\pi i \omega h}, -1/2\leq \omega\leq 1/2
\end{align}
% 4.20~4.22 (S.)input => output(卷积)的ACF谱关系
for a linear filter $y_t = \sum a_j x_{t-j}$, $\sum |a_j|<\infty$, frequency response function $A(\omega)=\sum a_j e^{-2\pi i\omega j}$, that
\begin{align}
	f_y(\omega) = |A(\omega)|^2 f_x(\omega)
\end{align}
% 由noise f=gamma=1,(sigma^2); A即theta/phi结合上式可得。直接计算x的gamma,f可验证
ARMA($p,q$)'s spectral density is
\begin{align}
	f_x(\omega) = \sigma_w^2 \frac{|\theta(e^{-2\pi i \omega})|^2}{|\phi(e^{-2\pi i \omega})|^2}
\end{align}
% 即S.P可以表示为(不相关increment)随机过程的积分
for mean-zero stationary process with spectral distribution $F$, $\exists Z(\omega)$ complex-valued stochastic process on $\omega\in[-1/2,1/2]$, s.t.
\begin{align}
	x_t = \int_{-\frac{1}{2}}^{\frac{1}{2}} e^{2\pi i \omega t} dZ(\omega)
\end{align}
and $\var(Z(\omega_2) - Z(\omega_1)) = F(\omega_2)-F(\omega_1)$.\\
% 周期图,与sample ACF的关系,类似 谱密度与ACF
% as n->infty, E[I]->f
periodogram, $I(\omega_j)=|D|^2=\sum_{|h|<n}\widehat{\gamma}(h)e^{-2\pi i\omega_j h}$.\\
% ex4.11 ACF matrix, large n, 特征值~f, X=GY
% (4.38)-(4.45) D_s(谱幅sin项),D_c COV的一些性质
% LP, 2I/f ->{d} xi_2^2(chi-squared 2维)(4.51)近似f置信区间
% 于是, E[I]~f, var(I)~f^2 (by chisq, mean k, var 2k)
% 由于上var不->0,section4.4 -(4.58)用一段(bandwidth)作平滑估计f
% prop4.7, S. P的谱密度g 存在casual AR(p)的谱密度f趋近
% section4.6, cross-spec f_{xy}, squared coherence
% prop4.8, vector S. P的谱表示; coherence估计的F-test
% LP, f_yy = |A|f_xx, f_yx = A f_xx (4.108). 高通低通
% 对4.108式的解释,A拆分|A|e^{-}, 即交叉谱相对于谱密度 等价于对原每个频率进行了相应的delay和振幅放大
% prop4.9 推广到vector LP
% optimal filter
% section4.10, 推广到 多维(space)序列
% sec5.1 Long Memory ARMA and Fractional Differencing; lib(fracdiff)
% unit root test(测试RW)(5.32);for ARMA, adf-test, pp-test, lib(tseries)
% ARCH,GARCH; martingale
GARCH, $z_t\sim N(0,1)$, $r_t$ return rate,
\begin{align}
	r_t = \sigma_t z_t\\
	\sigma_t^2 = \alpha_0 +\sum_{i=1}^q \alpha_i r_{t-i}^2 + \sum_{j=1}^p \beta_j \sigma_{t-j}^2
\end{align}
$\theta=(\alpha_0,\bm{\alpha},\bm{\beta})$, (quasi)MLE.
\end{document}