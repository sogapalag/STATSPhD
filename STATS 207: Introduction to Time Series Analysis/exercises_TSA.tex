%%%%%%%%%%%%%%%%%%%%%%%%%%%%%%%%%%%%%%%%%
% Short Sectioned Assignment
% LaTeX Template
% Version 1.0 (5/5/12)
%
% This template has been downloaded from:
% http://www.LaTeXTemplates.com
%
% Original author:
% Frits Wenneker (http://www.howtotex.com)
%
% License:
% CC BY-NC-SA 3.0 (http://creativecommons.org/licenses/by-nc-sa/3.0/)
%
%%%%%%%%%%%%%%%%%%%%%%%%%%%%%%%%%%%%%%%%%

%----------------------------------------------------------------------------------------
%	PACKAGES AND OTHER DOCUMENT CONFIGURATIONS
%----------------------------------------------------------------------------------------

\documentclass[paper=a4, fontsize=11pt]{scrartcl} % A4 paper and 11pt font size

\usepackage[T1]{fontenc} % Use 8-bit encoding that has 256 glyphs
\usepackage{fourier} % Use the Adobe Utopia font for the document - comment this line to return to the LaTeX default
\usepackage[english]{babel} % English language/hyphenation
\usepackage{amsmath,amsfonts,amsthm,bm} % Math packages

\usepackage{sectsty} % Allows customizing section commands
\allsectionsfont{\centering \normalfont\scshape} % Make all sections centered, the default font and small caps
\usepackage{url}
\usepackage{fancyhdr} % Custom headers and footers
\pagestyle{fancyplain} % Makes all pages in the document conform to the custom headers and footers
\fancyhead{} % No page header - if you want one, create it in the same way as the footers below
\fancyfoot[L]{} % Empty left footer
\fancyfoot[C]{} % Empty center footer
\fancyfoot[R]{\thepage} % Page numbering for right footer
\renewcommand{\headrulewidth}{0pt} % Remove header underlines
\renewcommand{\footrulewidth}{0pt} % Remove footer underlines
\setlength{\headheight}{13.6pt} % Customize the height of the header

\numberwithin{equation}{section} % Number equations within sections (i.e. 1.1, 1.2, 2.1, 2.2 instead of 1, 2, 3, 4)
\numberwithin{figure}{section} % Number figures within sections (i.e. 1.1, 1.2, 2.1, 2.2 instead of 1, 2, 3, 4)
\numberwithin{table}{section} % Number tables within sections (i.e. 1.1, 1.2, 2.1, 2.2 instead of 1, 2, 3, 4)

\setlength\parindent{0pt} % Removes all indentation from paragraphs - comment this line for an assignment with lots of text
\def \cov {\text{Cov}}
\def \var {\text{Var}}
\def \arg {\text{arg}}
\def \logit {\text{logit}}

\title{Exercises of STATS 207 Time Series Analysis}
% Time Series Analysis and Its Applications With R Examples. 4th ed. Robert H. Shumway  David S. Stoffer
\author{sogapalag}

\date{\normalsize\today}

\begin{document}

\maketitle
\begin{itemize}
	\item[1.6] note $\var(x_s,x_t)=\delta_{s,t}$, then shall be easy.
	\item[1.8] (d) $\sqrt{\min/\max}$, (e) $x_t-x_{t-1}$.
	\item[1.9] recall indep, mean-zero, and $\cos(\alpha-\beta)$.
	\item[1.10] note $E[X_{t+l}X_t]=\gamma(l)$, thus $\text{MSE}(A)=(1+A^2)\gamma(0)-2A\gamma(l)$, implies $A=\gamma(l)/\gamma(0)=\rho(l)$ reach the minimum $(1-\rho^2(l))\gamma(0)$.
	\item[1.11]
	\begin{align}
		\cov(x_{t+h}, x_t) &= \cov(\sum_k \psi_k w_{t+h-k}, \sum_j\psi_j w_{t-j})\\
			&= \sum_i\psi_i\psi_{i+h}\var( w_{t-i})\\
			&= \sigma^2\sum_i\psi_i\psi_{i+h}
	\end{align}
	recall Cauchy-criterion
	\begin{align}
		E[|S_m-S_n|^2] &= E[(\sum_{m\leq j\leq n}\psi_jw_{t-j})^2]\\
			&= (\mu^2+\sigma^2)E[\sum_{m\leq j\leq n}\psi_j^2]\\
			&\rightarrow 0
	\end{align}
	thus $x_t\stackrel{L2}{\rightarrow} x$.
	\item[1.12] trivial by def.
	\item[1.13] $\gamma(0)=\sigma_w^2(1+\theta^2)+\sigma_u^2$, $\gamma(\pm 1)=-\theta \sigma_w^2$, else $0$, easy to compute $\rho_y(h)$. for CCF, only $h=0,-1$ nonzero; trivial satisfy joint-S.
	\item[1.14] recall mgf, $E[y_t]=\exp(\mu_x+\gamma(0)/2)$. note sum of correlated normal still normal, i.e. $x_{t+h}+x_t \sim N(2\mu_x, 2\gamma(0)+2\gamma(h)$, hence ACF of $y_t$ is
	\begin{align}
		\gamma_y(h) &= \cov(y_{t+h},y_t)\\
			&= \exp(2\mu_x + \gamma(0)+\gamma(h)) - \exp^2(\mu_x+\gamma(0)/2)\\
			&= (e^{\gamma(h)}-1)\exp(2\mu_x + \gamma(0))
	\end{align}
	\item[1.15] mean-zero, only $\gamma(0)=\sigma^4$, i.e. stationary.
	\item[1.16] trivial mean-zero, and ACF, $h>0$
	\begin{align}
		\gamma(h) &= E[\sin(2\pi Ut)\sin(2\pi U(t+h))]\\
			&= .5*E[\cos(2\pi hU) - \cos(2\pi(2t+h)U)]\\
			&= 0\\
		\gamma(0) &= .5*E[1]=.5
	\end{align}
	i.e. weakly stationary. since $x_t,x_s$ trivially not same, hence not strictly S.
	\item[1.17] joint cf
	\begin{align}
		\phi &= E[\exp(i\sum_{i=1}^n t_i x_i)]\\
			&= E[\exp(i(t_n w_n-t_1\theta w_0+\sum_{i=1}^{n-1}(t_i-t_{i+1}\theta)w_i]\\
			&= \phi_w(t_n)\phi_w(-t_1\theta)\prod_{i=1}^{n-1}\phi_w(t_i-t_{i+1}\theta)
	\end{align}
	then use partial-d to calculate moments to show strcitly S(obvious).
	\item[1.18]
	\begin{align}
		\sum_{h=-\infty}^\infty |\gamma(h)| &\leq \sum_{h=-\infty}^\infty \sigma^2 \sum_{j=-\infty}^\infty |\psi_j\psi_{j+h}|\\
			&= \sigma^2 (\sum_{j=-\infty}^\infty |\psi_j|)^2\\
			&< \infty
	\end{align}
	\item[1.19]
	\begin{itemize}
		\item[(d)]
		\begin{align}
			\var(\overline{x}) &= \frac{1}{n}(\gamma(0)+\frac{2(n-1)}{n}\gamma(1))\\
			\var(\overline{x})|_{\theta=1} &= \frac{2(2n-1)}{n^2}\sigma_w^2\\
			\var(\overline{x})|_{\theta=0} &= \frac{1}{n}\sigma_w^2\\
			\var(\overline{x})|_{\theta=-1} &= \frac{2}{n^2}\sigma_w^2
		\end{align}
		\item[(e)] obviously $\theta=-1$ with $\rightarrow 0$ faster, which follow the intuition cancel most randomness.
	\end{itemize}
	\item[1.20] obvious $n$ larger, $\hat{\gamma}(h_{>0})$ more close to zero.
	\item[1.25]
	\begin{itemize}
		\item[(a)] let $Y=a'X$, that $\var(Y)= a'\Gamma a\geq 0$, where $\Gamma$ is cov-matrix of that stationary process. i.e. non-negative definite.
		\item[(b)] the estimate $\widehat{\var}(Y)=a'\widehat{\Gamma}a\geq 0$.
	\end{itemize}
	\item[1.26] obvious.
	\item[1.27] since stationary, thus
	\begin{align}
		V_x(h) &= \frac{1}{2} E[((x_{s+h}-\mu_x)- (x_s-\mu_x))^2]\\
			&= \gamma(0) - \gamma(h)
	\end{align}
	\item[1.28] by $x_t$'s formula, intuitively hold, formally each item's ratio $\frac{x_{t+h}-\overline{x}}{x_t-\overline{x}}=\frac{t+h-(n+1)/2}{t-(n+1)/2}\rightarrow 1$, for most $t$, since $h$ fixed. hence $\widehat{\rho}\rightarrow 1$.
	\item[1.29] note $E[\sqrt{n}\overline{x}]=0$, and recall
	\begin{align}
		\var(\sqrt{n} \overline{x}) &= n\frac{1}{n}\sum_{h=-n}^n(1-\frac{|h|}{n})\gamma(h)\\
			&\leq \sum_{h=-n}^n\gamma(h)\\
			&\rightarrow 0
	\end{align}
	hence $\sqrt{n}\overline{x}\stackrel{p}{\rightarrow}0$.
	\item[1.30]
	\begin{align}
		E[\overline{\gamma}(h)- \widehat{\gamma}(h)] &\sim \gamma(h)-(\gamma(h)+\var(\overline{x}))\\
			&\sim \var(\overline{x})\\
			&\sim \var((\sum_i \psi_i)\overline{w})\\
			&\sim 1/\sqrt{n}
	\end{align}
	thus $\sqrt{n}(\overline{\gamma}(h)- \widehat{\gamma}(h)) \sim o(1)$ by Markov's.
\end{itemize}

\end{document}