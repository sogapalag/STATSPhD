%%%%%%%%%%%%%%%%%%%%%%%%%%%%%%%%%%%%%%%%%
% Short Sectioned Assignment
% LaTeX Template
% Version 1.0 (5/5/12)
%
% This template has been downloaded from:
% http://www.LaTeXTemplates.com
%
% Original author:
% Frits Wenneker (http://www.howtotex.com)
%
% License:
% CC BY-NC-SA 3.0 (http://creativecommons.org/licenses/by-nc-sa/3.0/)
%
%%%%%%%%%%%%%%%%%%%%%%%%%%%%%%%%%%%%%%%%%

%----------------------------------------------------------------------------------------
%	PACKAGES AND OTHER DOCUMENT CONFIGURATIONS
%----------------------------------------------------------------------------------------

\documentclass[paper=a4, fontsize=11pt]{scrartcl} % A4 paper and 11pt font size

\usepackage[T1]{fontenc} % Use 8-bit encoding that has 256 glyphs
\usepackage{fourier} % Use the Adobe Utopia font for the document - comment this line to return to the LaTeX default
\usepackage[english]{babel} % English language/hyphenation
\usepackage{amsmath,amsfonts,amsthm,bm} % Math packages

\usepackage{sectsty} % Allows customizing section commands
\allsectionsfont{\centering \normalfont\scshape} % Make all sections centered, the default font and small caps
\usepackage{url}
\usepackage{fancyhdr} % Custom headers and footers
\pagestyle{fancyplain} % Makes all pages in the document conform to the custom headers and footers
\fancyhead{} % No page header - if you want one, create it in the same way as the footers below
\fancyfoot[L]{} % Empty left footer
\fancyfoot[C]{} % Empty center footer
\fancyfoot[R]{\thepage} % Page numbering for right footer
\renewcommand{\headrulewidth}{0pt} % Remove header underlines
\renewcommand{\footrulewidth}{0pt} % Remove footer underlines
\setlength{\headheight}{13.6pt} % Customize the height of the header

\numberwithin{equation}{section} % Number equations within sections (i.e. 1.1, 1.2, 2.1, 2.2 instead of 1, 2, 3, 4)
\numberwithin{figure}{section} % Number figures within sections (i.e. 1.1, 1.2, 2.1, 2.2 instead of 1, 2, 3, 4)
\numberwithin{table}{section} % Number tables within sections (i.e. 1.1, 1.2, 2.1, 2.2 instead of 1, 2, 3, 4)

\setlength\parindent{0pt} % Removes all indentation from paragraphs - comment this line for an assignment with lots of text
\def \cov {\text{Cov}}
\def \var {\text{Var}}

\title{STATS 300A: Theory of Statistics I}
% https://web.stanford.edu/~lmackey/stats300a/
% Theoretical Statistics: Topics for a Core Course
% Testing Statistical Hypotheses, 3rd
\author{sogapalag}

\date{\normalsize\today}

\begin{document}

\maketitle
% 即给定该统计,分布与参数无关,e.g. max Uniform, order statistic
% 可以用于equal risk Regenerate dataset
Sufficient Statistic: $\mathbf{X}|T=t$ not depend on $\theta$.\\
% 直觉上,T决定了某个截面,f可拆分 scale g,和shape h,当T给定时,h的某个截面scale是相同的(注,虽然看上去g仍是theta的函数,但并不影响当theta取定时g是const;所以归一化h某个截面时与theta无关)
Thm, Neyman-Fisher Factorization Criterion(NFFC): sufficient $T$ iff $f(\mathbf{x};\theta) = g_\theta(T(\mathbf{x})) h(\mathbf{x})$.\\
% A是归一化;T是SS;eta是一组参数向量函数(canonical, if eta_i(theta)=theta_i);维度,即T,eta的维度
Exponential Families: $f(\mathbf{x};\bm{\theta})=h(\mathbf{x})\exp(\langle\bm{\eta}(\bm{\theta}),\mathbf{T}(\mathbf{x})\rangle - A(\bm{\theta}))$\\
% 即能够归一化的自然定义的eta 的空间
canonical form; natural parameters; natural parameter space, $\bm{\eta}\in \Theta$, $\exists A(\bm{\eta})$ normalized constant s.t. $\int f(\mathbf{x},\bm{\eta})d\mu(\mathbf{x})=1$, equivalently
\begin{align}
	\Theta = \{\bm{\eta}: 0<\int \exp(\langle \bm{\eta}, \mathbf{T}(\mathbf{x})\rangle)h(\mathbf{x})d\mu(\mathbf{x}) <\infty\}
\end{align}
property, if $X_1,...,X_n \stackrel{i.i.d.}{\sim} f(x;\bm{\theta})$ EF, then $f(\mathbf{x};\bm{\theta})$ is EF.\\
property, if $f$ integrable, $\bm{\eta}\in\Theta$, Then
\begin{align}
	G(f,\bm{\eta}) = \int f(\mathbf{x}) \exp(\langle \bm{\eta}, \mathbf{T}(\mathbf{x})\rangle)h(\mathbf{x})d\mu(\mathbf{x})
\end{align}
is infinitely differentiable w.r.t. $\bm{\eta}$.\\
% 对A求偏导可以得到关于T的moment
notice $ A(\bm{\eta})=\ln G(1,\bm{\eta})$, $E_{\bm{\eta}}[f(\mathbf{X})] = G(f,\bm{\eta})/G(1,\bm{\eta})$.
\begin{align}
	\frac{\partial A}{\partial \eta_i} = E_{\bm{\eta}}[T_i(\mathbf{x})]; \frac{\partial^2 A}{\partial \eta_i \partial \eta_j} = \cov_{\bm{\eta}}(T_i(\mathbf{x}),T_j(\mathbf{x}))
\end{align}
def, minimal SS $T$: if all SS $T'$ that $T$ is  function of $T'$; equiv, $T(x)=T(y)$ whenever $T'(x)=T'(y)$.\\
Thm, $T$ is min-SS if $T(x)=T(y) \Leftrightarrow c_{x,y} = f(x;\theta)/f(y;\theta)$ independent of $\theta$.\\
def, ancillary, which distribution not depend on $\theta$.\\
def, first-order ancillary, $E[V]$ not depend on $\theta$.\\
% 即只有常数函数f(T)的期望为常数,其它都与theta相关
def, complete, no non-constant function of $T$ is first-order ancillary. i.e. $E[f(T)]=0,\forall \theta$, implies $f(T)\stackrel{a.e.}{=}0$.\\
Bahadur's theorem: complete sufficient $\Rightarrow$ minimal sufficient.\\
Thm, EF, full-rank $\Rightarrow$ complete sufficient.\\
Basu's theorem: CS $T$, ancillary $V$ $\Rightarrow$ $T\perp V$, i.e. independent.\\
% 可用于优化估计
Rao-Blackwell Theorem: sufficient $T$, estimator $\delta$, define $\eta(T)=E[\delta(X)|T]$, for convex $L(\theta,.)$, Then $R(\theta,\eta)\leq R(\theta,\delta)$. Furthermore strictly convex, then equality when $\delta(X)=\eta(T)$ a.s.\\
% 即估计的期望等于 想估计的g(theta)
def, unbiased, $E[\delta(X)]=g(\theta)$.\\
% 虽然一致最小Risk的估计是不存在的,但对于 unbiased估计‘空间 可以找到一致最小Risk的
def, uniformly minimum risk unbiased estimator(UMRUE).\\
def, uniformly minimum variance unbiased estimator(UMVUE), i.e. when $L(\theta,d)=(\theta-d)^2$.\\
% 注:随意缩写CS(css),一些唯一、相等的,可能是a.s.,会有的省略不写
% CS 无偏估计h(T)的 一些'唯一'性质
Lehmann-Scheffe Theorem: CS $T$, unbiased $h(T)$ for $g(\theta)$, Then (a) only function of $T$ that is unbiased; (b) UMRUE under any convex loss function; (c) unique UMRUE under any strictly convex loss func; (d) unique UMVUE.
% 结合R-B,L-S thm,即有一些方法策略找UMRUE,一些例子Lec.4
\end{document}