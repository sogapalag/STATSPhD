%%%%%%%%%%%%%%%%%%%%%%%%%%%%%%%%%%%%%%%%%
% Short Sectioned Assignment
% LaTeX Template
% Version 1.0 (5/5/12)
%
% This template has been downloaded from:
% http://www.LaTeXTemplates.com
%
% Original author:
% Frits Wenneker (http://www.howtotex.com)
%
% License:
% CC BY-NC-SA 3.0 (http://creativecommons.org/licenses/by-nc-sa/3.0/)
%
%%%%%%%%%%%%%%%%%%%%%%%%%%%%%%%%%%%%%%%%%

%----------------------------------------------------------------------------------------
%	PACKAGES AND OTHER DOCUMENT CONFIGURATIONS
%----------------------------------------------------------------------------------------

\documentclass[paper=a4, fontsize=11pt]{scrartcl} % A4 paper and 11pt font size

\usepackage[T1]{fontenc} % Use 8-bit encoding that has 256 glyphs
\usepackage{fourier} % Use the Adobe Utopia font for the document - comment this line to return to the LaTeX default
\usepackage[english]{babel} % English language/hyphenation
\usepackage{amsmath,amsfonts,amsthm,bm} % Math packages

\usepackage{sectsty} % Allows customizing section commands
\allsectionsfont{\centering \normalfont\scshape} % Make all sections centered, the default font and small caps
\usepackage{url}
\usepackage{fancyhdr} % Custom headers and footers
\pagestyle{fancyplain} % Makes all pages in the document conform to the custom headers and footers
\fancyhead{} % No page header - if you want one, create it in the same way as the footers below
\fancyfoot[L]{} % Empty left footer
\fancyfoot[C]{} % Empty center footer
\fancyfoot[R]{\thepage} % Page numbering for right footer
\renewcommand{\headrulewidth}{0pt} % Remove header underlines
\renewcommand{\footrulewidth}{0pt} % Remove footer underlines
\setlength{\headheight}{13.6pt} % Customize the height of the header

\numberwithin{equation}{section} % Number equations within sections (i.e. 1.1, 1.2, 2.1, 2.2 instead of 1, 2, 3, 4)
\numberwithin{figure}{section} % Number figures within sections (i.e. 1.1, 1.2, 2.1, 2.2 instead of 1, 2, 3, 4)
\numberwithin{table}{section} % Number tables within sections (i.e. 1.1, 1.2, 2.1, 2.2 instead of 1, 2, 3, 4)

\setlength\parindent{0pt} % Removes all indentation from paragraphs - comment this line for an assignment with lots of text
\def \cov {\text{Cov}}
\def \var {\text{Var}}
\def \arg {\text{arg}}
\def \logit {\text{logit}}

\title{Exercises of Bayesian}
% Bayesian Data Analysis, 3rd ed. Andrew Gelman..
\author{sogapalag}

\date{\normalsize\today}

\begin{document}

\maketitle
\begin{itemize}
	\item[1.1] (c) intuitively, $1/2$ as $\sigma \nearrow$, $1$ as $\sigma\rightarrow 0$, showed in script Ex.1.1.r.
	\item[1.3] first each parent, proportion of heterozygotes when browned, is $P_1=2p(1-p)/(1-p^2)= 2p/(1+p)$. since random mating, that proportion of heterozygotes children when browned is
	\begin{align}
		P_2(H|B) &= \frac{2P_1(1-P_1)*(1/2)+ P_1^2*(1/2)}{1-P_1^2*0.25}\\
			&= \frac{2p}{1+2p}
	\end{align}
	this view kinda messy, another view is to think proportion of x, $P_1/2$; then this is indeed probability transfer x to children, calculate conditional prob also get above.\\
	For him with hetero-mate condition on $n$ browned-children, that
	\begin{align}
		P_3(H|B^n) &= \frac{P_2* 0.75^n}{P_2*0.75^n+ 1-P_1}\\
			&= \frac{2p*0.75^n}{2p*0.75^n+1}
	\end{align}
	then note which children hetero-prob is
	\begin{align}
		P_4 &= \frac{P_2*0.75^{n-1}*0.5+(1-P_2)*0.5}{P_2*0.75^n + (1-P_2)}\\
			&= \frac{p*0.75^{n-1}+0.5}{2p*0.75^n+1}
	\end{align}
	hence first granchildren blue-eye prob is
	\begin{align}
		P_5 &= \frac{P_4}{2}p
	\end{align}
	\item[1.6] observe twin brother,
	\begin{align}
		P(i|tb) = \frac{p_i}{p_i+p_f*0.5}=\frac{5}{11}
	\end{align}
	\item[2.1]
	\begin{align}
		p(\theta|y<3) \propto \theta^3(1-\theta)^3\sum_{y=0}^2{10\choose y} \theta^y(10-\theta)^{10-y}\\
		C = \sum_{y=0}^2 {10\choose y} B(3+y+1, 3+10-y+1)=0.00103426
	\end{align}
	\item[2.2] first calculate
	\begin{align}
		P(C_1|TT) = \frac{.5 * 0.4^2}{.5 * .4^2 + .5 *.6^2} = 0.3076923
	\end{align}
	then recall geom-dist, hence $E=P/.6+(1-P)/.4 = 2.24359$.
	\item[2.3] by CLT $\frac{Y-\mu}{\sqrt{n}\sigma}\sim N(0,1)$, where $n=1000$, $\mu=1000/6$, $\sigma=\sqrt{5/36}$, i.e. $Y\sim N(\mu, n\sigma^2)$. quatile numbers $147.2819, 158.7177, 166.6667, 174.6156, 186.0515$.
	\item[2.4]
	\begin{itemize}
		\item[(a)] by CLT, $Y|\theta \sim N(n\theta, n\theta(1-\theta))$. thus $y$'s prior-dist is $\sum p(\theta)p(y|\theta)$, skectched in code.
		\item[(b)] for $.05, .95$, use quantiles of left $.20$, right $.80$ to approx; for $.25,.75$, which is happend to be $\theta$ cdf, so use valleys of left-mid, mid-right to approx; $.5$ use quantile of mid $.5$ to approx.
	\end{itemize}
	\item[2.5]
	\begin{itemize}
		\item[(a)] recall beta func
		\begin{align}
			Pr(y=k) &= \int_0^1 {n\choose k} \theta^k(1-\theta)^{n-k}d\theta\\
				&= {n\choose k} B(k+1,n-k+1)
		\end{align}
		\item[(b)] posterior
		\begin{align}
			p(\theta|y) &\propto \theta^{\alpha-1}(1-\theta)^{\beta-1} \theta^y(1-\theta)^{n-y}\\
				&\sim \text{Beta}(\alpha+y,\beta+n-y)
		\end{align}
		thus post-mean $\frac{\alpha+y}{\alpha+\beta+n}$ lies between.
		\item[(c)] recall beta dist var,
		\begin{align}
			\var(\theta|y) &= \frac{(y+1)(n-y+1)}{(n+2)^2(n+3)}\\
				&\leq \frac{(n+2)^2/4}{(n+2)^2(n+3)}\\
				&< \frac{1}{12} = \var(\theta)
		\end{align}
		\item[(d)] intuitively, observation far from prior beta, i.e. consider $\alpha\gg \beta$, $y=0,n=1$.
	\end{itemize}
	\item[2.8] recall sufficient S.
	\item[2.9]
	\begin{itemize}
		\item[(a)] let $t=\alpha+\beta$, that $\frac{.6*.4}{t+1}=.3^2$, $t=5/3$, i.e. $\alpha=1, \beta=2/3$.
		\item[(b)] recall 2.5(b), and $\alpha,\beta \ll y,n$, so approx Beta$(y,n-y)$, and note $y,n\gg 1$, i.e. $\sigma^2\sim 1/n\approx 0$, cdf step-func at $.65$.
	\end{itemize}
	\item[2.10]
	\begin{itemize}
		\item[(a)] $p(N|y)\propto p(N)\mathbb{I}_{N\geq y}/N$. $C=.0004705$.
		\item[(b)] analytic mean
		\begin{align}
			E &= \sum p(N)\mathbb{I}_{N\geq y}/C\\
				&= .99^{202}/C\\
				&= 279.1
		\end{align}
		computation $sd=80.0$.
	\end{itemize}
	\item[2.11]
	\begin{itemize}
		\item[(a)] by grid approx, $p(\theta|y)\propto \frac{1}{100m}\prod_{i=1}^5 \frac{1}{1+(y_i-\theta)^2}$
	\end{itemize}
	\item[2.12] recall Jeffreys prior and Fisher information.
	\item[2.15]
	\begin{align}
		E[Z^m(1-Z)^n] &= \frac{B(\alpha+m,\beta+n)}{B(\alpha,\beta)}\\
			&= \frac{\Gamma(\alpha+m)\Gamma(\beta+n)\Gamma(\alpha+\beta)}{\Gamma(\alpha)\Gamma(\beta)\Gamma(\alpha+\beta+m+n)}\\
			&= \frac{(\alpha+m-1)\dots\alpha (\beta+n-1)\dots\beta}{(\alpha+\beta+m+n-1)\dots(\alpha+\beta)}
	\end{align}
	thus for mean $m=1,n=0$, $E[Z]=\frac{\alpha}{\alpha+\beta}$; for variance similar $EZ^2-E^2Z$.
	\item[2.16]
	\begin{align}
		p(y) &\propto \int_0^1{n\choose y} \theta^{y+\alpha-1}(1-\theta)^{n-y+\beta-1}d\theta\\
			&= {n\choose y}B(y+\alpha, n-y+\beta)
	\end{align}
	recall Gamma, trivially to see only when $\alpha=\beta=1$, $p(y)\propto 1$.
	\item[2.20]
	\begin{itemize}
		\item[(a)]
		\begin{align}
			p(\theta|y\geq k) &\propto \int_{y\geq k} e^{-\theta y} \theta^{\alpha-1}e^{-\beta\theta}dy\\
				&= \theta^{\alpha-2} e^{-(k+\beta)\theta}\\
				&\sim \text{Gamma}(\alpha-1, k+\beta)
		\end{align}
		thus post-mean $E= (\alpha-1)/(k+\beta)$, post-var $\var = (\alpha-1)/(k+\beta)^2$.
		\item[(b)] $\sim \text{Gamma}(\alpha, k+\beta)$.
		\item[(c)] intuitively this information is worst situation, since else implies more restricted rate. mean-variance property related $E[\var(\theta|y)]$, not $\var(\theta|y)$ always less, so not contracting.
	\end{itemize}
\end{itemize}

\end{document}