%%%%%%%%%%%%%%%%%%%%%%%%%%%%%%%%%%%%%%%%%
% Short Sectioned Assignment
% LaTeX Template
% Version 1.0 (5/5/12)
%
% This template has been downloaded from:
% http://www.LaTeXTemplates.com
%
% Original author:
% Frits Wenneker (http://www.howtotex.com)
%
% License:
% CC BY-NC-SA 3.0 (http://creativecommons.org/licenses/by-nc-sa/3.0/)
%
%%%%%%%%%%%%%%%%%%%%%%%%%%%%%%%%%%%%%%%%%

%----------------------------------------------------------------------------------------
%	PACKAGES AND OTHER DOCUMENT CONFIGURATIONS
%----------------------------------------------------------------------------------------

\documentclass[paper=a4, fontsize=11pt]{scrartcl} % A4 paper and 11pt font size

\usepackage[T1]{fontenc} % Use 8-bit encoding that has 256 glyphs
\usepackage{fourier} % Use the Adobe Utopia font for the document - comment this line to return to the LaTeX default
\usepackage[english]{babel} % English language/hyphenation
\usepackage{amsmath,amsfonts,amsthm,bm} % Math packages

\usepackage{sectsty} % Allows customizing section commands
\allsectionsfont{\centering \normalfont\scshape} % Make all sections centered, the default font and small caps
\usepackage{url}
\usepackage{fancyhdr} % Custom headers and footers
\pagestyle{fancyplain} % Makes all pages in the document conform to the custom headers and footers
\fancyhead{} % No page header - if you want one, create it in the same way as the footers below
\fancyfoot[L]{} % Empty left footer
\fancyfoot[C]{} % Empty center footer
\fancyfoot[R]{\thepage} % Page numbering for right footer
\renewcommand{\headrulewidth}{0pt} % Remove header underlines
\renewcommand{\footrulewidth}{0pt} % Remove footer underlines
\setlength{\headheight}{13.6pt} % Customize the height of the header

\numberwithin{equation}{section} % Number equations within sections (i.e. 1.1, 1.2, 2.1, 2.2 instead of 1, 2, 3, 4)
\numberwithin{figure}{section} % Number figures within sections (i.e. 1.1, 1.2, 2.1, 2.2 instead of 1, 2, 3, 4)
\numberwithin{table}{section} % Number tables within sections (i.e. 1.1, 1.2, 2.1, 2.2 instead of 1, 2, 3, 4)

\setlength\parindent{0pt} % Removes all indentation from paragraphs - comment this line for an assignment with lots of text
\def \cov {\text{Cov}}
\def \var {\text{Var}}
\def \arg {\text{arg}}

\title{Notes of Game Theory}
% Essentials of Game Theory. Kevin Leyton-Brown and Yoav Shoham, 2008
\author{sogapalag}

\date{\normalsize\today}

\begin{document}

\maketitle
% players set, action profile, utility function.
% matrix
Normal form game, $(N,A,u)$.\\
% 即目标一致,共同优化
Common-payoff Games(pure coordination), $u_i(a)=u_j(a),\forall i,j,a$.\\
Constant-sum game, $u_1(a)+u_2(a)=c,\forall a$. zero-sum game(pure competition) $c=0$\\
Mixed strategy, $\Pi(X)$ all probability distribution of any set $X$. Then set of mixed strategies for player $i$ is $S_i=\Pi(A_i)$.\\
Mixed-strategy profile, cartesian product $S_1\times \dots \times S_n$.\\
% 即 概率为正的行动集
Support, $\{a_i:s_i(a_i)>0\}$.\\
Expected utility of a mixed strategy, $u_i(s)=\sum_a u(a)\prod_j s_j(a_j)$.\\
% 即 使得某些人收益更好,且不会使别人收益更坏
Pareto domination: $\forall i, u_i(s)\geq u_i(s')$, and $\exists j\in N, u_j(s)>u_j(s')$.\\
% 即 没有P-d 的策略
Pareto optimality(strictly Pareto efficient), no Pareto domination.\\
Best response $s_i^*$ to $s_{-i}$: $u_i(s_i^*, s_{-i}) \geq u_i(s_i,s_{-i}),\forall s_i\in S_i$.\\
% 对于mixed, 要点是使得别人indifferent
Nash equilibrium, $s$: $\forall i$, $s_i$ is best response to $s_{-i}$. strict Nash, $>$.\\
Thm, every game with finite $N,A$, has al least one NE.\\
Maxmin: value $\max_{s_i}\min_{s_{-i}} u_i(s_i,s_{-i})$; strategy, $\arg \max_{s_i}\min_{s_{-i}} u_i(s_i,s_{-i})$.\\
Minmax, two-player: $\min_{s_i}\max_{s_{-i}} u_{-i}(s_i,s_{-i})$.\\
% 所有人针对j
Minmax, $n$-player: $\min_{s_{-j}}\max_{s_j} u_{j}(s_j,s_{-j})$.\\
Minimax theorem: any finite, two-player, zero-sum game, NE $=$ maxmin $=$ minmax.\\
% 即相比BS收益相差多少
Regret, $i$'s regret play $a_i$ if others $a_{-i}$, $r(a_i,a_{-i})=[\max_{a_i'} u_i(a_i',a_{-i})] - u_i(a_i,a_{-i})$.\\
Max regret, $\max_{a_{-i}} r(a_i,a_{-i})$.\\
Minimax regret, $\min_{a_i}\max_{a_{-i}} r(a_i,a_{-i})$.\\
Domination: strictly, $\forall s_{-i}\in S_{-i}, u_i(s_i,s_{-i})>u_i(s_i',s_{-i})$; weakly, $\geq$, and $\exists s_{-i}$,$>$; very weakly, $\geq$.\\
Dominant strategy: $s_i$ dominate $\forall$ other $s_i'$.\\
% 可用于简化(solve)游戏
Dominated strategy: dominated by some $s_i'$.\\
% 删减stictly的顺序没有影响
Church-Rosser property.\\
Rationalizable strategies\\
% 即策略可以correlated; joint描述下均衡
Correlated equilibrium, $(v,\pi,\sigma)$, rvs, joint distribution, vector of mapping $\sigma_i:D_i\mapsto A_i$. s.t. $\forall i,\sigma', E[u_i(\sigma(D))] \geq E[u_i(\sigma'(D))]$.\\
Thm, NE $\Rightarrow$ CE.\\
% 即容忍扰动的均衡
(Trembling-hand) perfect equilibrium $S$: $\exists$ sequence $S^0,S^1,...$ fully mixed-strategy profiles, s.t. $\lim S^n = S$ and in each $S^k$, $\forall i$ $s_i$ is BR to $s_{-i}^k$.\\
% 即单方面改变不会获得超过epsi收益
$\epsilon$-Nash $s$: fix $\epsilon>0$, $\forall i,\forall s_i'$, $u_i(s_i,s_{-i})\geq u_i(s_i',s_{-i}) - \epsilon$.\\
% 即别人扰动,自己不会跟着同样变动;在等收益至S‘,别人扰动到S’,自己也会回去S
Evolutionarily stable strategy: given symmetric two-player normal-form game, ESS $S$ iff $\exists\epsilon>0$, $\forall$ other $S'$, $u(S,(1-\epsilon)S+\epsilon S')>u(S',(1-\epsilon)S+\epsilon S')$; equiv $u(S,S)>u(S',S)$ or both $u(S,S)=u(S',S)$ and $u(S,S')>u(S',S')$.\\
weakly ESS, for equiv, last inequi $\geq$.\\
Thm, given symmetric two-player normal-form game, ESS $S$ $\Rightarrow$ NE $(S,S)$.\\
Thm, given symmetric two-player normal-form game, strict NE $(S,S)$ $\Rightarrow$ ESS $S$.\\

% tree
Perfect-information game (in extensive-form)\\
% 直觉上即backward induction
Thm, every (finite) perfect-information game in extensive form has a pure-strategy NE.\\
% 即某个节点为根及其后裔组成的game
Subgame\\
% 即在所有subgame里都是NE; 显然SPE=>NE;另外SPE仍然存在 PIGiEF
Subgame-perfect equilibrium(SPE)\\
% 显然SPE就是没有noncredible threat
(non)credible threat\\
% def5.1.1 即增加一个信息集(表示一些当前节点的不可区分)
Imperfect-information game (in extensive-form), attached information set $I$.\\
% 即i对于他的各节点行动是独立的
behavioral strategies\\
% 即i对于他的信息集,历史长度一样、等深历史节点同信息集,他的行动历史相同;直觉上,即i回溯历史也无法区分当前位置。game of PC, 即所有i都是PC; 显然PIG => gPC,因为信息集都是单的。
Perfect recall of $i$; game of perfect recall, $\forall i$.\\
% 在收益分布的意义上有等价替换
Thm, for gPR, mixed- equiv behavioral-.\\
% 即推广子game在等价信息集的意义上
Sequential equilibrium\\
Thm, every finite gPR, has a SE.\\
Thm, SPE $\Rightarrow$ SE.\\
% 定义无限repeated game的收益
Average reward; Discounted reward.\\
% v, minmax vectors
Enforceable, $r\geq v$.\\
Feasible, $r = \sum_a \alpha_a u(a)$, $\alpha_a\in \mathbb{Q}^+, \sum \alpha_a=1$.\\
% 直觉上即(1)显然比minmax要优,然后(2)可构造minmax作为惩罚的NE
Folk Theorem: (1) $r$ for NE of IR $G$ $\Rightarrow$ $r$ enforceable. (2) $r$ enforceable and feasible $\Rightarrow$ $r$ is some NE of IR $G$ with average reward.\\
Stochastic game(Markov game).\\
Behavioral strategy, $s_i(h_t,a_{i_j})$ the probability of playing $a_{i_j}$ for history $h_t$.\\
% 即given t, 策略只与当前状态有关
Markov strategy, $s_i(h_t, a_{i_j}) = s_i(h_t', a_{i_j})$ if $q_t=q_t'$ final state of which.\\
% 即任意t, 策略只与当前状态有关
Stationary strategy, $t_1,t_2$.\\
Markov perfect equilibrium, consists of only Markov-s, NE regardless of start states.\\
Thm, every $n$-player, general-sum, discounted reward, stochastic game has a MPE.\\
Thm, every $2$-player, general-sum, average reward, irreducible stochastic game has a NE.\\
% 类似folk thm
Thm, for every $2$-player, general-sum, irreducible stochastic game, feasiable and enforceable $r$, there $\exists$ NE, for average reward, as well as large discount.\\

% def7.1.1 即games有一个分布,(简单情况,初始分布是common prior),然后I,对game的区分集
Bayesian game: information sets.\\
% 即type分布有common prior,但每个人持有自己的type,(即knowledge his own payoff function, beliefs about others' and own payoffs, etc higher-order beliefs.)
Bayesian game: types.\\
% 即i的期望收益,知道type
\textit{Ex post} expected utility:
\begin{align}
	EU_i(s,\theta) = \sum_{a\in A}(\prod_{j\in N}s_j(a_j|\theta_j)) u_i(a,\theta)
\end{align}
% 即只知道i的type时
\textit{Ex interim} expected utility:
\begin{align}
	EU_i(s,\theta_i) = \sum_{\theta_{-i}\in \Theta_{-i}} p(\theta_{-i}|\theta_i) EU_i(s, (\theta_i, \theta_{-i}))
\end{align}
% 即谁的type也不知道时
\textit{Ex ante} expected utility:
\begin{align}
	EU_i(s) = \sum_{\theta\in\Theta} p(\theta) EU_i(s,\theta) =\sum_{\theta_i\in\Theta_i} p(\theta_i) EU_i(s,\theta_i)
\end{align}
Best response: $BR_i(s_{-i})=\arg\max_{s_i'} EU_i(s_i',s_{-i})$.\\
Bayes-Nash equilibrium $s$: $\forall i, s_i\in BR_i(s_{-i})$.\\
% 即对任意type都是BR
\textit{Ex post} equilibrium $s$: $\forall\theta,i$.\\

% 即payoff是对group的
Coalitional game with transferable utility: $(N,v)$, $v:2^N\mapsto \mathbb{R}$.\\
% 即多个disjoint group合作收益不会比各自group的收益和少
Superadditive game, if $\forall S\cap T=\emptyset$ implies $v(S\cup T)\geq v(S)+v(T)$.\\
% 即上述为=
Additive game, $=$.\\
% 即两个disjoint总共为N的收益是constant; 显然additive game => constant-sum game
Constant-sum game, $v(S)+v(N\setminus S)=v(N)$.\\
% 比Superadd-更强
Convex game, $v(S\cup T)\geq v(S)+v(T)-v(S\cap T)$.\\
% Figure 8.1关系图
Simple game, $v(S)\in \{0,1\}$.
% 即单值和 leq 全合作收益
Feasible payoff\\
% =
Pre-imputation\\
% P里面 单值 geq 单收益
Imputation\\
% 即对于所有不含i,j的S   i,j分别参与S的 group收益是相同的
interchangeable\\
% 即对于interchangeable的, i,j个人回报应该是相同的,(显然为了公平性)
Axiom, Symmetry.\\
% 即i参与任何S的额外收益都是v({i}), 所以i个人回报应该就是这个值,(显然,没有给group更多收益)
Axiom, Dummy player\\
% 即回报 收益函数可加性
Axiom, Additivity\\
%
Thm, given a coalitional game, $\exists$ unique pre-imputation satisfy above three axioms.\\
% 即平均边际贡献;这个回报'fair'但不一定stable,采取更小group合作可能回报更高
Shapley value.\\
% 即任意S,单值和要不比v(S)少;即core没有动机采取其它合作
Core\\
Thm, constant-sum, not additive game has empty core.\\
% veto 即v(N\{i})=0即一票否决
Thm, simple game, core empty iff no veto. if vetos, core consist of vectors nonvetos' payoff zero.\\
Thm, convex game has nonempty core and Shapley value in core.
\end{document}