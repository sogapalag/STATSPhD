%%%%%%%%%%%%%%%%%%%%%%%%%%%%%%%%%%%%%%%%%
% Short Sectioned Assignment
% LaTeX Template
% Version 1.0 (5/5/12)
%
% This template has been downloaded from:
% http://www.LaTeXTemplates.com
%
% Original author:
% Frits Wenneker (http://www.howtotex.com)
%
% License:
% CC BY-NC-SA 3.0 (http://creativecommons.org/licenses/by-nc-sa/3.0/)
%
%%%%%%%%%%%%%%%%%%%%%%%%%%%%%%%%%%%%%%%%%

%----------------------------------------------------------------------------------------
%	PACKAGES AND OTHER DOCUMENT CONFIGURATIONS
%----------------------------------------------------------------------------------------

\documentclass[paper=a4, fontsize=11pt]{scrartcl} % A4 paper and 11pt font size

\usepackage[T1]{fontenc} % Use 8-bit encoding that has 256 glyphs
\usepackage{fourier} % Use the Adobe Utopia font for the document - comment this line to return to the LaTeX default
\usepackage[english]{babel} % English language/hyphenation
\usepackage{amsmath,amsfonts,amsthm,bm} % Math packages
\usepackage{amssymb}

\usepackage{sectsty} % Allows customizing section commands
\allsectionsfont{\centering \normalfont\scshape} % Make all sections centered, the default font and small caps
\usepackage{url}
\usepackage{fancyhdr} % Custom headers and footers
\pagestyle{fancyplain} % Makes all pages in the document conform to the custom headers and footers
\fancyhead{} % No page header - if you want one, create it in the same way as the footers below
\fancyfoot[L]{} % Empty left footer
\fancyfoot[C]{} % Empty center footer
\fancyfoot[R]{\thepage} % Page numbering for right footer
\renewcommand{\headrulewidth}{0pt} % Remove header underlines
\renewcommand{\footrulewidth}{0pt} % Remove footer underlines
\setlength{\headheight}{13.6pt} % Customize the height of the header

\numberwithin{equation}{section} % Number equations within sections (i.e. 1.1, 1.2, 2.1, 2.2 instead of 1, 2, 3, 4)
\numberwithin{figure}{section} % Number figures within sections (i.e. 1.1, 1.2, 2.1, 2.2 instead of 1, 2, 3, 4)
\numberwithin{table}{section} % Number tables within sections (i.e. 1.1, 1.2, 2.1, 2.2 instead of 1, 2, 3, 4)

\setlength\parindent{0pt} % Removes all indentation from paragraphs - comment this line for an assignment with lots of text
\def \cov {\text{Cov}}
\def \var {\text{Var}}
\def \arg {\text{arg}}
\def \logit {\text{logit}}

\title{exercises of MCMT}
% Markov Chains and Mixing Times. 2nd ed. David A. Levin, Yuval Peres, Elizabeth L. Wilmer
\author{sogapalag}

\date{\normalsize\today}

\begin{document}

\maketitle
\begin{itemize}
	\item[1.1] $2,x=y$; least length path, $x\neq y$.
	\item[1.2] connected $\Rightarrow$ llp$<\infty$ $\Rightarrow$ irreducible; inconnected $\Rightarrow$ $x\nrightarrow y$.\qed
	\item[1.6] fix $x$, define
	\begin{align}
		C_k := \{y:\tau_y^+ \equiv k \mod b\}
	\end{align}
	obvious disjoint, and note $x\in C_0$ by definition period $b=\tau_x^+/c,c\in \mathbb{Z}^+$, and irreducible i.e. $\exists$ at least fisrt through another $b-1$ distinct states to visit $x$. Hence it is a well-defined partition.\\
	for $P(y,z)>0$, suppose $y\in C_i$, note $x\rightarrow y\rightarrow z\rightarrow x$ and $x\rightarrow z \rightarrow x$ both divided by $b$ implies $z\in C_{i+1}$.\qed
	\item[1.7] easy to verify follow dbe.
	\item[1.8] recall Chapman-Kolmogorov's thm
	\begin{align}
		\pi(x)P^2(x,y) &= \pi(x)\sum_z P(x,z)P(z,y)\\
			&= \pi(x) \sum_z \frac{\pi(z)P(z,x)}{\pi(x)} \frac{\pi(y)P(y,z)}{\pi(z)}\\
			&= \pi(y) \sum_zP(y,z)P(z,x)\\
			&= \pi(y) P^2(y,x)
	\end{align}\qed
	\item[1.9] denote $I_{t,y}$ event $X_t=y$, thus
	\begin{align}
		\tilde{\pi}(y) &= E_z(V_y)\\
			&= \sum_{k=1}^\infty E_z(V_y|\tau_z^+=k)P(\tau_z^+=k)\\
			&= \sum_{k=1}^\infty \sum_{t<k} E_z(I_{t,y}|\tau_z^+=k)P_z(\tau_z^+=k)\\
			&= \sum_{t=0}^\infty \sum_{k>t} P_z(X_t=y|\tau_z^+=k)P_z(\tau_z^+=k)\\
			&= \sum_{t=0}^\infty P_z(X_t=y, \tau_z^+>t)
	\end{align}\qed
	\item[1.10] assume maximum not in $B$, i.e. $\forall y\in B$, that $h(y)<M=h(x_0)$, where $x_0\in \mathcal{X}\setminus B$. harmonic implies any $z$  one-step accessible by $x$, there must be $h(z)=M$, and this relation can repeatly apply, since irreducible, thus $\exists y\in B,h(y)=M$.\qed 
	\item[1.12]
	\begin{itemize}
		\item[(a)] trivial.
		\item[(b)] $E_x[\tau_A] = \sum_y (1+E_y[\tau_A])P(x,y)= 1 + \sum_y P(x,y)f(y)$.
		\item[(c)] by irreducible, note $\exists x\in B$ s.t. $\sum_{y\in B}P(x,y)<1$, thus $h=f_1-f_2$
		\begin{align}
			h(x) &= \sum_{y\in B}P(x,y)h(y)\\
				&= \sum_{y\in B}P(x,y)\sum_{y_1\in B}P(y,y_1)\dots\sum_{z\in B}P(y_n,z)h(z)\\
				&\rightarrow 0
		\end{align}
	\end{itemize}
	\item[1.14] for each e.c.class, $\pi_1,\pi_2,...$, thus set S. is $\sum_i P(i)\pi_i$, where $\sum_i P(i)=1, P(i)\geq 0$, i.e. convex hull.\qed
	\item[2.2] recall 2.1, $f_k/p$.
	\item[2.3] recall Birth-and-Death Chains, (2.14) let $(p,r,q)=(1/2,0,1/2)$. thus $E_n[\tau_0] \rightarrow n(n+1)$. Rigorous proof as below, note Unif is stationary, by (2.13) $E_{l-1}[\tau_l]=2l$, thus $E_0[\tau_n]=\sum_{l=1}^n 2l=n(n+1)$.
	\item[2.4] trivial, left apply right, right apply left.
	\item[2.5] note ${n \choose k} = \frac{n}{k} {n-1 \choose k-1}$, combine to $P$, satisfy dbe. implies S.\qed
	\item[2.6] $(\mathbb{Z}_3,+)$, with $\mu(1)+\mu(0)=1,\mu(1)>0$, which form a cycle-flow obvious not dbe. i.e. not reversible.
	\item[2.7] note Unif is S., dbe $P(h,g)=P(g,h)$ implies $\mu(gh^{-1})=\mu(hg^{-1})$ implies symmetry.\qed
	\item[2.8] for each pair $(x,y)$ transitive implies $\pi(\varphi(z))=\pi(z)$, thus implies 
	\begin{align}
		\widehat{P}(\varphi(z),\varphi(w)) &= \frac{\pi(\varphi(z))P(\varphi(z),\varphi(w))}{\pi(\varphi(w))}\\
			&= \frac{\pi(z)P(z,w)}{\pi(w)}\\
			&= \widehat{P}(z,w), \forall z,w
	\end{align}
	hence transitive time reversal.\qed
	\item[2.9] obvious $\mapsto \mathbb{Z}_n$ is the projection.
	\item[2.10] recall Amir-Ex5.1.14. or simply think when hit c, there at least $1/2$ still $\geq c$ by reflection principle.\qed
	\item[2.11] by induction.
	\item[3.1] by dbe.
	\begin{align}
		\pi(x)\Psi(x,y)a(x,y) = \pi(y)\Psi(y,x)a(y,x)
	\end{align}
	and $a\leq 1$ implies best $a(x,y)=1\wedge \frac{\pi(y)\Psi(y,x)}{\pi(x)\Psi(x,y)} ,y\neq x$, i.e. $P$.\qed
	\item[3.2] note only $y\in \mathcal{X}(x,v)$ transition, which def implies $\mathcal{X}(y,v)=\mathcal{X}(x,v)$, and recall transition prob defined in (3.7) obvious $\pi(x)\pi(y)=\pi(y)\pi(x)$ satisfy dbe. i.e. reversible.\qed
\end{itemize}
\end{document}