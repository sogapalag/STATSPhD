%%%%%%%%%%%%%%%%%%%%%%%%%%%%%%%%%%%%%%%%%
% Short Sectioned Assignment
% LaTeX Template
% Version 1.0 (5/5/12)
%
% This template has been downloaded from:
% http://www.LaTeXTemplates.com
%
% Original author:
% Frits Wenneker (http://www.howtotex.com)
%
% License:
% CC BY-NC-SA 3.0 (http://creativecommons.org/licenses/by-nc-sa/3.0/)
%
%%%%%%%%%%%%%%%%%%%%%%%%%%%%%%%%%%%%%%%%%

%----------------------------------------------------------------------------------------
%	PACKAGES AND OTHER DOCUMENT CONFIGURATIONS
%----------------------------------------------------------------------------------------

\documentclass[paper=a4, fontsize=11pt]{scrartcl} % A4 paper and 11pt font size

\usepackage[T1]{fontenc} % Use 8-bit encoding that has 256 glyphs
\usepackage{fourier} % Use the Adobe Utopia font for the document - comment this line to return to the LaTeX default
\usepackage[english]{babel} % English language/hyphenation
\usepackage{amsmath,amsfonts,amsthm,bm} % Math packages

\usepackage{sectsty} % Allows customizing section commands
\allsectionsfont{\centering \normalfont\scshape} % Make all sections centered, the default font and small caps
\usepackage{url}
\usepackage{fancyhdr} % Custom headers and footers
\pagestyle{fancyplain} % Makes all pages in the document conform to the custom headers and footers
\fancyhead{} % No page header - if you want one, create it in the same way as the footers below
\fancyfoot[L]{} % Empty left footer
\fancyfoot[C]{} % Empty center footer
\fancyfoot[R]{\thepage} % Page numbering for right footer
\renewcommand{\headrulewidth}{0pt} % Remove header underlines
\renewcommand{\footrulewidth}{0pt} % Remove footer underlines
\setlength{\headheight}{13.6pt} % Customize the height of the header

\numberwithin{equation}{section} % Number equations within sections (i.e. 1.1, 1.2, 2.1, 2.2 instead of 1, 2, 3, 4)
\numberwithin{figure}{section} % Number figures within sections (i.e. 1.1, 1.2, 2.1, 2.2 instead of 1, 2, 3, 4)
\numberwithin{table}{section} % Number tables within sections (i.e. 1.1, 1.2, 2.1, 2.2 instead of 1, 2, 3, 4)

\setlength\parindent{0pt} % Removes all indentation from paragraphs - comment this line for an assignment with lots of text
\def \cov {\text{Cov}}
\def \var {\text{Var}}
\def \arg {\text{arg}}
\def \logit {\text{logit}}

\title{notes of STATS 318: Modern Markov Chains}
% Markov Chains and Mixing Times. 2nd ed. David A. Levin, Yuval Peres, Elizabeth L. Wilmer
% Finite Markov Chains and Algorithmic Applications. 2002. Olle Häggström.
\author{sogapalag}

\date{\normalsize\today}

\begin{document}

\maketitle
\section{Introduction to Finite Markov Chains}
% depend on current. so matrix P sufficient to describe
Markov property\\
state space $\mathcal{X}$, transition$\mathcal{X}\rightarrow \mathcal{X}$ matrix $P:\mathcal{X}\times \mathcal{X}\rightarrow \mathbb{R}^+$. $\sum_y P(x,y)=1,\forall x\in\mathcal{X}$.\\
% 即状态转移 右乘P
next distribution $\mu P$.\\
% 即Pf的含义 为下步的f期望; 下标含义为init
$Pf(x)=E_x[f(X_1)]$.\\
% 即划分U[0,1],落在相应区域对应状态
prop1.5, finite $P$ has random mapping representation.\\
% 即所有状态可互相抵达
def, irreducible, any $x,y$ $\exists t<\infty$ s.t. $P^t(x,y)>0$.\\
% 注意gcd, 不代表t=gcd一定抵达
def, period, gcd$T(x)$, where $T(x):=\{t\geq 1:P^t(x,x)>0\} $.\\
% 因为x<->y, 所以x自抵经过y,加上任意T(y)仍可,故divide T(y)
lemma, irreducible $\Rightarrow$ gcd$T(x)=c,\forall x$.\\
def, aperiodic, $c=1$.\\
% Schur, coin problem(Frobenius number, NP-hard)
% 存在证明,对于coprime a,b, a(0~b-1)=Z_b, 然后平移相应kb 
% 故由T(x)的加法封闭性可知, 大于某值后x->x一直,由于irreducible, x->任意y一直=>任意x->任意y一直
prop, aperiodic, irreducible $\Rightarrow$ $\exists r_0$ s.t. $P^r(x,y)>0$, $\forall x,y\forall r\geq r_0$.\\
def, stationary(S.) distribution, $\pi=\pi P$.\\
% easytocheck 正好每条边传1份抵消
for RW on graph, $\pi(y)=\frac{\text{deg}(y)}{2|E|}$.\\
def, hitting time $\tau_x:=\min\{t\geq 0:X_t=x\}$.\\
def, first return time, when $X_0=x$, $\tau_x^+:=\min\{t\geq 1:X_t=x\}$.\\
lemma, irreducible $\Rightarrow$ $E_x[\tau_y^+]<\infty,\forall x,y$.\\
% prop1.4, averge停留即为S.
def, harmonic(on $D$), $Ph=h$.\\
lemma, irreducible $\Rightarrow$ harmonic(on $\mathcal{X}$) $h$ is constant.\\
coll, irreducible $\Rightarrow$ unique $\pi$.\\
prop, irreducible $\Rightarrow$ $\pi(z)=1/E_z(\tau_z^+)$.\\
% 即任意x-y 流出=流入
def, detailed balance equations(dbe), $\pi(x)P(x,y)=\pi(y)P(y,x),\forall x,y$.\\
prop, dbe $\Rightarrow$ S.\\
% 即当init满足dbe时,(X0,...,Xn)与其reverse同分布; 显然,平衡反演
def, reversible.\\
% 直觉上即,水(气体)一直顺时针流,是S.(恒单位质量)但不是dbe的(截面有流出)
e.g. for RW on $n$-cycle, $\pi(k)=1/n$ is S. but not dbe(reversible) when biased.\\
% 想像时间反演
def, time reversal of irreducible, $\widehat{P}(x,y):=\frac{\pi(y)P(y,x)}{\pi(x)}$.\\
% prop1.23 显然反演S.; 当init为此,即原历程与反演同分布。
prop, $\pi$ is S. for $\widehat{P}$ and $P_{\pi}(\rightarrow)=\widehat{P}_\pi(\leftarrow)$.\\
def, accessible, $\exists 0<r<\infty, P^r(x,y)>0$, write $x\rightarrow y$.\\
def, essential, $x\rightarrow y$ implies $y\leftarrow x$, $\forall y$.\\
% e.g. 三维RW,每个点都是essential,都不是recurrent.
remark. for finite, essential $\Leftrightarrow$ $\rho_{xx}=P_x(\tau_x^+<\infty)=1$(recurrent).\\
def, communicates, $x \leftrightarrow y$ iff $x\rightarrow y,y\leftarrow x$, or $x=y$. denote c.class $[x]$.\\
lemma, essential $x\rightarrow y$ $\Rightarrow$ $y$ essential.\\
lemma, finite chain has at least one e.class.\\
prop, finite, inessential $\pi(y)=0$.\\
% recall 分解disjoint recurrents, transient; recurrent的S.分布组合显然仍为S.
prop, unique S. iff $\exists$ unique e.c.class.\\

\section{Classical (and Useful) Markov Chains}
% Gambler's ruin
% Coupon Collecting
% n-dim cube RW, Ehrenfest urn
% lemma2.5 projection(lumping), constructing chain ([X_t]) 
% Pólya’s urn; lemma2.7, d-color, init,(1,1,..1), then N_t uniform.
% Birth-and-Death Chains, prop, reversible. E_a[tau_b]
% RW on finite group, Unif is S.
% prop, irreducible iff <S>=G; increment dist. 显然,生成即对应一条抵达链
% inc-d. irreducible iff symmetric.
% def, transitive, any pair exist bijection states remain P.
% prop, transitive, finite => Unif is S.
% Reflection Principle; recall Ballot problem on Amir-note, Ex5.5.30-31

\section{Markov Chain Monte Carlo: Metropolis and Glauber Chains}
% 即由对称Q,构造出一个pi的P
% 注:构造出来的是dbe的
(Symmetric base) Metropolis chain.
\begin{align}
	P(x,y) = \begin{cases} \Psi(x,y)[1\wedge \frac{\pi(y)}{\pi(x)}] &\mbox{if $y\neq x$,}\\
		1 - \sum_{z\neq x} \Psi(x,z)[1\wedge \frac{\pi(z)}{\pi(x)}] &\mbox{if $y=x$.}\end{cases}
\end{align}
% 这个构造好处只依赖ratio
(General base(irreducible)) Metropolis chain.
\begin{align}
	P(x,y) = \begin{cases} \Psi(x,y)[1\wedge \frac{\pi(y)\Psi(y,x)}{\pi(x)\Psi(x,y)}] &\mbox{if $y\neq x$,}\\
		1 - \sum_{z\neq x} \Psi(x,z)[1\wedge \frac{\pi(z)\Psi(z,x)}{\pi(x)\Psi(x,z)}] &\mbox{if $y=x$.}\end{cases}
\end{align}
%(Gibbs)
% def, j allowable at v, 即v的邻居没有使用的颜色
% proc, 随机v, 随机重着色v(从allowable set)
% 这样从一个q-coloring 生成了unif的q-color分布的chain
Glauber dynamics for proper q-colorings.\\
% hardcore configuration
% Glauber dynamics for pi, (3.7)即从只同v的分布select
% Hardcore model with fugacity
% The Ising model. A spin system (3.12) create Gibbs-d. mu
\end{document}