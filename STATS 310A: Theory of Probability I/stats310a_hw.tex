%%%%%%%%%%%%%%%%%%%%%%%%%%%%%%%%%%%%%%%%%
% Short Sectioned Assignment
% LaTeX Template
% Version 1.0 (5/5/12)
%
% This template has been downloaded from:
% http://www.LaTeXTemplates.com
%
% Original author:
% Frits Wenneker (http://www.howtotex.com)
%
% License:
% CC BY-NC-SA 3.0 (http://creativecommons.org/licenses/by-nc-sa/3.0/)
%
%%%%%%%%%%%%%%%%%%%%%%%%%%%%%%%%%%%%%%%%%

%----------------------------------------------------------------------------------------
%	PACKAGES AND OTHER DOCUMENT CONFIGURATIONS
%----------------------------------------------------------------------------------------

\documentclass[paper=a4, fontsize=11pt]{scrartcl} % A4 paper and 11pt font size

\usepackage[T1]{fontenc} % Use 8-bit encoding that has 256 glyphs
\usepackage{fourier} % Use the Adobe Utopia font for the document - comment this line to return to the LaTeX default
\usepackage[english]{babel} % English language/hyphenation
\usepackage{amsmath,amsfonts,amsthm} % Math packages

\usepackage{sectsty} % Allows customizing section commands
\allsectionsfont{\centering \normalfont\scshape} % Make all sections centered, the default font and small caps
\usepackage{url}
\usepackage{fancyhdr} % Custom headers and footers
\pagestyle{fancyplain} % Makes all pages in the document conform to the custom headers and footers
\fancyhead{} % No page header - if you want one, create it in the same way as the footers below
\fancyfoot[L]{} % Empty left footer
\fancyfoot[C]{} % Empty center footer
\fancyfoot[R]{\thepage} % Page numbering for right footer
\renewcommand{\headrulewidth}{0pt} % Remove header underlines
\renewcommand{\footrulewidth}{0pt} % Remove footer underlines
\setlength{\headheight}{13.6pt} % Customize the height of the header

\numberwithin{equation}{section} % Number equations within sections (i.e. 1.1, 1.2, 2.1, 2.2 instead of 1, 2, 3, 4)
\numberwithin{figure}{section} % Number figures within sections (i.e. 1.1, 1.2, 2.1, 2.2 instead of 1, 2, 3, 4)
\numberwithin{table}{section} % Number tables within sections (i.e. 1.1, 1.2, 2.1, 2.2 instead of 1, 2, 3, 4)

\setlength\parindent{0pt} % Removes all indentation from paragraphs - comment this line for an assignment with lots of text
\def \cov {\text{Cov}}
\def \var {\text{Var}}

\title{Homeworks of STATS 310A}
% https://web.stanford.edu/~montanar/TEACHING/Stat310A/hw.html
\author{sogapalag}

\date{\normalsize\today}

\begin{document}

\maketitle
\begin{itemize}
	\item[Ex1.1.4] 
	\begin{itemize}
		\item[(a)] note $B = A \cup (B\setminus A)$, which disjoint and $P(B\setminus A)\geq 0$, thus by additive $P(A)=P(B)-P(B\setminus A)\leq P(B)$.\qed
		\item[(b)] let $B_1= A_1$, $B_i = A_i\setminus \bigcup_{k=1}^{i-1} A_k$, thus $B_i\subseteq A_i$ and disjoint
		\begin{align}
			P(\bigcup_i A_i)  = P(\bigcup_i B_i) =\sum_i P(B_i) \leq \sum_i P(A_i)
		\end{align}
		combine (a) proved.\qed
		\item[(c)] let $B_1= A_1$, $B_n = A_n\setminus A_{n-1}$, which are disjoint, thus
		\begin{align}
			P(A) = P(\bigcup_i A_i) = P(\bigcup_i B_i) = \lim_{n\rightarrow\infty} \sum_{i=1}^n P(B_i) = \lim_{n\rightarrow \infty} P(A_n)
		\end{align}\qed
		\item[(d)] consider $A_i^c$ and recall DeMorgan's law, combine (c), that $P(A_i^c)\uparrow P(A^c)$, thus
		\begin{align}
			P(A) = 1-P(A^c) = 1 - \lim_{n\rightarrow \infty} P(A_i^c) = \lim_{n\rightarrow \infty} (1 - P(A_i^c)) = \lim_{n\rightarrow \infty}  P(A_i)
		\end{align}\qed
	\end{itemize}
	\item[Ex1.1.13]
	\begin{itemize}
		\item[(a)] let $F = \bigcap_\alpha F_\alpha$, since $\emptyset\in F_{\alpha}, \forall \alpha$, that $\emptyset\in F$; if $A\in F$, implies $\forall \alpha,  A\in F_\alpha$, thus $A^c\in F_\alpha, \forall \alpha$, i.e. $A^c\in F$; for some countable $i=1,2,...$, $A_i\in F$, let $A=\bigcup_i A_i$, since $\forall i,\alpha, A_i\in F_\alpha$, implies $\forall \alpha, A\in F_\alpha$, i.e. $A\in F$.\qed
		\item[(b)] $\emptyset\in \mathcal{H}^H$ is trivial; if $A\in \mathcal{H}^H$, implies $A\cap H\in \mathcal{H}$, note $H\in \mathcal{H}$, thus
		\begin{align}
			A^c \cap H &= (A^c \cap H) \cup (H^c \cap H) \\
				&= (A^c \cup H^c) \cap  H \\
				&= (A\cap H)^c \cap H
		\end{align}
		by closed under intersection, i.e. $A^c\cap H\in \mathcal{H}$, that $A^c\in \mathcal{H}^H$; similarily
		\begin{align}
			(\bigcup_i A_i)\cap H &= ((\bigcup_i A_i)\cup H^c) \cap H\\
				&= (\bigcup_i(A_i\cup H^c)) \cap H\\
				&= \bigcup_i ((A_i\cap H)^c \cap H)
		\end{align}
		by closed under union, that $\bigcup_i A_i \in \mathcal{H}^H$.\qed
		\item[(c)] let $H_1\subseteq H_2$, then
		\begin{align}
			A\cap H_1 &= A\cap (H_1 \cap H_2)\\
				&= (A\cap H_2) \cap H_1
		\end{align}
		i.e. $A\in \mathcal{H}^{H_2}$ will imply $A\in \mathcal{H}^{H_1}$, so non-increasing. Thus $\mathcal{H}^{H\cup H'} \subseteq \mathcal{H}^H \cap \mathcal{H}^{H'}$, and if $A\in \mathcal{H}^H \cap \mathcal{H}^{H'}$, that
		\begin{align}
			A \cap (H\cup H') = (A\cap H) \cup (A\cap H')
		\end{align}
		implies $A\in \mathcal{H}^{H\cup H'}$, proved equality.\qed
	\end{itemize}
	\item[Ex1.1.21] denote second $B_{\times d}$, third $F$. $F\subset B_{\times d}$ and $F\subset B_{\mathbb{R}^d}$ is trivial. to show $F\supset B_{\mathbb{R}^d}$, note every open $G$ on $\mathbb{R}^d$, there $\exists x\in G$, and $\exists$ open $(.)\times\dots \times(.), \in Q$, which contain $x$, thus exists a countable union to $G$, since $d$ is finite still countable, thus $F= B_{\mathbb{R}^d}$. to show $F\supset B_{\times d}$, recall Ex1.1.17, and def of Ex1.1.20, that
	\begin{align}
		\sigma(\{A_1\times\dots \times A_d: A_i\in B\}) = \sigma(\{ (.)\times\dots \times A_d: (.)\in R, A_i\in B\})
	\end{align}
	continue $d$ steps, that $F= B_{\times d}$.\qed
	\item[Ex1.1.22] denote $C$ the class of this property, if $A\in C$, i.e. $A\in \sigma(\{A_{\alpha_1},A_{\alpha_2},...\})$, which implies $\emptyset, A^c, \bigcup_i A_i$ in that algebra too, thus $\emptyset, A^c, \bigcup_i A_i$ in $C$, that $C$ is a $\sigma$-algebra. Note $\forall \alpha, A_\alpha\in C$, thus $F\subset C$, so $B\in C$.\qed
	% 直觉上,一个无限齿的齿轮,合势将大于1
	\item[Ex1.1.24] if $P(A)=1$, then $\mathbb{R}$ is the open set. if $P(A)<1$, there are infinite disjoint open on $A^c$, so exists $N\epsilon>1$, which implies there is a open $H$ with $P(H)<\epsilon$, thus $H\cup A$ is the open set.\qed
	\item[Ex1.1.27] for each $A\in \mathcal{B}_{(0,1]}$, there exitst contains a $c_i\in \mathbb{Q}$ in each sub interval of $A$, thus countable, which split into two nonempty unions $\bigcup_i(a_i,c_i]$ and $\bigcup_i (c_i,b_i]$, by Lebesgue measure, $0<U<U(A)$, so it's a non-atomic probability space. Since $U$ is restriction of $\lambda$, surely non-atomic measure.\qed
	\item[Ex1.1.33] $\mathcal{A}=\{\{1,3\},\{2,3\}\}$, which is sufficient to generate $2^\Omega$, then define $u(w_i)=1/4$, $v(\{1\})=v(\{2\})=1/3$, $v(\{3\})=v(\{4\})=1/6$.
	% Banach-Tarski paradox in one dimension
	\item[A1] $\Rightarrow$, let $f|_{A_i} = A_i - t_i$, clearly it's an equidecomposition $A\rightarrow B$, note $\forall i$, $f|_{A_i}$ is a bijective, and note that $\{A_i\}$ and $\{B_i\}$ are partition, so $A_i$ won't inject to $B_{j\neq i}$, i.e. $f$ is bijective.\\
	$\Leftarrow$, sufficient to show $\{B_i\}$ is countable partition. if $\exists y \in B_i \cap B_j,i\neq j$, that $f(x_i)=y=f(x_j)$ violate to injective, so $\{B_i\}$ is partition; since $\{A_i\}$ is countable, by bijective
	\begin{align}
		B = f(A) = f(\bigcup_i A_i) = \bigcup_i f(A_i) = \bigcup_i B_i
	\end{align}\qed
	 \item[A2] by definition of hint, note $A$ with partition $\{A_i\}$, $B$ with partition $\{B_i\}$, note $(A^{(0)}\subset A^{(*)})\subset A$, that 
	 \begin{align}
	 	A &= A^{(*)} \cup (A\setminus A^{(*)})\\
	 		&= (A\cap A^{(*)}) \cup ((A\setminus A^{(0)}) \cap (A\setminus A^{(*)}))\\
	 		&= (\bigcup_i A_i \cap A^{(*)}) \cup (g(B)\cap (A\setminus A^{(*)}))\\
	 		&= (\bigcup_i(A_i\cap A^{(*)})) \cup (\bigcup_i(g(B_i) \cap (A\setminus A^{(*)})))
	 \end{align}
	 which is countable, clearly disjoint. that $h$ is an equidecomposition. and $h^{-1}$ can be defined as
	 \begin{align}
	 	h^{-1} = \begin{cases} f^{-1}(y), &\mbox{if $ g(y)\in A^{(*)}$}\\ g(y), &\mbox{otherwise} \end{cases}
	 \end{align}
	 clearly it's well defined, since $h$ is well defined, and we see that if $x\in A^{(*)}$, $g(f(x)) = (g\circ f)(x) \in A^{(*)}$, which $h^{-1}$ should apply $f^{-1}$ corresponding, indeed is inverse, thus $h$ is bijective.\qed
	 \item[B1] define a collections of equivalent class, with $x-y\in \mathbb{Q}$, say $x,y\in E$, by mod$1/2$, clearly $E\cap [0,1/2]\neq \emptyset$, by choice axiom, $x_E\in E\cap [0,1/2]$ (by mod$1/2$ operation), hence
	 \begin{align}
	 	\mathbb{R} = \bigcup_{E} E = \bigcup_{x_E\in E\cap [0,1/2]} E = \bigcup_{x\in C}(x+\mathbb{Q})
	 \end{align}\qed
	 \item[B2] note $\mathbb{Q}$ is countable, thus
	 \begin{align}
	 	&\mathbb{Q} = \bigcup_i \{p_i\} \\
	 	&\mathbb{Q}\cap [0,1/2] = \bigcup_i \{q_i\}
	 \end{align}
	 let $t_i = p_i-q_i$, proved the equidecomposable.\qed
	 \item[B3] by B1, note there are countable partitions:
	 \begin{align}
	 	&\mathbb{R}  =  \bigcup_i (p_i + C)\\
	 	&A = \bigcup_i(q_i+C)
	 \end{align}
	 surjective is obvious, if not disjoint, there $c_1,c_2\in E$ conflict to partition of B1, so it's indeed partition. and by B2, still $t_i=p_i-q_i$, proved the equidecomposable.\qed
	 \item[B4] clearly $[0,1]$ is equidecomposable to $[0,1]$ with letting $t=0$. since $[0,1]\subset \mathbb{R}$ and $A\subset [0,1]$, recall B3 and A2, that $[0,1]$ is equidecomposable to $\mathbb{R}$.\qed
	 \item[Ex1.1.45] Recall Benford's law, $B=\{x\in \mathbb{N}: x \text{ start with 1}\}$, is not CES, we slightly change it, let
	 \begin{align}
	 	A = \{ x\in \mathbb{N}: x\text{ is even and start with 1}\}
	 \end{align}
	 one can verify $A$ is not CES, since
	 \begin{align}
	 	\limsup_{n\rightarrow\infty} \frac{A(n)}{n} = \frac{5}{18} \\
	 	\liminf_{n\rightarrow\infty} \frac{A(n)}{n} = \frac{1}{18}
	 \end{align}
	 then define $V_1$ and $V_2$ split $\mathbb{N}\setminus B$ to odd and even, i.e.
	 \begin{align}
	 	V_1 = A \cup \{x\in \mathbb{N}\setminus B: x\text{ is odd} \}\\
	 	V_2 = A \cup \{x\in \mathbb{N}\setminus B: x\text{ is even} \}
	 \end{align}
	 obviously they are CES, since
	 \begin{align}
	 	\lim \frac{V_1(n)}{n} = \lim \frac{V_2(n)}{n} =\frac{1}{2}
	 \end{align}
	 which shows CES is not algebra.\qed
	% NOT rigorous, which is indeed random set	 
	 %let 
	 %\begin{align}
	 %	&V_1(n) = \{1,2,...,\lceil n \rceil/2\}\\
	 %	&V_2(n) = \begin{cases} \{1,2,...,\lceil n \rceil/2\}, &\mbox{if $n$ even}\\
	 %		\{1,3,5,...,n\}, &\mbox{otherwise}\end{cases}
	 %\end{align}
	 %thus $\gamma(V_1)$ and $\gamma(V_2)$ with limits $1/2$, but $\gamma(V_1\cap V_2)$ without limits($1/2,1/4$) hence CES is not algebra.\qed
	 \item[Ex1.1.47] for finite $A_i$, $\sup i<\infty$, so hold algebra is trivial; but $\sigma$-algebra require countable infinite union closing, which $\sup i$ may $\infty$. e.g. consider Borel class $\mathcal{B}_{(0,1]}\subset\mathcal{B}_{(0,2]}\subset...\subset\mathcal{B}_{(0,n]}\subset...$, let $A_i=(i-1, i-0.5)$, then $\bigcup_{i=1}^\infty A_i\notin \mathcal{B}_{(0,n]},\forall n$ thus $\bigcup_{i=1}^\infty A_i\notin \bigcup_{n=1}^\infty \mathcal{B}_{(0,n]}$.
	 \item[Ex1.2.11] To show $\mathcal{S}=\sigma(\mathcal{A})$ implies $X^{-1}(\mathcal{S})=\sigma(X^{-1}(\mathcal{A}))$. Though Thm1.2.10 call $X^{-1}(\mathcal{S})$ $\sigma$-algebra, one first verify, $\emptyset,D^c,\bigcup_i D_i$ is indeed $\in X^{-1}(\mathcal{S})$ since $\mathcal{S}$ is $\sigma$-algebra. Then note $\mathcal{S}\supset \mathcal{A}$, $X^{-1}(\mathcal{S}) \supset X^{-1}(\mathcal{A})$, thus by definition of generating $\sigma$-algebra(smallest), $X^{-1}(\mathcal{S}) \supset \sigma(X^{-1}(\mathcal{A}))$. note $\mathcal{S}=\sigma(\mathcal{A})$, and binary operations under preimage $X^{-1}$ remains, thus $X^{-1}(\mathcal{S}) \subset \sigma(X^{-1}(\mathcal{A}))$.\qed
	 \item[Ex1.2.14]
	 \begin{itemize}
	 	\item[(a)] note Borel R can write as $\mathcal{B}_{\mathbb{R}}= \sigma(\{(-\infty,\alpha],\alpha\in \mathbb{R}\})$, hence $X^{-1}(A_\alpha) = \{\omega:X(\omega)\leq \alpha\}$, then by Ex1.2.11 proved.\qed
	 	\item[(b)] recall Ex1.1.21, one can show $\mathcal{B}_{\mathbb{R}^n}$ can be generated by collection of sets $\{(-\infty,\alpha_1]\times\dots\times(-\infty,\alpha_n]\},\alpha_k\in \mathbb{R}$. and recall Ex1.2.3, by Ex1.2.11 proved.\qed
	 	\item[(c)] each $X_k$ is measurable with RHS, thus RHS$\supset$LHS, and each $\{.\}$ is element of LHS (recall (b)), thus RHS$\subset$LHS.\qed
	 	\item[(d)] each $X_k$ is measurable with RHS, thus RHS$\supset$LHS, and each $\sigma(k)$ is cantained in LHS by (b), thus RHS$\subset$LHS.\qed
	 \end{itemize}
	 \item[Ex1.2.15]
	 \begin{itemize}
	 	\item[(a)] let $\mathcal{C}=\{B\in\mathcal{F}: \inf_{A\in\mathcal{A}} P(A\bigtriangleup B)=0\}$, note if $B\in\mathcal{A}$, choose $A=B$ that $P(A\bigtriangleup B)=0$, hence $\mathcal{A}\subset \mathcal{C}$. obvious $\emptyset\in \mathcal{C}$, and notice that
	 	\begin{align}
	 		A^c\bigtriangleup C^c = (A^c\cap C) \cup (A\cap C^c) = A \bigtriangleup C
	 	\end{align}
	 	implies $C^c\in \mathcal{C}$. let $D= \bigcup_i C_i$, $D_n= \bigcup_i^n C_i$, note $D_n,D\in\mathcal{F}$, i.e. $P(D_n)\uparrow P(D)$, thus choose $n$ s.t. $P(D\setminus D_n)< \epsilon/2$, then choose $A_i$ s.t. $P(A_i\bigtriangleup C_i)<\epsilon/(2n)$
	 	\begin{align}
	 		P(\bigcup_i^n A_i \bigtriangleup \bigcup_i C_i) &\leq P(\bigcup_i^n A_i\bigtriangleup \bigcup_i^n C_i)) + P(D\setminus D_n)\\
	 		&< P(\bigcup_i^n (A_i\bigtriangleup  C_i)) + \epsilon/2\\
	 		&< \epsilon
	 	\end{align}
	 	since $\epsilon\rightarrow 0$, which shows $D\in \mathcal{C}$, implies $\mathcal{C}$ is $\sigma$-algebra, contains $\mathcal{F}=\sigma(\mathcal{A})$.\qed
	 	\item[(b)] recall SF in prop1.2.6, use the technique cut $n$ into $n2^n$ subintervals, let $B_k = X^{-1}((k2^{-n},(k+1)2^{-n}])\in \mathcal{F}$, by (a), $\exists A_k\in\mathcal{A}$ s.t. $P(A_k\bigtriangleup B_k)<\delta=\epsilon/(n2^n)$, let $Y=\sum_{k=0}^{n2^n-1}k2^{-n} I_{A_k}$, note $n$ is selected as $X$ is bounded, and that $2^{-n}<\epsilon$, then
	 	\begin{align}
	 		P(|X-Y|>\epsilon)&\leq P(\bigcup_{k=0}^{n2^n-1} I_{B_k}\neq I_{A_k}) \\
	 			& \leq \sum_{k=0}^{n2^n-1} P(A_k\bigtriangleup B_k)\\
	 			& <\epsilon
	 	\end{align}\qed
	 \end{itemize}
	 \item[Ex1.2.20]
	 \begin{itemize}
	 	\item[(a)] $x_n\rightarrow x$, then by l.s.c. $g(x)\leq \liminf_n g(x_n)\leq b$, thus $\{x:g(x)\leq b\}$ is closed.\qed
	 	\item[(b)] note Borel can be generated by $(-\infty,b]$, so l.s.c. surely measurable; note Borel can also be generated by $(-\infty,b)$, and for u.s.c. $C=\{x:g(x)<b\}$ is open, since forall $x\in C$, $\limsup_n g(x_n) \leq g(x)<b$, implies open. so measurable too.\qed
	 	\item[(c)] c. is both l.s.c. and u.s.c. by (b) surly measurable.\qed
	 \end{itemize}
% lemma 2.1.6
% 1.4.18 zeta(s)
% Example 2.1.8 (Coupon collector's problem) T/(n logn) ->^{L2} 1; 即n个全收集的渐近期望次数 n logn
% Example 2.1.10 a=r/n,  N_n/n  ->  e^{-a}; 即n^r分配; 空箱比例渐近为e^{-a}.
	\item[Ex2.2.18] M.I., and $\sum P(X_n)= \sum 1/n^2 < \infty$, thus $P(X_n \text{i.o.})=0$, thus $n^{-1}\sum^n X_i \stackrel{a.s.}{\rightarrow} -1$ as $n\rightarrow \infty$.\qed
	% 注意右边不是 ->^{p}, P(>e)<e; 右边是P(>e i.o.)=0 
	\item[Ex2.2.19] $X_n \stackrel{a.s.}{\rightarrow} 0$ is merely latter since $P(\{\omega:X(\omega)\neq 0\})=0$; latter let $\epsilon\rightarrow 0$, implies former.\qed
% Example 2.2.21(HEAD RUNS), trick is P_n<> 1/n, 这样就能用B-C-I,II;于是 L_n/log_2 n ->{a.s.}1 ;double, ..-2,-1,0,1,2,; 以 {1,2,...,n}结尾的run的长max, 渐近
	\item[Ex2.2.22] $p\geq 1/2$, by hint, denote $B_i$ the event there is a $\geq k$ runs start at $2^k +ik$, note $B_i$ are mutually independent, thus
	\begin{align}
		P(A_k) &\geq P(\bigcup_{i=0}^{[2^k/k]} B_i)\\
			&= 1 - \prod_{i=0}^{[2^k/k]} (1-p^k)\\
			&\geq 1 - \prod_{i=0}^{[2^k/k]} e^{-p^k}, \text{by $1-x\leq e^{-x}$}\\
			&= 1 - \exp(- \frac{(2p)^k}{k} - cp^k)\\
			&\geq 1 - \exp(-\frac{1}{k})\\
			&\geq \frac{1}{k} - \frac{1}{2k^2}, \text{by $1-x+\frac{x^2}{2}\geq e^{-x}$}
	\end{align}
	since $\sum 1/k^2$ converge, $\sum 1/k$ diverge, thus $\sum P(A_k)=\infty$, note $A_k$ are mutually independent, implies $P(A_k\text{i.o.})=1$.\\
	% think too much, wanna tight bound!!!....
	$p<1/2$, simply denote $C_j$ there is valid run start at $2^k+j$, then  
	\begin{align}
		P(A_k) &= P(\bigcup_{j=0}^{2^k-k} C_j) \\
			&\leq \sum_{j=0}^{2^k-k} P(C_j) \\
			&\leq (2p)^k
	\end{align}
	since $\sum c^k$ with $c<1$ converges, hence $\sum P(A_k)<\infty$, implies $P(A_k\text{i.o.})=0$.\qed
\end{itemize}

\end{document}