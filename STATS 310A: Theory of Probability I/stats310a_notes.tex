%%%%%%%%%%%%%%%%%%%%%%%%%%%%%%%%%%%%%%%%%
% Short Sectioned Assignment
% LaTeX Template
% Version 1.0 (5/5/12)
%
% This template has been downloaded from:
% http://www.LaTeXTemplates.com
%
% Original author:
% Frits Wenneker (http://www.howtotex.com)
%
% License:
% CC BY-NC-SA 3.0 (http://creativecommons.org/licenses/by-nc-sa/3.0/)
%
%%%%%%%%%%%%%%%%%%%%%%%%%%%%%%%%%%%%%%%%%

%----------------------------------------------------------------------------------------
%	PACKAGES AND OTHER DOCUMENT CONFIGURATIONS
%----------------------------------------------------------------------------------------

\documentclass[paper=a4, fontsize=11pt]{scrartcl} % A4 paper and 11pt font size

\usepackage[T1]{fontenc} % Use 8-bit encoding that has 256 glyphs
\usepackage{fourier} % Use the Adobe Utopia font for the document - comment this line to return to the LaTeX default
\usepackage[english]{babel} % English language/hyphenation
\usepackage{amsmath,amsfonts,amsthm} % Math packages

\usepackage{sectsty} % Allows customizing section commands
\allsectionsfont{\centering \normalfont\scshape} % Make all sections centered, the default font and small caps
\usepackage{url}
\usepackage{fancyhdr} % Custom headers and footers
\pagestyle{fancyplain} % Makes all pages in the document conform to the custom headers and footers
\fancyhead{} % No page header - if you want one, create it in the same way as the footers below
\fancyfoot[L]{} % Empty left footer
\fancyfoot[C]{} % Empty center footer
\fancyfoot[R]{\thepage} % Page numbering for right footer
\renewcommand{\headrulewidth}{0pt} % Remove header underlines
\renewcommand{\footrulewidth}{0pt} % Remove footer underlines
\setlength{\headheight}{13.6pt} % Customize the height of the header

\numberwithin{equation}{section} % Number equations within sections (i.e. 1.1, 1.2, 2.1, 2.2 instead of 1, 2, 3, 4)
\numberwithin{figure}{section} % Number figures within sections (i.e. 1.1, 1.2, 2.1, 2.2 instead of 1, 2, 3, 4)
\numberwithin{table}{section} % Number tables within sections (i.e. 1.1, 1.2, 2.1, 2.2 instead of 1, 2, 3, 4)

\setlength\parindent{0pt} % Removes all indentation from paragraphs - comment this line for an assignment with lots of text
\def \cov {\text{Cov}}
\def \var {\text{Var}}

\title{STATS 310A: Theory of Probability I}
% notes by Amir Dembo; first 3 chapters.
% Probability and Measure, 3rd. billingsley
% probability with martingales, williams
\author{sogapalag}

\date{\normalsize\today}

\begin{document}

\maketitle

\section{Probability, measure and integration}
The probability space $(\Omega,\mathcal{F},\mathbf{P})$.\\
% sigma域:对偶操作,可数并(交,由DeMoregan等价)操作下封闭
def, $\mathcal{F}\subseteq 2^\Omega$ is $\sigma$-algebra (or $\sigma$-field) if: (a) $\Omega\in \mathcal{F}$; (b) if $A\in \mathcal{F}$ then $A^c\in \mathcal{F}$, where $A^c=\Omega\setminus A$; (c) If $A_i\in \mathcal{F}$ for $i=1,2,3,...$ then $\bigcup_i A_i\in \mathcal{F}$.\\
% 测量:非负,可数不交集可加性
def, a pair $(\Omega, \mathcal{F})$ with $\sigma$-algebra of subsets of $\Omega$ is called a measurable space. a measure $\mu$ is any countably additive non-negative set function on this space. i.e. $\mu: \mathcal{F}\rightarrow [0,\infty]$ with properties: (a) $\mu(A)\geq \mu(\emptyset)=0, \forall A\in \mathcal{F}$. (b) $\mu(\bigcup_n A_n) = \sum_n \mu(A_n) $ for any countable collection of disjoint sets $A_n\in \mathcal{F}$.\\
when $\mu(\Omega)=1$, called probability measure, often label by $\mathbf{P}$.\\
def, called measure space is $(\Omega,\mathcal{F},\mu)$; called probability space is $(\Omega, \mathcal{F}, \mathbf{P})$.\\
% 类似于实数区间可以无限微分找到
def, non-atomic: if $\mathbf{P}(A)>0$ implies $\exists B\in \mathcal{F}, B\subset A$ with $0<\mathbf{P}(B)<\mathbf{P}(A)$.
% 最小生成sigma域
def, the $\sigma$-algebra generated by the sets $A_\alpha$:
\begin{align}
	\sigma(\{A_\alpha\}) = \bigcap \{\mathcal{G}: \mathcal{G}\subseteq 2^\Omega \text{is a $\sigma$-algebra,} A_\alpha\in \mathcal{G},\forall\alpha\in \Gamma\}
\end{align}
% 某个拓扑上所有开集的最小生成sigma域;在R上有多种表述,EX1.1.17
Borel $\sigma$-algebra.\\
% 即B并没有包含所有R的子集
prop, $\exists A\subset \mathbb{R}$, that $A\notin \mathcal{B}$.\\
% 与sigma域的区别在于,满足有限并(交)操作封闭即可,而sigma域满足可数无穷
def, a collection $\mathcal{A}$ of subsets of $\Omega$ is an algebra(or a field) if: (a) $\Omega\in\mathcal{A}$, (b) if $A\in\mathcal{A}$ then $A^c\in\mathcal{A}$, (c) if $A,B\in\mathcal{A}$, then $A\cup B\in \mathcal{A}$.
% 最小生成域
$f(\mathcal{A})$, the intersection of all algebras of subsets of $\Omega$ containing $\mathcal{A}$.\\
% 域上可数可加函数,可扩充为在sigma域上的测量,若原函数有限,则扩充唯一
Caratheodory's extension theorem. If $\mu_0: \mathcal{A}\rightarrow [0,\infty]$ is a countably additive set function on an algebra $\mathcal{A}$ then there exists a measure $\mu$ on $(\Omega, \sigma(\mathcal{A}))$ s.t. $\mu=\mu_0$ on $\mathcal{A}$. Furthermore, if $\mu_0(\Omega)<\infty$ then such a measure $\mu$ is unique.
\end{document}