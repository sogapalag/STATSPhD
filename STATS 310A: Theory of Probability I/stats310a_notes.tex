%%%%%%%%%%%%%%%%%%%%%%%%%%%%%%%%%%%%%%%%%
% Short Sectioned Assignment
% LaTeX Template
% Version 1.0 (5/5/12)
%
% This template has been downloaded from:
% http://www.LaTeXTemplates.com
%
% Original author:
% Frits Wenneker (http://www.howtotex.com)
%
% License:
% CC BY-NC-SA 3.0 (http://creativecommons.org/licenses/by-nc-sa/3.0/)
%
%%%%%%%%%%%%%%%%%%%%%%%%%%%%%%%%%%%%%%%%%

%----------------------------------------------------------------------------------------
%	PACKAGES AND OTHER DOCUMENT CONFIGURATIONS
%----------------------------------------------------------------------------------------

\documentclass[paper=a4, fontsize=11pt]{scrartcl} % A4 paper and 11pt font size

\usepackage[T1]{fontenc} % Use 8-bit encoding that has 256 glyphs
\usepackage{fourier} % Use the Adobe Utopia font for the document - comment this line to return to the LaTeX default
\usepackage[english]{babel} % English language/hyphenation
\usepackage{amsmath,amsfonts,amsthm} % Math packages

\usepackage{sectsty} % Allows customizing section commands
\allsectionsfont{\centering \normalfont\scshape} % Make all sections centered, the default font and small caps
\usepackage{url}
\usepackage{fancyhdr} % Custom headers and footers
\pagestyle{fancyplain} % Makes all pages in the document conform to the custom headers and footers
\fancyhead{} % No page header - if you want one, create it in the same way as the footers below
\fancyfoot[L]{} % Empty left footer
\fancyfoot[C]{} % Empty center footer
\fancyfoot[R]{\thepage} % Page numbering for right footer
\renewcommand{\headrulewidth}{0pt} % Remove header underlines
\renewcommand{\footrulewidth}{0pt} % Remove footer underlines
\setlength{\headheight}{13.6pt} % Customize the height of the header

\numberwithin{equation}{section} % Number equations within sections (i.e. 1.1, 1.2, 2.1, 2.2 instead of 1, 2, 3, 4)
\numberwithin{figure}{section} % Number figures within sections (i.e. 1.1, 1.2, 2.1, 2.2 instead of 1, 2, 3, 4)
\numberwithin{table}{section} % Number tables within sections (i.e. 1.1, 1.2, 2.1, 2.2 instead of 1, 2, 3, 4)

\setlength\parindent{0pt} % Removes all indentation from paragraphs - comment this line for an assignment with lots of text
\def \cov {\text{Cov}}
\def \var {\text{Var}}

\title{STATS 310A: Theory of Probability I}
% notes by Amir Dembo; first 3 chapters.
% Probability and Measure, 3rd. billingsley
% probability with martingales, williams
\author{sogapalag}

\date{\normalsize\today}

\begin{document}

\maketitle

\section{Probability, measure and integration}
\subsection{Probability spaces, measures and $\sigma$-algebras}
The probability space $(\Omega,\mathcal{F},\mathbf{P})$.\\
% sigma域(sigma代数):对偶操作,可数并(交,由DeMoregan等价)操作下封闭
def, $\mathcal{F}\subseteq 2^\Omega$ is $\sigma$-algebra (or $\sigma$-field) if: (a) $\Omega\in \mathcal{F}$; (b) if $A\in \mathcal{F}$ then $A^c\in \mathcal{F}$, where $A^c=\Omega\setminus A$; (c) If $A_i\in \mathcal{F}$ for $i=1,2,3,...$ then $\bigcup_i A_i\in \mathcal{F}$.\\
% 测度:非负,可数不交集可加性
def, a pair $(\Omega, \mathcal{F})$ with $\sigma$-algebra of subsets of $\Omega$ is called a measurable space. a measure $\mu$ is any countably additive non-negative set function on this space. i.e. $\mu: \mathcal{F}\rightarrow [0,\infty]$ with properties: (a) $\mu(A)\geq \mu(\emptyset)=0, \forall A\in \mathcal{F}$. (b) $\mu(\bigcup_n A_n) = \sum_n \mu(A_n) $ for any countable collection of disjoint sets $A_n\in \mathcal{F}$.\\
when $\mu(\Omega)=1$, called probability measure, often label by $\mathbf{P}$.\\
def, called measure space is $(\Omega,\mathcal{F},\mu)$; called probability space is $(\Omega, \mathcal{F}, \mathbf{P})$.\\
% 类似于实数区间可以无限微分找到
def, non-atomic: if $\mathbf{P}(A)>0$ implies $\exists B\in \mathcal{F}, B\subset A$ with $0<\mathbf{P}(B)<\mathbf{P}(A)$.\\
% 最小生成sigma代数
def, the $\sigma$-algebra generated by the sets $A_\alpha$:
\begin{align}
	\sigma(\{A_\alpha\}) = \bigcap \{\mathcal{G}: \mathcal{G}\subseteq 2^\Omega \text{is a $\sigma$-algebra,} A_\alpha\in \mathcal{G},\forall\alpha\in \Gamma\}
\end{align}
% 某个拓扑上所有开集的最小生成sigma代数;在R上有多种表述,EX1.1.17
Borel $\sigma$-algebra.\\
% 即B并没有包含所有R的子集; recall Banach-Tarski paradox, C is not in Borel.
prop, $\exists A\subset \mathbb{R}$, that $A\notin \mathcal{B}$.\\
% 代数; 与sigma域的区别在于,满足有限并(交)操作封闭即可,而sigma代数满足可数无穷
def, a collection $\mathcal{A}$ of subsets of $\Omega$ is an algebra(or a field) if: (a) $\Omega\in\mathcal{A}$, (b) if $A\in\mathcal{A}$ then $A^c\in\mathcal{A}$, (c) if $A,B\in\mathcal{A}$, then $A\cup B\in \mathcal{A}$.
% 最小生成代数
$f(\mathcal{A})$, the intersection of all algebras of subsets of $\Omega$ containing $\mathcal{A}$.\\
% 域上可数可加函数,可扩充为在sigma代数上的测度,若原函数有限,则扩充唯一
Caratheodory's extension theorem. If $\mu_0: \mathcal{A}\rightarrow [0,\infty]$ is a countably additive set function on an algebra $\mathcal{A}$ then there exists a measure $\mu$ on $(\Omega, \sigma(\mathcal{A}))$ s.t. $\mu=\mu_0$ on $\mathcal{A}$. Furthermore, if $\mu_0(\Omega)<\infty$ then such a measure $\mu$ is unique.\\
% 完备(补全),测度为0的所有集合的所有子集
completion, $N=\{N:N\subset A, A\in F, \mu(A)=0$.\\
def, $\pi$-system, closed under finite intersections.\\
def, $\lambda$-system, contain $\Omega$ and $B\setminus A,\forall A\subset B,A,B\in \mathcal{L}$. which is also closed under monotone increasing limits(i.e. if $A_i\in\mathcal{L}$ and $A_i\uparrow A$, then $A\in\mathcal{L}$)\\
prop, $\sigma$-algebra iff both $\pi$-system and $\lambda$-system.\\
% lambda系统的pi系统子集的生成sigma代数仍为子集
Dynkin's $\pi$-$\lambda$ theorem. $P_\pi \subset L_\lambda$, then $\sigma(P_\pi)\subset L_\lambda$.\\
% increasing, nonnegative, countably sub-additive
def, outer measure, $\mu^*$.\\
% A是lambda可测;即对测度空间里所有F进行了可行性划分
def, $\lambda$-measureable set $A\in \mathcal{F}$, $\lambda\geq 0,\lambda(\emptyset)=0$, if $\lambda(F)=\lambda(F\cap A)+\lambda(F\cap A^c),\forall F\in\mathcal{F}$.\\
Caratheodory's lemma, $(\Omega, \mathcal{G},\mu^*)$ is a measure space.\\
% 封闭under 单调极限
def, monotone class\\
% 单调类的代数子集的生成sigma代数仍为子集
Halmos's monotone class theorem, algebra $\mathcal{A}$, monotone class $\mathcal{M}$, $\mathcal{A}\subset \mathcal{M}$, then $\sigma(\mathcal{A})\subset \mathcal{M}$.

\subsection{Random variables and their distribution}
def, measurable mapping between two measurable space $X:\Omega\rightarrow \mathbb{S}$, Random Variable (R.V.): $X^{-1}(B)\in \mathcal{F},\forall B\in\mathcal{S}$\\
def, indicator function $I_A(\omega)$.\\
def, simple function $X(\omega)=\sum_{n=1}^N c_n I_{A_n}(\omega)$.\\
prop, $\forall X\in R.V.$, $\exists$ sequence $X_n\in SF$, s.t. $X_n(\omega)\rightarrow X(\omega)$, as $n\rightarrow \infty$, $\forall \omega\in \Omega$.\\
% detail omit
Monotone clas theorem(less restrictive version), contain $m\sigma(\mathcal{P})$\\
% a.e.(almost everywhere); a.s.(almost sure); w.p.1 (with probability 1). equiv.
def, almost sure for R.V. denote $X\stackrel{a.s.}{=} Y$, with $P(\{\omega:X(\omega)\neq Y(\omega)\})=0$.\\
% 为减少判断是否R.V. 即减少判断是否在原测度空间中
Thm, if $\mathcal{S}=\sigma(\mathcal{A})$, $X^{-1}(A)\in\mathcal{F},\forall A\in\mathcal{A}$, then $X$ is $(\mathbb{S},\mathcal{S})$-valued R.V.\\
% sigma代数的反射原像 为sigma代数
% 即随机变量X所需包含所有信息的最小空间F^X
mapping $X$, $\sigma$-algebra $\mathcal{S}$, denote collection $\{X^{-1}(B):B\in\mathcal{S}\}$(which is $\sigma$-algebra) as $\sigma(X)$ or $\mathcal{F}^X$.\\
% R.V. 原像为原测度空间子集
$X$ is R.V. iff $\sigma(X)\subset \mathcal{F}$.\\
% 即sigma(X)可以用生成集对应的原像生成
if R.V. X, with $\mathcal{S}=\sigma(\mathcal{A})$, then $\sigma(X)=\sigma(X^{-1}(\mathcal{A}))$.\\
map to Borel $(\mathbb{R},\mathcal{B})$,
\begin{align}
	\sigma(X) = \sigma(\{\omega:X(\omega)\leq \alpha,\alpha\in\mathbb{R}\}) = 
		\sigma(\{\omega:X(\omega)\leq q,q\in\mathbb{Q}\}) 
\end{align}
general, index countable $i$ or uncountable $\gamma$, $\sigma(X_\gamma, \gamma\in \Gamma)$

% https://web.stanford.edu/~montanar/TEACHING/Stat310A/handouts.html
% omit below, just simple comment
% F:R->[0,1]分布函数(DF) iff 不减;负无穷极限0,正无穷极限1;右连续。
% 测度空间可以complete,并上N,即测度为0的集的所有子集
% f>=0, (a,b], Riemann可积 则 Lebesgue可测(显然)
% DEF2.2.1
% liminf, cupcap, 即在足够大的n后出现在所有 eventually(ev.) subset below
% 想象一个 交 不停加上(不再交)1,2,.. (释放)的部分,即最终被释放出来的,即w最终一直出现在某个n后
% limsup, capcup, 即出现无限次的 infinitly often(i.o.)
% 想象一个 并 不停扣掉(不再并)1,2,.. (失去)的部分,即不会被扣掉的,即w无限(不是一直)出现在某些n
% P(limsup) >=  limsup P >= liminf P >= P(liminf)
% by DeMorgan's, {An ev.} = {An^c i.o.}^c 即最终一直出现 <=> 对偶不要无限出现,(有限的)
% U.I.一致可积,limsup_a E[X_a| |X_a|>M]=0,即存在足够大的M使得所有>M的期望上确界极限为0
% def, Lq space,(1<=q<infty) 满足|X|_q = E[|X|^q]^{1/q}<infty, 即^q绝对可积空间
% thm, if X_n->^{p} X,即limP(>e)=0; 则 {X_n} is U.I.  <=> X_n->^{L1} X, 即 E[|X_n-X|] -> 0  <=> X_n integrable n<infty and E[|X_n|]->E[|X|]
% 1.3.27 注意区分各种趋近
% Lq趋近即度量距离->0,即E|-|^q->0; 距离趋近0
% a.s. P(X neq Y)=0; 概率为0
% p ,  P(|-|>e) -> 0. 即概率趋近0
% q \geq r, (X_n -> X) Lq趋近 =>  Lr趋近 
%  (X_n -> X) Lq趋近 => p趋近
%  (X_n -> X)Lq趋近, (X_n -> Y)p趋近  =>  X->Y (a.s.)
%  (X_n -> X) a.s.  =>  (X_n -> X) p趋近
% Weak law of large numbers: i.i.d.; xP(X>x)->0, as x->infty; 则算术平均 n^{-1}S_n -> u_n , p趋近, u_n=E[X I_{|X|leq n}]
% Strong law of large numbers: p-i.i.d, E[X_-,+]其一有限,则 n^{-1}S_n -> EX_1, (a.s.) 即只要两两独立的同变量,EX_-or+ 有一个有限, 就有(a.s.)-> EX_1就是well defined.(无穷也可以) 
% 1.3.17
% Holder's inequality,p,q>1, 1/p+1/q=1, 则  E[|XY|]\leq |X|_p|X|_q
% Minkowski's inequality, 即在Lp空间三角不等式,|X+Y|_p \leq |X|_p+|Y|_p
% 1.4.4
% def, P-mutually independent(任意k,均满足独立式子), 对于events; 对于collection of events;对于R.V. 即collect, sigma(X_a) P-m.i.
% thm, pi系统A_i; G_i=sigma(A_i)则 A_i P互独 => G_i P互独
% def, tail of 随机过程{X_k}, 1.4.9
% Kolmogorov’s 0-1 law: {X_k} P-互独,则tail是P-trivial (即P={0 or 1})
% 1.4.22 and web lec.5
% Kolmogorov’s extension theorem: (R^n, B_{R^n}) -> (R^N, B_c) 概率测度可唯一推广;即序列(x_1,...,x_n)可唯一推广至(x_1,x_2,...)
% web lec.6
% Law of iterated logarithm: indep,mu=0,sigma^2=1, (sum X)=S_n~ sqrt(2n loglog n); 即P(limsup . =1)=1
% 2.2.4 技巧 {1/n}; 划分了收敛发散;以此看渐近情况 是i.o. 1还是0
% lemma, Borel-Cantelli I, sum^infty P(A_n)<infty, 则 P(A_n i.o.)=0;
% relax condition, sum^infty P(A_n \ A_{n+1})< infty, P(A_n)->0 则 P(A_n i.o.)=0;
% lemma, Borel-Cantelli II,{A_n}互独,\sum^infty P(A_n)=infty 则 P(A_n i.o.)=1
% 结合 B-C I,II,若互独A_n, 要么 几乎所有结果 都是 无限出现或有限出现
% Thm, X_n ->^{p} X,  <=>  X_n的每个子序列X_{n(m)},都存在子序列X_{n(m(k))} ->^{a.s.} X.
% prop, {X_i} geq 0, identical, pairwise-indep; then n^{-1}S_n ->^{a.s.} EX_1.
% Glivenko-Cantelli, i.i.d., sup |F_n-F|->0 (a.s.)
% 2.3.15 Kolmogorov’s maximal inequality. 比Chebyshev's更强
% 3.1.2 (Central Limit Theorem) i.i.d;    .optinal, m-i. Lindeberg's CLT
% Kolmogorov’s three series theorem; 随机级数的收敛(a.s.) iff
% Characteristic functions, Phi_X(t) = E[e^{itX}]
% 3.3.12
% Lévy’s inversion theorem. 从特征函数Phi恢复分布函数F, 若在R上积分有限,还可直接计算f; f<-> Phi, duality. recall (Laplace's transform)
\end{document}