%%%%%%%%%%%%%%%%%%%%%%%%%%%%%%%%%%%%%%%%%
% Short Sectioned Assignment
% LaTeX Template
% Version 1.0 (5/5/12)
%
% This template has been downloaded from:
% http://www.LaTeXTemplates.com
%
% Original author:
% Frits Wenneker (http://www.howtotex.com)
%
% License:
% CC BY-NC-SA 3.0 (http://creativecommons.org/licenses/by-nc-sa/3.0/)
%
%%%%%%%%%%%%%%%%%%%%%%%%%%%%%%%%%%%%%%%%%

%----------------------------------------------------------------------------------------
%	PACKAGES AND OTHER DOCUMENT CONFIGURATIONS
%----------------------------------------------------------------------------------------

\documentclass[paper=a4, fontsize=11pt]{scrartcl} % A4 paper and 11pt font size

\usepackage[T1]{fontenc} % Use 8-bit encoding that has 256 glyphs
\usepackage{fourier} % Use the Adobe Utopia font for the document - comment this line to return to the LaTeX default
\usepackage[english]{babel} % English language/hyphenation
\usepackage{amsmath,amsfonts,amsthm} % Math packages

\usepackage{sectsty} % Allows customizing section commands
\allsectionsfont{\centering \normalfont\scshape} % Make all sections centered, the default font and small caps
\usepackage{url}
\usepackage{fancyhdr} % Custom headers and footers
\pagestyle{fancyplain} % Makes all pages in the document conform to the custom headers and footers
\fancyhead{} % No page header - if you want one, create it in the same way as the footers below
\fancyfoot[L]{} % Empty left footer
\fancyfoot[C]{} % Empty center footer
\fancyfoot[R]{\thepage} % Page numbering for right footer
\renewcommand{\headrulewidth}{0pt} % Remove header underlines
\renewcommand{\footrulewidth}{0pt} % Remove footer underlines
\setlength{\headheight}{13.6pt} % Customize the height of the header

\numberwithin{equation}{section} % Number equations within sections (i.e. 1.1, 1.2, 2.1, 2.2 instead of 1, 2, 3, 4)
\numberwithin{figure}{section} % Number figures within sections (i.e. 1.1, 1.2, 2.1, 2.2 instead of 1, 2, 3, 4)
\numberwithin{table}{section} % Number tables within sections (i.e. 1.1, 1.2, 2.1, 2.2 instead of 1, 2, 3, 4)

\setlength\parindent{0pt} % Removes all indentation from paragraphs - comment this line for an assignment with lots of text


\title{Exercises}
%A First Course in Probability, Ross. 8th edition
\author{sogapalag}

\date{\normalsize\today}

\begin{document}

\maketitle
\begin{itemize}
	\item[1.17] ${10 \choose 7} 7! = 604800$. Or $10*9*\dots*4 = 604800$.
	\item[1.20] 
	\begin{itemize}
		\item[(a)] ${8\choose 5} - {6\choose 3} =36= 2{6\choose 4}+{6\choose 5}$; 
		\item[(b)] ${8\choose 5} - 2*{6\choose 4} =26= {6\choose 3} + {6\choose 5}$.
	\end{itemize}
		%$4^8= 65536$; $4^8-4*3^8+12*2^8+4 = 42368$. thought different blackboard.
	\item[1.31] ${8+4-1\choose 4-1} = 165$; ${8-1\choose 4-1}=35$.
	\item[1.32] ${8+6-1 \choose 6-1} = 1287$; ${5+6-1 \choose 6-1}*{3+6-1 \choose 6-1} = 14112$.
	\item[1.33] ${9+4-1 \choose 4-1} = 220$; ${20-7+3-1 \choose 3-1} +2{13\choose 2} +{14\choose 2} +220 = 572$.
	\item[T1.5] $\sum_{i=k}^n {n\choose i}$.
	\item[T1.11] when last in $i$, sum of choices.
	\item[T1.12]
	\begin{itemize}
		\item[(a)] $x_j$ register, for each chosen which, $+1$, then sum is left; by looking each register, which be chosen $2^{n-1}$, since there are $n$ registers, sum is right. another way consider $(1_a+1_b)^n$.
		\item[(b)]
		\begin{align}
			\sum k (k-1) {n\choose k } &= n(n-1) \sum {n-2 \choose k-2}\\
			&= n(n-1)2^{n-2}
		\end{align}
		then plus $\sum k{n\choose k}$.\qed
		\item[(c)] calculate $\sum k(k-1)(k-2){n\choose k}$.\qed
	\end{itemize}
	\item[T1.13] consider $(1+(-1))^n=0$.
	\item[T1.15]
	\begin{itemize}
		\item[(a)] when maximum $x_k = j$.\qed
		\item[(b)] 35.
	\end{itemize}
	\item[T1.16]
	\begin{itemize}
		\item[(b)] when last group with $i$ players tie for.\qed
		\item[(c)] let $j=n-i$.\qed
	\end{itemize}
	\item[T1.20] $\sum x_j=n$ with $x_j\geq m_j$, let $y_j = x_j-m_j$, then equiv to $\sum y_j = n-\sum m_j$ with $y_j\geq 0$, so answer is ${n-\sum m_j + r -1 \choose r-1}$.
	\item[T1.21] consider choosing $k$ of $x_i$ set to zero, then left $x_j>0$.\qed 
	\item[T1.22] for $\frac{\partial^r}{\partial x_{i_1}\dots \partial x_{i_r}}$, different $i_j$ represent different PDE, thus total $r^n$.
	\item[T1.23]
	\begin{align}
		\sum_{i=1}^k {i+n-1 \choose n-1} &= \sum_{i=1}^k \left[{i+n \choose n} - {i+n-1 \choose n}\right]\\
		&= {k+n \choose n}
	\end{align}
	\item[S1.14] recall T1.23, ${k\choose n}$. Another solution, consider $\{1,2,...,k\}$, select $n$ items, which one-to-one a vector which $x_i = y_i-y_{i-1}$.
	\item[S1.17] both in $k$ or $n-k$, or one each.\qed
	\item[2.15]
	\begin{itemize}
		\item[(a)] $4*{13 \choose 5} =  5148$; $0.00198$
		\item[(b)] $13*{4\choose 2}*{12\choose 3}*4^3 = 1098240$; $0.42257$
		\item[(c)] ${13\choose 2}{4\choose 2}^2*11*4 = 123552$; $0.04754$
		\item[(d)] $13*{4\choose 3}{12\choose 2}*4^2 = 54912$; $0.02113$
		\item[(e)] $13*12*4=624$. $0.00024$
	\end{itemize}
	\item[2.17] $8!=40320$.
	\item[2.45] 
	\begin{itemize}
		\item[(a)] shuffle the keys, solution in $k$th place, $P=1/n$.
		\item[(b)] $\frac{(n-1)^{k-1}}{n^k}$.
	\end{itemize}
	\item[2.46] consider general case, $n$ distinct items distribute to $k$ distinct groups, i.e. $\sum_{i=1}^k x_i= n$ with $x_i\geq 0$, note $k^n$ choices. Then we want find $P(\exists x\geq 2)$, which complement is $x_j\leq 1,\forall j$, with ${k \choose n} n!$ choices. Thus we get
	\begin{align}
	P(\exists x\geq 2) &= 1 - \frac{ {k \choose n}n! } { k^n } \\
		&=1 - \frac{k!}{k^n (k-n)!} \\
		&=1 - \frac{k(k-1)\dots (k-n+1)}{k^n}
	\end{align}
	let $k=12$, $P\geq 1/2$, conclude $n\geq 5$. if $k=365$, which called birthday problem, $n\geq 23$.
	\item[2.51] $\frac{ {n\choose m} (N-1)^{n-m}}{N^n}$.
	\item[2.53] one solution calculate $P(\bigcup E) = 23/35$, then answer $12/35$. Another solution, by the symmetry between couples and symmetry between husband and wife, we can consider $h_1<h_2<h_3<h_4$ and $h_i<w_i, \forall i$, then $h_1=1$, there is $7$ choices for $w_1$, after which, $h_2$ will be the minimal unseated, then $w_2$ has $5$ choices, after which $h_3$ seated, $w_3$ has $3$ choices. Thus there is $7*5*3=105$ arrangements. Now consider which arrangement is legal, for $h_1h_2h_3h_4$, left $3*3!=18$ choices; $h_1h_2h_3w_1$, note $h_4<7$, then there is $2*2!+2=6$ choices; since $w_2$ like as $w_1$, so $h_1h_2h_3$ has $18+2*6=30$ total; for $h_1h_2w_1h_3$, $2*2! + 1=5$ choices; for $h_1h_2w_1w_2h_3h_4$, only $1$ choice with connection $w_3w_4$. Then legal choices total $30+5+1=36$, $P=36/105 = 12/35$.
	\item[2.54] let $E_i$ be void in suit $i$,
	\begin{align}
		P(\bigcup E_i) =& \sum_{r=1}^4 (-1)^{r+1} \sum_{i_1<\dots<i_r} P(E_1\dots E_r) \\
			=& \frac{1}{{52\choose 13}} \left[ {4\choose 1}{39\choose 13} -6 {26\choose 13} 
			 + 4 {13 \choose 13} \right] \\
			 =& \frac{4(39*38*\dots*27) - 6(26*25*\dots*14) + 4(13!)} { 52*51*\dots*40 }\\
			 \approx & 0.051
	\end{align}
	\item[2.56] consider a vs b, $P(\text{a win}| a,b) = 5/9$; b vs c $P(\text{b win}| b,c) = 5/9$; consider a vs c, $P(\text{c win}|a,c)= 5/9$. then it's like Rock-paper-scissors game. Then B player is better.
	\item[T2.8] select $i$ items from $\{1,...,n\}$ to join $\{n+1\}$, left do partition, let $T_0=1$ thus
	\begin{align}
		T_{n+1} &= \sum_{i=0}^n {n\choose i} T_{n-i} \\
			&= 1 + \sum_{k=1}^n {n\choose k} T_k
	\end{align}\qed
	\item[T2.10] note $P(FG) = P((E\cup E^c)FG)= P(EFG\cup E^cFG)  = P(EFG)+P(E^cFG)$.\qed
	\item[T2.16] 
	\begin{align}
		& P(E_1E_2\dots E_n) \geq P(E_1)+\dots+P(E_n) -(n-1)\\
		\Leftrightarrow & P(\bigcap E_i)-1 \geq \sum [P(E_i)-1] \\
		\Leftrightarrow & P(\bigcup E_i^c) \leq \sum P(E_i^c)
	\end{align}\qed
	\item[T2.17] suppose the Nth man take the $i\in\{1,...,N-1\}$ hat, then $i$th man treat $N$th hat as his own, then $A_(N-1)$; since $N$th hat is just virtual hat his own, which can take by him indeed, then $A_(N-2)$.\
	\item[T2.18] when last is $T$, there is $f_{n-1}$ ways; when last is $H$, $n-1$ must be $T$, then $f_{n-2}$ ways.\qed
	\item[T2.19]
	\begin{align}
		P(k) &= \frac{ {n\choose 1}{n-1 \choose r-1}{m \choose k-r} (k-1)! }{ {n+m\choose k} k!} \\
			&= \frac{ r{n\choose r}{m\choose k-r}}{k{n+m\choose k }}
	\end{align}
	\item[S2.11] $4{13\choose 2} 13^3 / {52 \choose 5} = 0.26375$.
	\item[S2.20] satisfy iff the last is blue , i.e. $P = 10/(20+10) = 1/3$.
\end{itemize}

\end{document}