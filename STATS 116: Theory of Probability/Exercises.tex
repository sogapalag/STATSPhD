%%%%%%%%%%%%%%%%%%%%%%%%%%%%%%%%%%%%%%%%%
% Short Sectioned Assignment
% LaTeX Template
% Version 1.0 (5/5/12)
%
% This template has been downloaded from:
% http://www.LaTeXTemplates.com
%
% Original author:
% Frits Wenneker (http://www.howtotex.com)
%
% License:
% CC BY-NC-SA 3.0 (http://creativecommons.org/licenses/by-nc-sa/3.0/)
%
%%%%%%%%%%%%%%%%%%%%%%%%%%%%%%%%%%%%%%%%%

%----------------------------------------------------------------------------------------
%	PACKAGES AND OTHER DOCUMENT CONFIGURATIONS
%----------------------------------------------------------------------------------------

\documentclass[paper=a4, fontsize=11pt]{scrartcl} % A4 paper and 11pt font size

\usepackage[T1]{fontenc} % Use 8-bit encoding that has 256 glyphs
\usepackage{fourier} % Use the Adobe Utopia font for the document - comment this line to return to the LaTeX default
\usepackage[english]{babel} % English language/hyphenation
\usepackage{amsmath,amsfonts,amsthm} % Math packages

\usepackage{sectsty} % Allows customizing section commands
\allsectionsfont{\centering \normalfont\scshape} % Make all sections centered, the default font and small caps
\usepackage{url}
\usepackage{fancyhdr} % Custom headers and footers
\pagestyle{fancyplain} % Makes all pages in the document conform to the custom headers and footers
\fancyhead{} % No page header - if you want one, create it in the same way as the footers below
\fancyfoot[L]{} % Empty left footer
\fancyfoot[C]{} % Empty center footer
\fancyfoot[R]{\thepage} % Page numbering for right footer
\renewcommand{\headrulewidth}{0pt} % Remove header underlines
\renewcommand{\footrulewidth}{0pt} % Remove footer underlines
\setlength{\headheight}{13.6pt} % Customize the height of the header

\numberwithin{equation}{section} % Number equations within sections (i.e. 1.1, 1.2, 2.1, 2.2 instead of 1, 2, 3, 4)
\numberwithin{figure}{section} % Number figures within sections (i.e. 1.1, 1.2, 2.1, 2.2 instead of 1, 2, 3, 4)
\numberwithin{table}{section} % Number tables within sections (i.e. 1.1, 1.2, 2.1, 2.2 instead of 1, 2, 3, 4)

\setlength\parindent{0pt} % Removes all indentation from paragraphs - comment this line for an assignment with lots of text


\title{Exercises}
%A First Course in Probability, Ross.
\author{sogapalag}

\date{\normalsize\today}

\begin{document}

\maketitle
\begin{itemize}
	\item[1.17] ${10 \choose 7} 7! = 604800$. Or $10*9*\dots*4 = 604800$.
	\item[1.20] 
	\begin{itemize}
		\item[(a)] ${8\choose 5} - {6\choose 3} =36= 2{6\choose 4}+{6\choose 5}$; 
		\item[(b)] ${8\choose 5} - 2*{6\choose 4} =26= {6\choose 3} + {6\choose 5}$.
	\end{itemize}
		%$4^8= 65536$; $4^8-4*3^8+12*2^8+4 = 42368$. thought different blackboard.
	\item[1.31] ${8+4-1\choose 4-1} = 165$; ${8-1\choose 4-1}=35$.
	\item[1.32] ${8+6-1 \choose 6-1} = 1287$; ${5+6-1 \choose 6-1}*{3+6-1 \choose 6-1} = 14112$.
	\item[1.33] ${9+4-1 \choose 4-1} = 220$; ${20-7+3-1 \choose 3-1} +2{13\choose 2} +{14\choose 2} +220 = 572$.
	\item[T1.5] $\sum_{i=k}^n {n\choose i}$.
	\item[T1.11] when last in $i$, sum of choices.
	\item[T1.12]
	\begin{itemize}
		\item[(a)] $x_j$ register, for each chosen which, $+1$, then sum is left; by looking each register, which be chosen $2^{n-1}$, since there are $n$ registers, sum is right. another way consider $(1_a+1_b)^n$.
		\item[(b)]
		\begin{align}
			\sum k (k-1) {n\choose k } &= n(n-1) \sum {n-2 \choose k-2}\\
			&= n(n-1)2^{n-2}
		\end{align}
		then plus $\sum k{n\choose k}$.\qed
		\item[(c)] calculate $\sum k(k-1)(k-2){n\choose k}$.\qed
	\end{itemize}
	\item[T1.13] consider $(1+(-1))^n=0$.
	\item[T1.15]
	\begin{itemize}
		\item[(a)] when maximum $x_k = j$.\qed
		\item[(b)] 35.
	\end{itemize}
	\item[T1.16]
	\begin{itemize}
		\item[(b)] when last group with $i$ players tie for.\qed
		\item[(c)] let $j=n-i$.\qed
	\end{itemize}
	\item[T1.20] $\sum x_j=n$ with $x_j\geq m_j$, let $y_j = x_j-m_j$, then equiv to $\sum y_j = n-\sum m_j$ with $y_j\geq 0$, so answer is ${n-\sum m_j + r -1 \choose r-1}$.
	\item[T1.21] consider choosing $k$ of $x_i$ set to zero, then left $x_j>0$.\qed 
	\item[T1.22] for $\frac{\partial^r}{\partial x_{i_1}\dots \partial x_{i_r}}$, different $i_j$ represent different PDE, thus total $r^n$.
	\item[T1.23]
	\begin{align}
		\sum_{i=1}^k {i+n-1 \choose n-1} &= \sum_{i=1}^k \left[{i+n \choose n} - {i+n-1 \choose n}\right]\\
		&= {k+n \choose n}
	\end{align}
	\item[S1.14] recall T1.23, ${k\choose n}$. Another solution, consider $\{1,2,...,k\}$, select $n$ items, which one-to-one a vector which $x_i = y_i-y_{i-1}$.
	\item[S1.17] both in $k$ or $n-k$, or one each.\qed
\end{itemize}

\end{document}