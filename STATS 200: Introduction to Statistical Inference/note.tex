%%%%%%%%%%%%%%%%%%%%%%%%%%%%%%%%%%%%%%%%%
% Short Sectioned Assignment
% LaTeX Template
% Version 1.0 (5/5/12)
%
% This template has been downloaded from:
% http://www.LaTeXTemplates.com
%
% Original author:
% Frits Wenneker (http://www.howtotex.com)
%
% License:
% CC BY-NC-SA 3.0 (http://creativecommons.org/licenses/by-nc-sa/3.0/)
%
%%%%%%%%%%%%%%%%%%%%%%%%%%%%%%%%%%%%%%%%%

%----------------------------------------------------------------------------------------
%	PACKAGES AND OTHER DOCUMENT CONFIGURATIONS
%----------------------------------------------------------------------------------------

\documentclass[paper=a4, fontsize=11pt]{scrartcl} % A4 paper and 11pt font size

\usepackage[T1]{fontenc} % Use 8-bit encoding that has 256 glyphs
\usepackage{fourier} % Use the Adobe Utopia font for the document - comment this line to return to the LaTeX default
\usepackage[english]{babel} % English language/hyphenation
\usepackage{amsmath,amsfonts,amsthm} % Math packages

\usepackage{sectsty} % Allows customizing section commands
\allsectionsfont{\centering \normalfont\scshape} % Make all sections centered, the default font and small caps
\usepackage{url}
\usepackage{fancyhdr} % Custom headers and footers
\pagestyle{fancyplain} % Makes all pages in the document conform to the custom headers and footers
\fancyhead{} % No page header - if you want one, create it in the same way as the footers below
\fancyfoot[L]{} % Empty left footer
\fancyfoot[C]{} % Empty center footer
\fancyfoot[R]{\thepage} % Page numbering for right footer
\renewcommand{\headrulewidth}{0pt} % Remove header underlines
\renewcommand{\footrulewidth}{0pt} % Remove footer underlines
\setlength{\headheight}{13.6pt} % Customize the height of the header

\numberwithin{equation}{section} % Number equations within sections (i.e. 1.1, 1.2, 2.1, 2.2 instead of 1, 2, 3, 4)
\numberwithin{figure}{section} % Number figures within sections (i.e. 1.1, 1.2, 2.1, 2.2 instead of 1, 2, 3, 4)
\numberwithin{table}{section} % Number tables within sections (i.e. 1.1, 1.2, 2.1, 2.2 instead of 1, 2, 3, 4)

\setlength\parindent{0pt} % Removes all indentation from paragraphs - comment this line for an assignment with lots of text
\def \cov {\text{Cov}}
\def \var {\text{Var}}

\title{STATS 200: Introduction to Statistical Inference}
%Mathematical Statistics and Data Analysis, 3rd, Rice.
\author{sogapalag}

\date{\normalsize\today}

\begin{document}

\maketitle

\section{Distributions Derived from the Normal Distribution}
def, chi-square distribution with $1$ degree of freedom, $U=Z^2$, where $Z$ is standard normal.\\
def, chi-square with $n$ degrees of freedom, $V=U_1+\dots+U_n$, independent $U_i$. denoted by $\chi_n^2$. which MGF is $M(t)=(1-2t)^{-n/2}$, is a gamma with parameter $(\alpha=n/2, \lambda=1/2)$.\\
$t_n$ distribution. $Z/\sqrt{U/n}$, where $Z\sim N(0,1)$, $U\sim \chi_n^2$.\\
$F_{m,n}$, $W=\frac{U/m}{V/n}$, where independent $U\sim \chi_m^2$, $V\sim \chi_n^2$.\\
independent $X_i\sim N(\mu,\sigma^2)$, def sample mean $\overline{X}=\frac{1}{n}\sum^n X_i$, sample varice $S^2= \frac{1}{n-1}\sum^n (X_i-\overline{X})^2$.\\
% is indep to vector, not vector's coords indep.
Thm, $\overline{X}$ is independent to vector $(X_1-\overline{X},...,X_n-\overline{X})$.\\
Corollary, $\overline{X}$ and $S^2$ are independent.\\
Thm, $(n-1)S^2/\sigma^2$ is chi-square with $n-1$ df.\\
Corollary, $\frac{\overline{X}-\mu}{S/\sqrt{n}} \sim t_{n-1}$.\\

\section{Survey Sampling}
srs, simple random sampling.\\
def, unbiased, $E[\widehat{\theta}]=\theta$.\\
lemma, srs without replacement, $\cov(X_i,X_j)=-\sigma^2/(N-1)$, $i\neq j$.\\
% finite population correction, approx 1 - sampling fraction
Thm,
\begin{align}
	\var(\overline{X}) = \frac{\sigma^2}{n}\frac{N-n}{N-1}
\end{align}
estimation of population variance, $\widehat{\sigma}^2 = \frac{1}{n}\sum^n(X_i-\overline{X})^2$.\\
Thm,
\begin{align}
	E[\widehat{\sigma}^2] = \sigma^2 \frac{n-1}{n}\frac{N}{N-1}
\end{align}
corollary, unbiased estimate of $\var(\overline{X})$ is
\begin{align}
	s_{\overline{X}}^2 &= \frac{\sigma^2}{n}\frac{n}{n-1}\frac{N-1}{N}\frac{N-n}{N-1}\\
		&=\frac{S^2}{n}(1-\frac{n}{N})
\end{align}
corollary, unbiased estimate of $\var(\widehat{p})$ is
\begin{align}
	s_{\widehat{p}}^2 = \frac{\widehat{p}(1-\widehat{p})}{n-1}(1-\frac{n}{N})
\end{align}
ratio estimate.\\
stratified estimate, $\overline{X}_s = \sum^L W_l \overline{X}_l$, where $W_l=N_l/N$.\\
$\var(\overline{X}_s)\approx \sum^L \frac{W_l^2 \sigma_l^2}{n_l}$.\\
% use Lagrange proof
Thm, minimize $\var(\overline{X}_s)$ with 
\begin{align}
	n_l = n \frac{W_l\sigma_l}{\sum^L W_l\sigma_l}	
\end{align}
corollary, use this optimal Neyman allocation. That
\begin{align}
	\var(\overline{X}_{so}) = \frac{(\sum^L W_l\sigma_l)^2}{n}
\end{align}
proportional allocation, $n_l = nW_l$.\\
Thm, with proportional allocation, ignoring the finite population correction,
\begin{align}
	\var(\overline{X}_{sp}) = \frac{\sum^L W_l\sigma_l^2}{n}
\end{align}
% lack sigma distri information; propor more 'average' than srs.
Thm, ignoring the finite population correction,
\begin{align}
	&\var(\overline{X}_{sp}) - \var(\overline{X}_{so}) = \frac{1}{n}\sum^L W_l(\sigma_l -\bar{\sigma})^2 \\
	&\var(\overline{X}) - \var(\overline{X}_{sp}) = \frac{1}{n}\sum^L W_l(\mu_l -\mu)^2
\end{align}

\section{fitting distribution}
% 关于X的分布函数可看作关于参数t的函数,对于已知数据向量(X),不同的t对应不同的概率,即找到使得已有数据向量(X)发生概率最大的参数
maximum likelihood estimate(MLE), $\text{lik}(\theta) = f(x_1,...,x_n|\theta)$, under i.i.d, log like, $l(\theta) = \sum_{i=1}^n \log [f(X_i|\theta)]$, then $l'(\theta)=0$ find maximum $\theta$ is estimate.\\
% 参数t也是变量,即用Bayes基于先验求后验f(t|X)
The Bayesian Approach to Parameter Estimation, 
\begin{align}
	f_{\Theta| X}(\theta|x) &= \frac{f_{\Theta,X}(\theta,x)}{f_X(x)} \\
		&= \frac{f_{X|\Theta}(x|\theta)f_\Theta(\theta)} {\int f_{X|\Theta}(x|\theta)f_\Theta(\theta) d\theta}
\end{align}
% mathematical convenient, need to be flat and 'uninformative'.
conjugate priors, prior G for data H, posterior is G.\\
% 两个参数t,k, fix一个,看作另一个的分布,在定了一个初值后,代入交替分布一定步数(e.g. 100)之后sample出的的pair(t_i, k_i)作为后验的参数分布
Gibbs Sampling\\
% 即Var(t^) 估计t^的方差,越小越有效
Mean squared error; (asymptotic) relative efficiency.\\
% lower bound of 估计,若等于称为(渐进)有效
Thm, Cramer-Rao Inequality.\\
% 基于表达式T的条件概率不依赖theta, 即获得了关于分布的所有知识
def, sufficient, a statistic $T(X_1,...,X_n)$ is sufficient for $\theta$ if conditional distribution of $X_1,...,X_n$ given $T=t$ does not depend on $\theta$ for any value $t$.\\
% 直观上,拆分成scale放大部分g和分布shape部分h,在给定t的条件概率即约束占比与不含x的scale无关
Thm, A necessary and sufficient condition for sufficient $T$, is that joint pdf as form
\begin{align}
	f(x_1,...,x_n|\theta) = g[T(x_1,...,x_n),\theta] h(x_1,...,x_n)
\end{align}
% 显然,对theta 偏导只剩关于T
corollay, MLE is a function of sufficient $T$.\\
% 给定最佳划分下的期望估计比当前估计要严格更优,除非当前估计已经是T的函数,所以不是sufficient statistic T的函数的估计都可以improve
Rao-Blackwell Theorem, sufficient T, $\tilde{\theta}=E[\widehat{\theta}|T]$, then for all $\theta$,
\begin{align}
	E(\tilde{\theta}-\theta)^2 \leq E(\widehat{\theta}-\theta)^2
\end{align}
equality only when $\tilde{\theta}=\widehat{\theta}$, which is function of T.
\end{document}