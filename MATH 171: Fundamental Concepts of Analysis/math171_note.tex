%%%%%%%%%%%%%%%%%%%%%%%%%%%%%%%%%%%%%%%%%
% Short Sectioned Assignment
% LaTeX Template
% Version 1.0 (5/5/12)
%
% This template has been downloaded from:
% http://www.LaTeXTemplates.com
%
% Original author:
% Frits Wenneker (http://www.howtotex.com)
%
% License:
% CC BY-NC-SA 3.0 (http://creativecommons.org/licenses/by-nc-sa/3.0/)
%
%%%%%%%%%%%%%%%%%%%%%%%%%%%%%%%%%%%%%%%%%

%----------------------------------------------------------------------------------------
%	PACKAGES AND OTHER DOCUMENT CONFIGURATIONS
%----------------------------------------------------------------------------------------

\documentclass[paper=a4, fontsize=11pt]{scrartcl} % A4 paper and 11pt font size

\usepackage[T1]{fontenc} % Use 8-bit encoding that has 256 glyphs
\usepackage{fourier} % Use the Adobe Utopia font for the document - comment this line to return to the LaTeX default
\usepackage[english]{babel} % English language/hyphenation
\usepackage{amsmath,amsfonts,amsthm} % Math packages

\usepackage{sectsty} % Allows customizing section commands
\allsectionsfont{\centering \normalfont\scshape} % Make all sections centered, the default font and small caps
\usepackage{url}
\usepackage{fancyhdr} % Custom headers and footers
\pagestyle{fancyplain} % Makes all pages in the document conform to the custom headers and footers
\fancyhead{} % No page header - if you want one, create it in the same way as the footers below
\fancyfoot[L]{} % Empty left footer
\fancyfoot[C]{} % Empty center footer
\fancyfoot[R]{\thepage} % Page numbering for right footer
\renewcommand{\headrulewidth}{0pt} % Remove header underlines
\renewcommand{\footrulewidth}{0pt} % Remove footer underlines
\setlength{\headheight}{13.6pt} % Customize the height of the header

\numberwithin{equation}{section} % Number equations within sections (i.e. 1.1, 1.2, 2.1, 2.2 instead of 1, 2, 3, 4)
\numberwithin{figure}{section} % Number figures within sections (i.e. 1.1, 1.2, 2.1, 2.2 instead of 1, 2, 3, 4)
\numberwithin{table}{section} % Number tables within sections (i.e. 1.1, 1.2, 2.1, 2.2 instead of 1, 2, 3, 4)

\setlength\parindent{0pt} % Removes all indentation from paragraphs - comment this line for an assignment with lots of text
\def \cov {\text{Cov}}
\def \var {\text{Var}}

\title{MATH 171: Fundamental Concepts of Analysis}
% Foundations of Mathematical Analysis, by Richard Johnsonbaugh and W.E. Pfaffenberger, Dover 2010 edition.
\author{sogapalag}

\date{\normalsize\today}

\begin{document}

\maketitle
% bound -> sup; asure no 'hole'
completeness axiom\\
lemma, $\mathbb{N}\times \mathbb{N}$ is countable.\\
% corollary Q=p/q countable.
Thm, union of countable famlity of countable sets is countable.\\
% bounded partial sum, a; decreasing to 0, b; sum ab conv.\\
% corollary, alternating test
Dirichlet's test.\\
% nonnegative, if conv, by row=by column.
% if sum |a| conv, a conv, and by row=by column.
% permutation, rearrangement.
% 若绝对收敛,则下标置换,依然绝对收敛。
% cauchy product 
double series.\\
% collection开 若覆盖 闭,则可有限覆盖
% <-> {R^n, compact <-> closed and bounded }
Heine-Borel Theorem\\
% dxy <= dxz + dzy; 即没有虫洞
metric space\\
% R^\infty 幅长收敛 sum a^2
$l^2$\\
% R^n point converge
Thm, converge $R^n$ iff converge each $a_i$\\
% 封闭的,即含所有极限点
def, closed, all limit points in. i.e. iff $\overline{X}= X$.\\
% note, M, emptyset are closed and open.
def, open, all points have open ball.\\
Thm, open $X$ iff closed $X^c$.\\
def, compact, every open cover has a finite subcover.\\
Thm, metric space, compact iff $\forall$ sequence, has convergent subsequence.\\
% no 'missing' point
def, complete, every cauchy converge in.\\
Thm, every metric space has a completion.\\
totally bounded\\
% forall y\in M, \forall e, dxy<e
def, dense $\overline{X}=M$\\
Thm, countable intersections dense open in complete, dense.\\
% 闭包的内部为空集
def, nowhere dense, $(X^-)^\circ=\emptyset$. note $Y^{\circ}=Y^{c-c}$, so equiv, $X^{-c-}=M$.\\
def, of first category, countable union of nowhere dense sets.\\
def, of second category, not first category.\\
Baire Category Theorem: if nonempty complete metric space, second category.\\
% 可以被countable open 总长任意epsilon覆盖
measure zero. countable union of measure zero is measure zero.\\
almost everywhere, fails only measure zero.\\
Thm, bounded $f$ on $[a,b] $F-intergrable, iff $f$ continuous on $[a,b]$ almost everywhere.\\
% f在phi_n正交基下的表示 sum <f,phi_n>phi_n
Fourier series\\
Bessel's Inequality\\
def, Complete orthonormal set, $\forall x$, $x= \sum_{i=1}^\infty\langle x,x_n\rangle x_n$.\\
% <f,p_n> -> 0
Riemann-Lebesgue Lemma, $f\in\mathcal{R}[a,a+2\pi]$:
\begin{align}
	\lim_{n\rightarrow \infty} \int_a^{a+2\pi}f(t)\sin {nt} dt=0=\lim_{n\rightarrow \infty} \int_a^{a+2\pi}f(t)\cos {nt} dt
\end{align} 
Dirichlet kernel\\
Riemann Localization Theorem\\
differentiable at $x$, $\Rightarrow$ Lipschitz condition at $x$, $\Rightarrow$ Fourier series converges at $x$.\\
% (Riemann 可积 L2 norm), 三角展开完备
Thm, The trigonometric set is complete in $\mathcal{R}[a,a+2\pi]$.\\

\section{Lebesgue Integral}
% almost everywhere (a.e.)
% 扩展Riemann积分
extended $\overline{\mathbb{R}} = \mathbb{R}\cup \{\infty,-\infty\}$\\
$\sigma$-algebra.\\
def, measurable, $f^{-1}(V)\in \mathcal{M}, \forall \text{open} V\in[-\infty,\infty]$.\\
% sup int s_n du, 0<=s_n<=f
simple function.\\
Lebesgue's Dominated Convergence Theorem\\
Thm, $\mathcal{L}^2(X,\mathcal{M},\mu)$ is Hilbert space.\\
Thm, The trigonometric set is complete in $\mathcal{L}^2[a,a+2\pi]$.\\



\end{document}